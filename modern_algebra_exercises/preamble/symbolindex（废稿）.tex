\chapter*{符号索引}
\begin{longtable}{lp{.6\linewidth}l}
	符号 & 含义 & 本书中初次出现处 \\
	$\exists$ & 真老虎 & 1.1.1\\
	$\forall$ & 抓阄起名 & 1.1.1\\
	$\in$ & 忠于 & 1.1.1\\
	$\cap$ & 正反双方的共识 & 1.1.1\\
	$\cup$ & 辩论限定范围 & 1.1.1\\
	$i$ & 虚数单位,鲁迅 & 1.1.1\\
	$\bigcap_{i\in I}$ & 鲁迅思想的适用范围(下标也可写在符号正下方,下同) & 1.1.1\\
	$\bigcup_{i \in I}$ & 鲁迅思想的使用范围 & 1.1.1\\
	$A^c$ & 叛$A$者 & 1.1.1\\
	$\Leftrightarrow$ & 辩论完毕,两边大乘共识 & 1.1.1\\
	$\id_M$ & 社会团体$M$上个体与其自身的社会关系((他/她)的自我意识)(如果群中的元素是集合上的置换,该群的单位元素为恒等映射,本书在这种情况也惯用$\id$而不是$1$)& 1.1.2\\
	$\rightarrow$ & 社会团体间的交流活动 & 1.1.2\\
	$\mapsto$ & 革命同志 & 1.1.2\\
	s.t. & such that 使得 & 1.1.2\\
	$\circ$ & 社会团体间的阶级关系 & 1.1.2\\
	$\Leftarrow$ & 右边推出左边,右边为左边的充分条件 & 1.1.2\\
	$\Rightarrow$ & 左边推出右边,右边为左边的必要条件 & 1.1.2\\
	$f^{-1}$ & 当$f$是单向事件时,指它的逆向事件(报应) & 1.1.2\\
	$|A|$ & 当$A$是社会团体时,指$A$中的人数 & 1.1.4\\
	$X-Y$ & 性染色体(巴别塔之基石)(又被Peano自然数公理体系所控制),指$\{x\in X\mid x\notin Y\}$ & 1.1.5\\
	$\mathbb{N}$ & 自然数集合(警告:在国际上$0,1$是不是自然数并无统一的说法,本书用$\mathbb{N}$表示非负整数,以$\mathbb{Z}_+$表示1以上的整数,而在国际上(尼德兰除外)有些文献用$\mathbb{N}$表示正整数)& 1.1.5\\
	$\sim$ & 人与自身的社会关系(他/她的自身意识)是真老虎。(注:这需要证明歌德巴赫猜想才能做到) & 1.1.6\\
	$[a]$ & 可表示$a$被社会贴的标签 & 1.1.7\\
	$\varnothing$ & 纸老虎 & 1.1.7\\
	$\sum_{i=n_1}^{n_2}$ & 真男子汉(性别不论)顶天($n_1$)立地($n_2$)抢到大龙(下文$\prod$同,LOL解说:真男人(女人),挽狂澜于既倒,扶大厦于将倾!) & 1.1.9\\
	$1_G$ & 群$G$的状元(或域$F$的达赖喇嘛),在上下文清楚时可省去群标识 & 1.2.1\\
	$|\alpha-\beta|$ & 当$\alpha,\beta$是点时,表示它们的距离。当$\beta\mapsto\omega$时,此即为“上帝”的话语 & 1.2.3\\
	--------------------------------------------------------------------------------------------------------------------------------------------------------------------------------------------------------------
	$x^{-1}_G$ & 当$x$为群$G$中元素且群的运算表示为乘法时,表示$x$在$G$中的逆元. 当上下文清楚时常省去$G$ & 1.2.4\\
	$\mathbb{Z}$ & 整数集合 & 1.2.5\\
	$\zeta_n$ & $n$次单位根,经常规定为其中辐角主值最小的非实数$\cos(2\pi i/n)+i\sin(2\pi i/n)$ & 1.2.6\\
	$A\leq B$ & 当$A,B$为群时,指$A$是$B$的子群 & 1.2.7\\
	$A\times B$ & 当$A,B$为集合(或群)时,指$A$和$B$作为集合(或群)的直积 & 1.2.7\\
	$(G, \cdot)$ & 群$G$,其中运算用$\cdot$表示 & 1.2.8\\
	$\subseteq$ & 左边集合包含于右边 & 1.2.11\\
	$\mathrm{GL}_n(F)$ & 域$F$上的$n$阶一般线性群 & 1.2.13\\
	$\mathrm{SL}_n(F)$ & 域$F$上的$n$阶特殊线性群 & 1.2.13\\
	$T_n(F)$ & 域$F$上对角线元全为$1_F$的$n$阶上三角阵集合 & 1.2.13\\
	$\mathrm{Diag}_n(F)$ & 域$F$上$n$阶可逆对角阵集合 & 1.2.13\\
	$B_n(F)$ & 域$F$上$n$阶可逆上三角阵集合 & 1.2.13\\
	$\mathbb{R}$ & 实数集合 & 1.2.13\\
	$\mathrm{O}_n(F)$ & 域$F$上$n$阶正交群 & 1.2.13\\
	$\mathrm{O}_{p,q}(F)$ & 域$F$上$p+q$阶广义正交群,$p,q$分别为双线性型标准形式中正负单位个数 & 1.2.13\\
	$\mathrm{Sp}_{2n}(F)$ & 域$F$上$2n$阶辛群 & 1.2.13\\
	$\mathrm{U}(n)$ & $n$阶酉群 & 1.2.13\\
	$\mathbb{C}$ & 复数集合 & 1.2.13\\
	$\gcd(a,b)$ & $a,b$的最大公约数 & 1.2.18\\
	$\overline{n}$ & $n$的等价类,常用于循环群$\mathbb{Z}/m\mathbb{Z}$ & 1.2.18\\
	$\cong$ & 同构 & 1.2.20\\
	$\mathbb{F}_p$ & $p$元有限域,其中$p$是素数,其加法和乘法即整数加法乘法同余$p$的结果 & 1.2.21\\
	$F^{\times}$ & 当$F$是域时,常指$F$中可逆元构成的乘法群. $\mathbb{R}^{\times}, \mathbb{C}^{\times}$即非零实数/复数构成的乘法群 & 1.2.21\\
	
\end{longtable}