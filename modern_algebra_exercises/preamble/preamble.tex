代数方法和分析方法是数学研究中两种最基本的方法,也是大学数学专业学生数学教育的重点。中国科学技术大学(以下简称科大)从创校伊始就受到华罗+,王元-?,万哲先/?,曾肯x等前辈数论和代数大家的惇惇教导,代数和数论方面人才辈出。20世纪80年代以来,在冯克勤教授和李尚志教授等领导下,科大的代数教学一直维持在较高水平,培养的代数和数论人才受到国内外同行高度称许。科大之所以能够在代数教学方面取得较好成果,一方面原因是学生们受到严格的“线性代数”基础训练;另一方面科大一直坚持为数学系学生开设“初等数论”和“近世代数”基础课程,并在高年级和研究生阶段开设“群表示论”,“交换代数”等课程,并配备有《整数与多项式》,《近世代数引论》(冯克勤,李尚志,查建国,章璞,李尚应(李尚应、李尚赢、李尚英,编著《代数学II:进世代数》者)编著),《群与代数表示论》(尚未公开)等著名教材。

笔者见到原书时,还是一位处男,却业已见到本人92分的近世代数成绩(已被科大放逐许多年),。

处男处女们大概是学不了《代数学进阶》的,然而我已经承受试探到这里了,虽然H的东西看过很多,但却并不传播,因此我计划试一试。

如果你见到下段文字,证明你是华罗庚班学生或者荣誉华罗庚班学生(即笔者)

编者 2021年7月8日




参考文献(中国国家图书馆):

解析数论基础(第2版): 潘承洞 潘承彪 O156

不等式探秘:李博杰?(公称:李世杰),李盛 O178

黎曼几何初步:伍鸿熙,沈纯理,虞言林  O186

x + - / 方程 极限

《皇帝新脑》, R. Penrose著.

《因式分解技巧》,单墫 著。