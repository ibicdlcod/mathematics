\section{正规子群与商群}
\subsection{}
令$G=\{(a,b)|a\in\mathbb{R}^{\times}, b\in\mathbb{R}\}$. 乘法定义为
$$(a,b)(c,d)=(ac,ad+b)$$
则$K=\{(1,b)|b\in\mathbb{R}\}$为$G$的正规子群且$G/K\cong\mathbb{R}^{\times}$.

\zm{
	令$g=(a,b)$则$gK=\{(a,ab_k+b)\mid b_k\in\mathbb{R}, a\neq0\}=\{(a,c)\mid c\in\mathbb{R}, a\neq 0\}$.
	$Kg=\{(a,b+b_k)\mid b_k\in\mathbb{R}, a\neq 0\}=\{(a,c)\mid c\in\mathbb{R}, a\neq 0\}$,故对任何$g$有$gK=Kg$,$K$为$G$的正规子群. 且$h: gK\rightarrow R^{\times}, \{(a,c)\mid c\in\mathbb{R}, a\neq 0\}\mapsto a\;(a\neq 0)$为一一对应.
	
	$g_1Kg_2K=(a_1,c_1)(a_2,c_2)\quad(c_1,c_2\in\mathbb{R})
	=(a_1a_2,a_1c_2+c_1)$
	故$h(g_1Kg_2K)=a_1a_2$,$h$是同态,故为同构. $G/K\cong\mathbb{R}^{\times}$
}

\subsection{}
证明行列式为正的实矩阵组成的$G=\mathrm{GL}_n(\mathbb{R})$的子集$H$构成一个正规子群,并描述商群$G/H$.

\zm{
	由行列式在矩阵乘法中的性质知$H$对乘法和逆封闭,故$H$是子群. $\forall h\in H$有$hH=H=Hh$,$\forall h\notin H$有$hH=G-H=Hh$,故$H$为$G$的正规子群. 令$G-H=J$,则$J=AH, A=
	\begin{pmatrix}
	0 & 1 & \cdots & \cdots \\
	1 & 0 & \cdots & \cdots \\
	\vdots & \vdots & \ddots & \vdots \\
	\cdots & \cdots  & \cdots  & 1\\
	\end{pmatrix}$为第二类初等矩阵. $A^2=I$,$G/H\cong\{1, -1\}^{\times}\cong\mathbb{Z}/2\mathbb{Z}$.
}

\subsection{}
设$G$为群,$N\leq M\vartriangleleft G$.
\subsubsection{(1)}
若$N\vartriangleleft G$则$N\vartriangleleft M$.

\zm{
	$N\vartriangleleft G \Leftrightarrow\forall g\in G, gN=Ng$.
	$N\vartriangleleft M \Leftrightarrow\forall g\in M, gN=Ng$.
	前者为后者充分条件.
}

\subsubsection{(2)}
若$N\vartriangleleft M$是否一定$N\vartriangleleft G$?

\jie
否,考虑交换群$A_4$,$B=\{\id, (12)(34),(13)(24),(14)(23)\}\cong K_2$,$C=\{\id, (12)(34)\}\cong \mathbb{Z}/2\mathbb{Z}$,则$C\vartriangleleft B, B\vartriangleleft A_4$,但$C$不是$A_4$的正规子群.

\subsection{}
\subsubsection{(1)}
试证:群$G$的中心$Z(G)$是$G$的正规子群.

\zm{
	因为$Z(G)$中元素各自成一个共轭类,故它是共轭类之并,只需证明$Z(G)$是子群.
	
	$\forall z_1,z_2\in Z(G), \forall g\in G, g^{-1}z_1z_2^{-1}g=g^{-1}z_1gg^{-1}z_2^{-1}g=g^{-1}z_1g(g^{-1}z_2g)^{-1}=z_1z_2^{-1}$,
	故$z_1z_2^{-1}\in Z(G)$,$Z(G)$是子群.
}
\subsubsection{(2)}
群$G$的指数为$2$的子群一定是$G$的正规子群.

\zm{
	$(G:H)=2$,则$G$的陪集分解为$G=H\sqcup gH=H\sqcup Hg\quad(g\notin H)$,故$gH=Hg$对$g\notin H$成立,从而也对一切$g\in G$成立,$H$是正规子群.
}

\subsection{}
试证直积群$G\times G^{\prime}$的子集$G\times 1$是一个与$G$同构的正规子群,且$G\times G^{\prime}/G\times 1\cong G^{\prime}$.

\zm{
	两个同构请读者自证. $\forall g=(g_1,g_2)\in G\times G^{\prime},
	g^{-1}(G\times 1)g=\{(g^{-1}hg,g_2^{-1}g_2)\}\quad(h\in G)$
	\\$=\{(h^{\prime}, 1)|h^{\prime}=g^{-1}hg\in G\}$
	\\$=G\times 1$.
}

\subsection{}
若$G/Z(G)$是循环群,则$G$为阿贝尔群.

\zm{
	令$G/Z(G)=\langle a\rangle Z(G)$,$\forall g_1,g_2\in G$,有$g_1=a^{n_1}z_1, g_2=a^{n_2}z_2$,其中$z_1, z_2\in Z(G)$.
	$g_1g_2g_1^{-1}g_2^{-1}=a^{n_1}z_1a^{n_2}z_2z_1^{-1}a^{-n_1}z_2^{-1}a^{-n_2}$.
	
	由于$z_1,z_2,z_1^{-1},z_2^{-1}\in Z(G)$与$G$中一切元素交换,上式$=a^{n_1}a^{n_2}a^{-n_1}a^{-n_2}z_1z_1^{-1}z_2z_2^{-1}=1$.
	
	故$z_1z_2=z_2z_1$.
}

\subsection{}
设$G_i\quad(1\leq i \leq n)$为$n$个群. 则(1)$Z(\prod_{i=1}^{n}G_i)=\prod_{i=1}^{n}Z(G_i)$,(2)$\prod_{i=1}^{n}G_i$为阿贝尔群$\Leftrightarrow \forall i, g_i$为阿贝尔群.

\Proofbyintimidation

\subsection{}
设$G$为群.
\subsubsection{(1)}
对于$x\in G$,证明映射$\sigma_x: g\mapsto xgx^{-1}$是$G$的自同构,$\sigma_x$称为{\heiti
内自同构}.

\zm{
	易验证$\sigma_x^{-1}: g\mapsto x^{-1}gx$是$\sigma_x$的唯一逆映射,故$\sigma_x$是双射.
	
	$\sigma_x(g_1)\sigma_x(g_2)=xg_1x^{-1}xg_2x^{-1}=xg_1g_2x^{-1}=\sigma_x{g_1g_2}$,故$\sigma$是同态,故为自同构.
}

\subsubsection{(2)}
令$I(G)$为所有$\sigma_x: x\in G$构成的集合,试证$I(G)$为$\Aut(G)$的子群,称为{\heiti 内自同构群}.

\zm{
	只需证$I(G)$对乘法和逆封闭即可. $\sigma_{x_1}\sigma_{x_2}^{-1}(g)=\sigma_{x_1}(x_2^{-1}gx_2)=x_1x_2^{-1}gx_2x_1^{-1}=\sigma_{x_1x_2^{-1}}(g)$.
}

\subsubsection{(3)}
证明$I(G)\cong G/Z(G)$.

\zm{
	构造同态$\varphi: G\rightarrow I(G); x\mapsto \sigma_x$,显然它是满射. 若$x\in\ker(\varphi)$则$I(x)=\id\Leftrightarrow \forall g, xgx^{-1}=g
	\Leftrightarrow \forall g, xg=gx\Leftrightarrow x\in Z(G)$,故$\ker\varphi=Z(G)$,由第一同构定理即得结论.
}

\subsection{}
线性代数

\subsection{}
设$f: G\rightarrow H$为同态,$M\leq G$. 试证$f^{-1}(f(M))=KM, K=\ker f$.

\zm{
	由$f(K)=1$知$f(KM)=f(M)$. 只需证$\forall g\notin KM, f(g)\notin f(M)$. 事实上,此时$\forall m\in M, gm^{-1}\notin K, f(gm^{-1})\neq 1$,由消去律$f(g)=f(gm^{-1})f(m)\neq f(m)$对所有$m\in M$成立.
}

\subsection{}
设$M,N\vartriangleleft G$.
\subsubsection{(1)}
若$M\cap N=\{1\}$则$\forall a\in M, b\in N, ab=ba$.

\zm{
	令$g=aba^{-1}b^{-1}$,则$g=(aba^{-1})b^{-1}=b_1b^{-1}\in N$,$g=a(ba^{-1}b^{-1})=aa_1\in M$,故总有$g=1, ab=ba.$
}
\subsubsection{(2)}
在此基础上,若$MN=G$则$G\cong M\times N$.

\zm{
	令$f: G\rightarrow MN; g=ab\mapsto (a,b),\quad (a\in M,b\in N)$,$G=MN$导致$G$中任何元素都有像,我们只需证明$f$是良好定义的,即像$(a,b)$唯一.
	
	事实上,$g=a_1b_1=a_2b_2\Rightarrow a_1a_2^{-1}=b_1^{-1}b_2$,左边属于$M$,右边属于$N$,故两边均等于$1$,即$a_1=a_2, b_1=b_2$,$(a,b)$唯一确定. 
	
	$g_1g_2=a_1b_1a_2b_2=a_1a_2b_1b_2(\because ab=ba\forall a\in M, b\in N)=(a_1a_2)(b_1b_2)$,故$f$为同态且$\ker f=1$,故$G\cong M\times N$.
}

\subsection{}
设$N\vartriangleleft G$,$g$是$G$的任一元素,若$g$的阶和$|G/N|$互素,则$g\in N$.

\zm{
	$g$的阶为$m, g\in bN$则$(bN)^m=g^mN^m=1\cdot N^m=N$,得$(bN)^m=N$,故$bN$在$G/N$中的阶$n$整除$m$,但$n$整除$|G/N|$,故$n$是$g$的阶和$|G/N|$的公约数,而两者互素,只能$n=1, bN=N$,即$g\in N$.
}

\subsection{}
证明非阿贝尔群的自同构群不是循环群.

\zm{
	反证法,若$\Aut(G)$是循环群,则$I(G)$为它的子群,从而也是循环群。由{\heiti 习题}\textbf{1.4.8},$G/Z(G)\cong I(G)$也是循环群,由{\heiti 习题}\textbf{1.4.6},$G$是阿贝尔群,矛盾.
}

\subsection{}
线性代数