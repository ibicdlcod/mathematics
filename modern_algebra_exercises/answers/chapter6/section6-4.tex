\section{方程的根式可解性}
\subsection{}
设$p,q$为不同的素数,证明:
\subsubsection{(1)}
$p$的幂次阶群是可解群.

\zm{
	设$|G|=p^n$,当$n=0$时,平凡群是可解群,当$n=1$时,$G$是素数阶循环群,它显然是可解群.
	
	若$n<k$时结论成立,则若$|G|=p^k$,由{\heiti 命题}\textbf{2.42},$|G|$的中心非平凡,故$|Z(G)|=p^r\;(1\leq r\leq k)$,$|G/Z(G)|=p^{k-r}\;(0\leq k-r\leq k-1)$,且$Z(G)\vartriangleleft G$,故要么$G=Z(G)$,此时$G$是阿贝尔群,故为可解群,要么$Z(G)$和$G/Z(G)$的阶都满足归纳假设,它们都是可解群,故由{\heiti 命题}\textbf{6.44(1)},$G$也是可解群.
}
\subsubsection{(2)}
$pq$阶群是可解群.

\zm{
	不妨设$p<q$,则由{\heiti 命题}\textbf{2.56(1)},$G$的西罗$q$-子群$H$是$G$的正规子群,它是可解群. $G/H$的阶为$p$,故它也是可解群,故由{\heiti 命题}\textbf{6.44(1)},$G$也是可解群.
}

\subsubsection{(2)}
$p^2q$阶群是可解群.

\zm{
	由{\heiti 命题}\textbf{2.56(2)},存在两种情况:
	
	(i) $G$的西罗$q$-子群$H$是$G$的正规子群,它是可解群. $G/H$的阶为$p^2$,故它也是可解群.
	
	(ii) $G$的西罗$p$-子群$H$是$G$的正规子群,它的阶为$p^2$,故是可解群. $G/H$的阶为$q$,故它也是可解群.
	
	由{\heiti 命题}\textbf{6.44(1)},$G$也是可解群.
}

\subsection{}
对于群$G$,定义$G^{(1)}=G^{\prime}$为$G$的换位子群,并归纳定义$G^{(i+1)}=G^{(i)\prime}$,证明$G$是可解群当且仅当存在$n\geq 1$,$G^{(n)}=1$.

\zm{
	由{\heiti 命题}\textbf{2.69},对任意群$H$有$H^{\prime}\vartriangleleft H$,且$H/H^{\prime}$为阿贝尔群,故$G\vartriangleright G^{(1)}\vartriangleright \cdots \vartriangleright G^{(n)}$是可解列,当它终结于$\{1\}$时$G$可解.
	
	反之,若$G$不可解,由$G^{(1)}\vartriangleleft G$知$G^{(1)}$和$G/G^{(1)}$中必有一个不可解,但后者是阿贝尔群,故$G^{(1)}$不可解,同样推理即$G^{(n)}$不可解对一切$n\geq 1$成立,故无论$n$取何正整数$G^{(n)}$不可能等于$\{1\}$.
}

\subsection*{以下适用于6.4.3-6.4.5题.}

设$G$是群. 我们定义$C_1(G)=Z(G)$为$G$的中心,并归纳定义$C_{i+1}(G)$为$Z(G/C_i(G))$在映射$G\rightarrow G/C_i(G)$的原像. 如存在$n\geq 1$使得$C_n(G)=G$,则称$G$为{\heiti 幂零群.}

{\heiti 注记.} 该定义实际定义了$G$的{\heiti 上中心系列},$G$是幂零群即上中心系列终结于$G$,此时它是$1$到$G$的{\heiti 中心系列}(中心系列的定义见文献\cite{wiki:cen})

\jie 我们证明以下事实以供下面几题使用. 我们定义$C_0(G)=\{1\}$,易见这定义是平滑的,即$C_1(G)=Z(G)$也是$Z(G/\{1\})$在映射$\id: G\rightarrow G/\{1\}$的原像.

(i) $C_i(G)$是$G$的正规子群.

\zm{
	$C_1(G)=Z(G)$显然是$G$的正规子群. 若$C_i(G)\vartriangleleft G$,则由$Z(G/C_i(G))\vartriangleleft G/C_i(G)$知$gC_{i+1}(G)g^{-1}=gZ(G/C_i(G))C_i(G)g^{-1}$(请理解为$C_i(G)$的陪集!)$=gC_i(G)Z(G/C_i(G))(gC_i(G))^{-1}C_i(G)=Z(G/C_i(G))C_i(G)=C_{i+1}(G)$,故$C_{i+1}(G)\vartriangleleft G$.
}

(ii) $C_i(G)\geq[C_{i+1}(G),(G)]$,其中$[H,G]$为$H$和$G$的换位子群$hgh^{-1}g^{-1}$(注:等号未必成立,即使$i+1<n$且$G$是幂零群,见文献\cite{wiki:cen}中$C_2\times Q_8$的例子).

\zm{
	考虑$C_{i+1}(G)$中任意元素$u$,则由定义$uC_igC_i(uC_i)^{-1}(gC_i)^{-1}=C_i$对一切$g$成立,故$ugu^{-1}g^{-1}\in C_i$.
}

{\heiti 定义.} $G$的{\heiti 下中心系列}是指$G_0=G$(注:有些文献给出$G_1=G$),递归定义$G_{i+1}=[G_i,G]$. 由于$g[a,b]g^{-1}=[a,g]^{-1}[a,gb]$,故$G_i$是$G$的正规子群. 可以证明当且仅当存在$n$使$G_n=\{1\}$时,$G$是幂零群且上中心系列和下中心系列长度一致,详细证明见文献\cite{groupprops_nilpotency},本书不再赘述.

\subsection{}
幂零群一定是可解群.

\zm{
	$C_n=G, C_{n-1}\geq [G,G]=G_1, C_{n-2}\geq [C_{n-1}, G]\geq [G_1, G]=G_2, \cdots C_0\geq G_n$. 但$C_0=\{1\}$,故$G_n=\{1\}$.
	
	$G_1=[G,G]=G^{(1)}$,若$G^{i}\leq G_i$,则$G^{i+1}=[G^{i}, G^{i}]\leq [G^{i}, G]\leq [G_i, G]=G_{i+1}$,故$G^{k}\leq G_k$对一切$n$成立,$G^{n}\leq G_n=\{1\}$,由{\heiti 习题}\textbf{6.4.2},$G$是可解群.
}

\subsection{}
证明对称群$S_3$和$S_4$是可解群,但不是幂零群.

\zm{
	由于$S_n\;(n\geq 3)$的中心平凡,$C_1(G)=C_2(G)=\cdots C_i(G)=\{1\}$,故$S_n$在$n\geq 3$时不是幂零群.
	
	$A_n$是$S_n$的正规子群,且$S_n/A_n\cong\mathbb{Z}/2\mathbb{Z}$是阿贝尔群. $A_3$是$3$阶循环群,它自身即阿贝尔群,$A_4$有正规子群$K_2: \{\id, (12)(34), (13)(24), (14)(23)\}$,$A_4/K_2$的阶为$3$,$K_2$的阶为$4=2^2$,它们都是阿贝尔群,故$S_3$和$S_4$可解.
}

\subsection{}
设$N$为群$G$的正规子群,如果$N$和$G/N$都是幂零群,$G$是否也是幂零群?

\jie 不一定,取$G=A_4$,$N=K_2$,则$N$,$G/N$的阶分别为$4$和$3$,它们都是阿贝尔群,故幂零,但$A_4$的中心平凡,故不是幂零群.

\subsection{}
试导出三次方程的求根公式:设$F$是特征$0$域,
$$f(x)=x^3-t_1x^2+t_2x-t_3\in F(t_1,t_2,t_3)[x].$$

\jie 作移轴变换$y=x-t_1/3$,则$y^3+py+q=0$,其中$p=-\frac{1}{3}t_1^2+t_2,q=-\frac{2}{27}t_1^3+\frac{1}{3}t_1t_2-t_3$.

采用正文6.2节的记号,$\Delta^2=-((p/3)^3+(q/2)^2)*108$,故根的偶置换$A_3$的不变域为$F(\Delta)$,且$f(x)$的分裂域$K$满足$[K:F(\Delta)]=|A_3|=3, \mathrm{Gal}(K/F(\Delta))=\mathbb{Z}/3\mathbb{Z}$.

故$\mathrm{Gal}(K(\omega)/F(\Delta, \omega))=\{1\}$或$\mathbb{Z}/3\mathbb{Z}$,我们考虑后一种情况,即$K(\omega)/F(\Delta, \omega)$是三次扩张. 令$y_i\;(i=1,2,3)$为$f(x)$的三个根,则由$K/F$是平凡扩张或三次扩张,$K=F(y_1)$,令$\sigma: y_1\mapsto y_2, y_2\mapsto y_3, y_3\mapsto y_1\in \mathrm{Gal}(K(\omega)/F(\Delta, \omega))$,则$\sigma^3=\id$. 令$d_1=y_1+y_2\omega+y_3\omega^2, d_2=y_1+y_2\omega^2+y_3\omega, d_3=y_1+y_2+y_3$,则$\sigma(d_1)=d_1\omega^2, \sigma(d_2)=d_2\omega, \sigma_(d_3)=d_3=0$,且$K=F(\Delta, \omega, d_1)$,$d_1^3\in F(\Delta, \omega)$. 我们来计算$d_1^3$.

注意$\Delta=(y_1-y_2)(y_2-y_3)(y_3-y_1)$,$y_1+y_2+y_3=0, y_1y_2+y_2y_3+y_3y_1=p, y_1y_2y_3=q$.
$d_1^3=(y_1^3+y_2^3+y_3^3+6y_1y_2y_3)+3\omega(y_1^2y_2+y_2^2y_3+y_3^2y_1)+3\omega(y_1^2y_3+y_2^2y_1+y_3^2y_2)
=(y_1+y_2+y_3)^3-3(y_1+y_2+y_3)(y_1y_2+y_2y_3+y_3y_1)-9y_1y_2y_3
+3/2\cdot\omega((y_1+y_2+y_3)(y_1y_2+y_2y_3+y_3y_1)-\Delta+3y_1y_2y_3)
+3/2\cdot\omega^2((y_1+y_2+y_3)(y_1y_2+y_2y_3+y_3y_1)+\Delta+3y_1y_2y_3)
=-9q-\frac{3\omega}{2}\cdot\Delta+\frac{9\omega q}{2}+\frac{3\omega^2}{2}\cdot\Delta+\frac{9\omega^2 q}{2}
=-\frac{27}{2}q-\frac{3\sqrt{3}i}{2}\cdot\Delta
=-\frac{27}{2}q+27\sqrt{(q/2)^2+(p/3)^3}$.

故$d_1=\sqrt[3]{-\frac{27}{2}q+27\sqrt{(q/2)^2+(p/3)^3}}
=3\sqrt[3]{-\frac{q}{2}+\sqrt{(q/2)^2+(p/3)^3}}$,记它为$3\alpha$.

由于$\tau: y_1\mapsto y_1, y_2\mapsto y_3, y_3\mapsto y_2$保持$F(t_1,t_2,t_3)$不变并将$\Delta$映射到$-\Delta$,$d_1$映射到$d_2$,故$d_2^3=-\frac{27}{2}q+\frac{3\sqrt{3}i}{2}\cdot\Delta$,$d_2=3\sqrt[3]{-\frac{q}{2}-\sqrt{(q/2)^2+(p/3)^3}}$,记它为$3\beta$.

解方程$y_1+y_2+y_3=d_3=0, y_1+\omega y_2+\omega^2 y_3=3\alpha, y_1+\omega^2 y_2+\omega y_3=3\beta$,得$y_1=\alpha+\beta, y_2=\omega\alpha+\omega^2\beta, y_3=\omega^2\alpha+\omega\beta$,且$p=y_1y_2+y_2y_3+y_3y_1=(1+\omega+\omega^2)\alpha^2+(1+\omega+\omega^2)\beta^2-3\alpha\beta=-3\alpha\beta$,故$\alpha,\beta$需要满足$\alpha\beta=-p/3$.

\subsection{}
将$\cos 20^{\circ}$和$\cos\frac{360^{\circ}}{7}$表示为根式形式.

\jie $\cos \frac{\pi}{9}$的最小多项式为$x^3-\frac{3}{4}x-\frac{1}{8}$({\heiti 推论}\textbf{5.35}),利用{\heiti 习题}\textbf{6.4.6}的结果,$\alpha_0=\sqrt[3]{\frac{1}{16}+\frac{\sqrt{3}i}{16}}=\sqrt[3]{\frac{\zeta_6}{8}}=\zeta_{18}/2$(取幅角主值最小),$\beta_0=\sqrt[3]{\frac{1}{16}-\frac{\sqrt{3}i}{16}}=\sqrt[3]{\frac{\zeta_6^5}{8}}=\zeta_{18}^{17}/2$(取幅角主值最大),显然$\alpha_0\beta_0=1/4=-p/3$,方程的三个根为$\alpha_0+\beta_0=(\zeta_{18}+\zeta_{18}^{17})/2$,$\alpha_0\omega+\beta_0\omega^2, \alpha_0\omega^2+\beta_0\omega$,第一个根是实数,故它是$\cos 20^{\circ}$,即$\cos 20^{\circ}=\sqrt[3]{\frac{1}{16}+\frac{\sqrt{3}i}{16}}+\sqrt[3]{\frac{1}{16}-\frac{\sqrt{3}i}{16}}=\frac{1}{2}(\sqrt[3]{\frac{1+\sqrt{3}i}{2}}+\sqrt[3]{\frac{1-\sqrt{3}i}{2}})$.

$\cos \frac{2\pi}{7}$的最小多项式由{\heiti 习题}\textbf{5.2.4}为$a^3+1/2a^2-1/2a-1/8=0$. 利用{\heiti 习题}\textbf{6.4.6}的结果,$p=-7/12, q=-7/216$,$t_1/3=-1/6$. $\alpha=\sqrt[3]{\frac{7+21\sqrt{3}i}{432}},\beta=\sqrt[3]{\frac{7-21\sqrt{3}i}{432}}$,由于$\alpha^3,\beta^3$彼此共轭,取$\alpha_0$为幅角主值最小的立方根,$\beta_0$为幅角主值最大的立方根,它们彼此共轭,满足$(\alpha\beta)^3=343/46656=(-p/3)^3$,且$\cos \frac{2\pi}{7}=-1/6+\alpha_0+\beta_0$.

故$\cos \frac{2\pi}{7}=-1/6+\sqrt[3]{\frac{7+21\sqrt{3}i}{432}}+\sqrt[3]{\frac{7-21\sqrt{3}i}{432}}=\frac{1}{6}(-1+\sqrt[3]{\frac{7+21\sqrt{3}i}{2}}+\sqrt[3]{\frac{7-21\sqrt{3}i}{2}})$.

\subsection{}
求下列方程在复数域$\mathbb{C}$中的根:
\subsubsection{(1)}
$x^3-2x+4=0$.

\jie 该方程的有理根只可能为$\pm1,\pm2,\pm4$,经检验$-2$是根,故$x^3-2x+4=(x+2)(x^2-2x+2))$,对后者求根得$x^2-2x+2=(x-(1+i))(x-(1-i))$,故$-2,1+i,1-i$是根.

\subsubsection{(2)}
$x^3-15x+4=0$.

\jie
该方程的有理根只可能为$\pm1,\pm2,\pm4$,经检验$-4$是根,故该多项式可分解为$(x+4)(x^2-4x+1)$,对后者求根得$x^2-4x+1=(x-\sqrt{3}-2)(x+\sqrt{3}-2)$,故$-4,\sqrt{3}+2,2-\sqrt{3}$是根.

\subsubsection{(3)}
$x^4-2x^3-8x-3=0$.

\jie
该方程的有理根只可能为$\pm1,\pm3$,经检验$3$是根,故该多项式分解为$(x-3)(x^3+x^2+3x+1)$.

由卡尔达诺公式考虑三次方程$x^3+x^2+3x+1$的根,此时$p=8/3,q=2/27$,

$\alpha=\sqrt[3]{-1/27+\sqrt{19/27}}, \beta=\sqrt[3]{-1/27-\sqrt{19/27}}$,$\alpha^3,\beta^3$都是实数,取它们的实数立方根得$\gamma_1=1/3(-1+\sqrt[3]{-1+\sqrt{19}}+\sqrt[3]{-1-\sqrt{19}})$,
$\gamma_2=1/3(-1+\sqrt[3]{-1+\sqrt{19}}\omega+\sqrt[3]{-1-\sqrt{19}}\omega^2),\gamma_3=1/3(-1+\sqrt[3]{-1+\sqrt{19}}\omega^2+\sqrt[3]{-1-\sqrt{19}}\omega),
\;(\omega=(-1+\sqrt{3}i)/2=\zeta_3)$. 故$3, \gamma_1,\gamma_2,\gamma_3$是根.

\subsection{}
证明方程$x^p-x-t=0$在$\mathbb{F}_p(t)$上根式不可解,但是多项式$x^p-x-t$在$\mathbb{F}_p(t)$上的伽罗瓦群是循环群,这表明在{\heiti 定理}\textbf{6.42}中$\mathrm{char}F=0$的条件一般是不能去掉的.

\zm{
	令$F=\mathbb{F}_p(t)$,设$K$是$x^p-x-t=0$的分裂域,由{\heiti 习题}\textbf{5.1.18}的证明2,$x^p-x-t$的所有根之差是$\mathbb{F}_p$中常数,即若$\alpha$是$x^p-x-t$的一个根,则另一个根$\beta=\alpha+c, c\in\mathbb{F}_p$,故$K=F(\alpha), [K:F]=p$,由前述证明过程(显然$x^p-x-t$在$\mathbb{F}_p$中无根,故)知$\mathrm{Gal}(K/F)\rightarrow \mathbb{Z}/p\mathbb{Z}$为满同态,但$|\mathrm{Gal}(K/F)|\leq [K:F]=p$,故只能$\mathrm{Gal}(K/F)\cong\mathbb{Z}/p\mathbb{Z}$为循环群,$K/F$为伽罗瓦扩张.
	
	故$K/F$若有根式扩张塔,由$[K:F]=p$只能$K=F(\sqrt[p]{u(t)})$,由于$x^p-u(t)=0$的全部根为$p$重的$\sqrt[p]{u(t)}$,故$\mathrm{Gal}(K/F)=\{1\}$,与$\mathrm{Gal}(K/F)=\mathbb{Z}/p\mathbb{Z}$矛盾,故根式扩张塔不存在,$x^p-x-t$在$\mathbb{F}_p(t)$上根式不可解.
}