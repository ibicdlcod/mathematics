\section{伽罗瓦扩张的一些例子}
\subsection{}
设$F$为域,$c\in F$,$p$为素数.
\subsubsection{(1)}
当$\mathrm{char}F=p$时,证明$x^p-c$在$F[x]$中不可约当且仅当$x^p-c$在$F$中无根.

\zm{
	($\Rightarrow$) 显然.
	
	($\Leftarrow$) 设$b^p=c$,其中$b$是$F$的代数闭包中元素,则$x^p-c=(x-b)^p$,故$b\notin F$. 若$x^p-c=(x-b)^p$在$F[x]$中可约,则$(x-b)^k\;(1\leq k<p)$是它的非平凡因子且$(x-b)^k\in F[x]$,故$b^k\in F$,但$k$与$p$互素,故存在整数$u,v$使得$ku=1+pv$,故$b=b^{ku-pv}=(b^k)^uc^{-v}\in F$,矛盾. 故$x^p-c$在$F[x]$中不可约.
}

\subsubsection{(2)}
当$\mathrm{char}F\neq p$时,证明$F$有$p$个不同的$p$次单位根. 由此证明$x^p-c$在$F[x]$中不可约当且仅当$x^p-c$在$F$中无根.

\zm{
	题目实际所指的是:$F$的代数闭包中存在$p$个不同的$p$次单位根.
	
	令$g(x)=x^p-1$,则$g^{\prime}(x)=px^{p-1}\neq 0$,故$\gcd(g,g^{\prime})=1$,$g$在$F$的代数闭包中没有重根.
	
	我们只需证根的不存在性导致$f(x)=x^p-c$的不可约性即可. 设$f$的一个根为$\alpha$,则$f$的所有根为$g(x)$的所有根的$\alpha$倍,故$f$在$F$的代数闭包中也无重根.
	
	令$x^p-c$在$F$上的分裂域为$K$,则$K/F$为伽罗瓦扩张,类似{\heiti 习题}\textbf{6.2.4}我们知道$\mathrm{Gal}(K/F)\leq\left\{
	\begin{pmatrix}
		k & l\\
		0 & 1
	\end{pmatrix}
	\mid l\in\mathbb{F}_p, k\in\mathbb{F}_p^{\times} \right\}\leq S_p.$(第一个等号不一定成立,因为右边群中元素不一定保持$F$不变),并且由它在$x^p-c$的所有$p$个根上的作用唯一确定.

	若$\mathrm{Gal}(K/F)$中存在$p$阶元,则它在$S_p$中必为$p$-轮换,即它在根上的作用可迁,故所有的根均有相同的$F$上最小多项式,它是所有根的化零多项式且整除$f(x)$,只能是$f(x)$本身($f$无重根),即$f(x)$在$F[x]$上不可约.
	
	若$\mathrm{Gal}(K/F)$中不存在$p$阶元,则它不可能是平凡群(这导致$f(x)$在$F$上分裂,与它无根矛盾),并且(*)不存在$l_1\neq l_2$使得$\sigma_1: \alpha_u\mapsto \alpha_{ku+l_1}, \sigma_2: \alpha_u\mapsto \alpha_{ku+l_2}\in \mathrm{Gal}(K/F)$(若不然,则$\sigma_1^{-1}\sigma_2: \alpha_u\mapsto \alpha_{u+(l_2-l_1)k^{-1}}$为$p$阶元,其中$k^{-1}$是$k$在$\mathbb{F}_p$中的逆,矛盾),作同态$\varphi: \left\{
	\begin{pmatrix}
		k & l\\
		0 & 1
	\end{pmatrix}
	\mid l\in\mathbb{F}_p, k\in\mathbb{F}_p^{\times} \right\}\rightarrow \mathbb{F}_p^{\times}, \begin{pmatrix}
		k & l\\
		0 & 1
	\end{pmatrix}\mapsto k$,由于$\mathbb{F}_p^{\times}$为$p-1$阶循环群,考虑它的子群$\varphi(\mathrm{Gal}(K/F))$,它也是循环群,记其生成元为$k_1$,则$k_1\neq 1$,其原像(由(*),它必然是唯一的)为$\sigma^{\prime}: \alpha_u\mapsto \alpha_{k_1u+l}$,则由(*),$\mathrm{Gal}(K/F)$中不含任何$\langle \sigma^{\prime} \rangle$以外元素,故$\sigma^{\prime}$(考虑它作为$S_p$的元素作用在$x^p-c$的所有根上)的不动点$\alpha_{lm}\;(m(1-k_1)\equiv 1\mod p)$是$\mathrm{Gal}(K/F)$的不动点. 由于$K/F$为伽罗瓦扩张,故$f(x)$的一个根$\alpha_{lm}\in F$,矛盾. 故本情况不存在,必有$\mathrm{Gal}(K/F)$中存在$p$阶元且它在$\alpha_i$上的作用可迁.
}

{\heiti 注记.}\emph{文献}\cite{403963}\emph{给出了一种简单得多的证法,但它不涉及伽罗瓦群,考虑到本题的出现位置,本书并未采用那种证法.}

\subsection{}
试求出库默尔扩张的伽罗瓦群.

\jie 令$m$是最小的满足$\alpha^m\in F$的整数,则由$\alpha^n=a\in F$,读者可自行推出$m\mid n$.

故$\sigma(\alpha^m)=\zeta_n^{im}\alpha^m=\alpha^m$,即$im\equiv 0\mod n$,$i$必须是$n/m$的倍数,故$\mathrm{Gal}(K/F)\cong \mathbb{Z}/m\mathbb{Z}$.

\subsection{}
设$E$为$x^4-2$在$\mathbb{Q}$上的分裂域.
\subsubsection{(1)}
试求出$E/\mathbb{Q}$的全部中间域.

\jie 易验证$E=\mathbb{Q}(i,\sqrt[4]{2})$,记$a=\sqrt[4]{2}$,则$x^4-2$的全部四个根为$a,ia,-a,-ia$. 类似{\heiti 习题}\textbf{6.2.3(1)}我们有$G=\mathrm{Gal}(K/F)\cong D_4=\langle \sigma_a, \sigma_i\rangle$,其中$\sigma_i(i)=-i,\sigma_i(a)=a, \sigma_a(i)=i, \sigma_a(a)=ia$.

$G$的所有非平凡子群如下:

中心$H_1=\{\id, \sigma_a^2\}$. 此时$\sigma_a^2$将$a$变为$-a$,$i$保持不变,故$F_1=\mathbb{Q}(i,\sqrt{2})$是$H_1$的不变域.

其他$2$阶子群: $H_{20}=\{\id, \sigma_i\}, H_{21}=\{\id, \sigma_a\sigma_i\}, H_{22}=\{\id, \sigma_a^2\sigma_i\}, H_{23}=\{\id, \sigma_a^3\sigma_i\}$,此时这四个子群的非单位元在四个根上作用分别为:

$a,ia,-a,-ia\mapsto a,-ia,-a,ia;$

$a,ia,-a,-ia\mapsto ia,a,-ia,-a;$

$a,ia,-a,-ia\mapsto -a,ia,a,-ia;$

$a,ia,-a,-ia\mapsto -ia,-a,ia,a.$

故$F_{20}=\mathbb{Q}(a), F_{21}=\mathbb{Q}(a+ia), F_{22}=\mathbb{Q}(ia), F_{23}=\mathbb{Q}(a-ia)$各是这四个子群的不变域.

克莱茵群子群:$H_{30}=\{\id, \sigma_i, \sigma_a^2, \sigma_a^2\sigma_i\}=H_{20}H_{22}, H_{31}=\{\id, \sigma_a\sigma_i, \sigma_a^2, \sigma_a^3\sigma_i\}=H_{21}H_{23}$,此时$F_{30}=F_{20}\cap F_{22}=\mathbb{Q}(\sqrt{2}), F_{31}=F_{21}\cap F_{23}=\mathbb{Q}(i\sqrt{2})$各是这两个子群的不变域.

最大循环真子群:$H_{4}=\langle \sigma_a \rangle$,此时$F_4=\mathbb{Q}(i)$是该子群的不变域.

以上$F_1,F_{20},F_{21},F_{22},F_{23},F_{30},F_{31},F_4$是$E/\mathbb{Q}$的全部中间域,其中$H_1,H_{30},H_{31},H_4$是正规子群,故$F_1,F_{30},F_{31},F_4/\mathbb{Q}$是伽罗瓦扩张,$H_{20}$与$H_{22}$共轭,$H_{21}$与$H_{23}$共轭,故$F_{20}$与$F_{22}$共轭,$F_{21}$与$F_{23}$共轭.

\subsection{}
设$E$为$x^4-2$在$\mathbb{F}_5$上的分裂域,试求出$E/\mathbb{F}_5$的伽罗瓦群和全部中间域.

\jie 记$F=\mathbb{F}_5$,设$a$是$x^4-2$在$E$上的一个根,显然$a, a^2\notin F$,且$a, 2a, 3a, 4a$是$x^4-2$的根,故它们是$x^4-2$在$E$上的全部根,$E=F(a)$是库默尔扩张,由{\heiti 习题}\textbf{6.3.2}和$a^2\notin F$知$\mathrm{Gal}(E/F)\cong\mathbb{Z}/4\mathbb{Z}\cong F^{\times}$.

该伽罗瓦群只有一个非平凡子群$\mathbb{Z}/2\mathbb{Z}$,由于$\sigma: a\mapsto 4a=-a$是伽罗瓦群中唯一的$2$阶元,故子群$\{\id, \sigma\}$作用下的不变域$F(a^2)$即$E/F$的唯一中间域.

\subsection{}
对于$n=8,9,12$,求出$G=\mathrm{Gal}(\mathbb{Q}(\zeta_n)/\mathbb{Q})$,并列出$G$的全部子群和它们对应的$\mathbb{Q}(\zeta_n)/\mathbb{Q}$的中间域.

\jie 以下记$K=\mathbb{Q}(\zeta_n), F=\mathbb{Q}$.
\subsubsection{(1)}
$n=8$,$G=(\mathbb{Z}/8\mathbb{Z})^{\times}\cong K_2$为克莱茵群,其$2$阶元为$\sigma_{3,5,7}, \sigma_i:\zeta_8\mapsto\zeta_8^i$. 易验证$\sigma_3$固定$\sqrt{2}i=\zeta_8+\zeta_8^3$,$\sigma_5$固定$i=\zeta_8^2$,$\sigma_7$固定$\sqrt{2}=\zeta_8+\zeta_8^7$. 故中间域$F_3\supseteq F(\sqrt{2}i), F_5\supseteq F(i), F_7\supseteq F(\sqrt{2})$,易见$\zeta_8$在右边三个域上的最小多项式分别为$(x^2-\sqrt{2}ix-1),(x^2-i),(x^2-\sqrt{2}x+1)$(请读者自证不可约性),故$K$在右边三个域上的指数都是$2$,故它们是各非平凡子群$\{\id, \sigma_3\},\{\id, \sigma_5\},\{\id, \sigma_7\}$的不变域.
\subsubsection{(2)}
$n=9$,$G=(\mathbb{Z}/9\mathbb{Z})^{\times}\cong\mathbb{Z}/6\mathbb{Z}$,其非平凡子群为唯一的$2$阶和唯一的$3$阶子群,$2$阶元为$\sigma_8$,$3$阶元为$\sigma_4,\sigma_7$,$6$阶元为$\sigma_2,\sigma_5$,其中$\sigma_i: \zeta_9\mapsto \zeta_9^i$.

对$2$阶子群$\{\id,\sigma_8\}$,它固定$\zeta_9+\zeta_9^8\in\mathbb{R}$不变,故中间域$F_1\supseteq F(\zeta_9+\zeta_9^8)$,后者是$\mathbb{R}$的子域,故不是$K$,且$\zeta_9$在$F(\zeta_9+\zeta_9^8)$上有化零多项式$x^2-(\zeta_9+\zeta_9^8)x+1$,若它可约,则$\zeta_9\in F(\zeta_9+\zeta_9^8)$,矛盾,故$[K: F(\zeta_9+\zeta_9^8)]=2$,$F(\zeta_9+\zeta_9^8)=F_1$就是$\{\id,\sigma_8\}$的不变域.

对$3$阶子群$\{\id, \sigma_4, \sigma_7\}$,它固定$\zeta_9^3=\omega$不变. 故中间域$F_2\supseteq F(\omega)$. $\zeta_9$在$F(\omega)$上的化零多项式为$f(x)=x^3-\omega=(x-\zeta_9)(x-\zeta_9\omega)(x-\zeta_9\omega^2)$,若它在$F(\omega)$上可约,则它有一次因式,$\zeta_9\omega^i\;(i=0,1,2)\in F(\omega)$,$\zeta_9\in F(\omega)$,$K=F(\omega)$,但$[F(\omega):F]=2$(读者请自证),与$[K:F]\geq|\mathrm{Gal}(K/F)|=6$矛盾,故$f(x)$在$F(\omega)$上不可约,是$\zeta_9$的最小多项式,$[K:F(\omega)]=3$,$F(\omega)=F_2$就是$\{\id,\sigma_4,\sigma_7\}$的不变域.
\subsubsection{(3)}
$n=12$,$G=(\mathbb{Z}/12\mathbb{Z})^{\times}\cong K_2$为克莱茵群,其$2$阶元为$\sigma_{5,7,11}, \sigma_i:\zeta_{12}\mapsto\zeta_{12}^i$.

对$2$阶子群$\{\id, \sigma_{11}\}$,它固定$\zeta_{12}+\zeta_{12}^{11}\in\mathbb{R}$不变,故中间域$F_1\supseteq F(\zeta_{12}+\zeta_{12}^{11})$,后者是$\mathbb{R}$的子域,故不是$K$,且$\zeta_{12}$在$F(\zeta_{12}+\zeta_{12}^{11})$上有化零多项式$x^2-(\zeta_{12}+\zeta_{12}^{11})x+1$,若它可约,则$\zeta_{12}\in F(\zeta_{12}+\zeta_{12}^{11})$,矛盾,故$[K: F(\zeta_{12}+\zeta_{12}^{11})]=2$,$F(\zeta_{12}+\zeta_{12}^{11})=F_1$就是$\{\id,\sigma_{11}\}$的不变域.

对$2$阶子群$\{\id, \sigma_7\}$,它固定$\zeta_{12}^2=\omega+1$不变,故中间域$F_2\supseteq F(\omega)$. $\zeta_{12}$在$F(\omega)$上的化零多项式为$f(x)=x^2-(\omega+1)=(x-\zeta_{12})(x-\zeta_{12}\omega)$,若它在$F(\omega)$上可约,则它有一次因式,$\zeta_{12}\omega^i\;(i=0,1)\in F(\omega)$,$\zeta_{12}\in F(\omega)$,$K=F(\omega)$,但$[F(\omega):F]=2$(读者请自证),与$[K:F]\geq|\mathrm{Gal}(K/F)|=4$矛盾,故$f(x)$在$F(\omega)$上不可约,是$\zeta_{12}$的最小多项式,$[K:F(\omega)]=2$,$F(\omega)=F_2$就是$\{\id,\sigma_7\}$的不变域.

对$2$阶子群$\{\id, \sigma_5\}$,它固定$\zeta_{12}^3=i$不变,故中间域$F_3\supseteq F(i)$.
$\zeta_{12}$在$F(i)$上的化零多项式为$f(x)=x^2-ix-1$,但显然$\zeta_{12}\notin F(i)$,故$f(x)$在$F(i)$上不可约,$[K:F(i)]=2$,$F(i)=F_3$就是$\{\id,\sigma_5\}$的不变域.

\subsection{}
设$n\geq 2$为正整数,证明$\mathbb{Q}(\zeta_n)\cap\mathbb{R}=\mathbb{Q}(\zeta_n+\zeta_n^{-1})$.

\zm{
	题意要求的集合必在复共轭的不变域中,类似正文的讨论(复共轭未必是伽罗瓦群中唯一$2$阶元,但我们已经特定了复共轭的不变域,请读者自己完成)即得$\mathbb{Q}(\zeta_n)\cap\mathbb{R}\subseteq \mathbb{Q}(\zeta_n+\zeta_n^{-1})$,又$\zeta+\zeta_n^{-1}\in\mathbb{R}$,故$\mathbb{Q}(\zeta_n)\cap\mathbb{R}\supseteq \mathbb{Q}(\zeta_n+\zeta_n^{-1})$,得到结论.
}

\subsection{}
设$p$是奇素数,$\left(\frac{a}{p}\right)_{\mathrm{Le}}$是勒让德符号,设{\heiti 高斯和}
$$g=\sum_{a\in\mathbb{F}_p}\zeta_p^a\left(\frac{a}{p}\right)_{\mathrm{Le}}.$$
证明:
\subsubsection{(1)}
$\sum_{a\in\mathbb{F}_p}\zeta_p^a=0$.

\zm{
	注意$x^p-1$的次项系数为$0$即可.
}
\subsubsection{(2)}
$g\cdot\overline{g}=p$,其中$\overline{g}$是$g$的复共轭.

\zm{
	$u=g\overline{g}=\sum_{a\in\mathbb{F}_p}\zeta_p^a\left(\frac{a}{p}\right)_{\mathrm{Le}}\sum_{b\in\mathbb{F}_p}\zeta_p^{-b}\left(\frac{b}{p}\right)_{\mathrm{Le}}=\sum_{i=0}^{p-1}g_i\zeta_p^i$,其中$g_i=\sum_{a\in\mathbb{F}_p}\left(\frac{a}{p}\right)_{\mathrm{Le}}\left(\frac{a-i}{p}\right)_{\mathrm{Le}}$.
	
	$g_0$容易计算,因为$a\neq 0$时$\left(\frac{a^2}{p}\right)_{\mathrm{Le}}$由定义等于$1$,$a\neq0$时它等于$0$. 故$g_0=\sum_{a\in\mathbb{F}_p}\left(\frac{a}{p}\right)_{\mathrm{Le}}\left(\frac{a}{p}\right)_{\mathrm{Le}}=\sum_{a\in\mathbb{F}_p}\left(\frac{a^2}{p}\right)_{\mathrm{Le}}=0+\underbrace{1+\cdots+1}_{p-1\text{个}}=p-1$.
	
	由于当$j\neq 0$时$g_1=\left(\frac{j^2}{p}\right)_{\mathrm{Le}}g_1=\sum_{a\in\mathbb{F}_p}\left(\frac{a}{p}\right)_{\mathrm{Le}}\left(\frac{a-1}{p}\right)_{\mathrm{Le}}\left(\frac{j^2}{p}\right)_{\mathrm{Le}}=\sum_{a\in\mathbb{F}_p}\left(\frac{ja}{p}\right)_{\mathrm{Le}}\left(\frac{ja-j}{p}\right)_{\mathrm{Le}}$,由于$j$与$p$互素,$a\mapsto ja$是$\mathbb{F}_p$到自身的双射,故上式$=\sum_{a^{\prime}\in\mathbb{F}_p}\left(\frac{a^{\prime}}{p}\right)_{\mathrm{Le}}\left(\frac{a^{\prime}-j}{p}\right)_{\mathrm{Le}}=g_j$,故$g_1,\cdots,g_{p-1}$都相等.
	
	考虑$\sum_{i=0}^{p-1}g_i=\sum_{i\in\mathbb{F}_p}\sum_{a\in\mathbb{F}_p}\left(\frac{a}{p}\right)_{\mathrm{Le}}\left(\frac{a-i}{p}\right)_{\mathrm{Le}}=\sum_{a\in\mathbb{F}_p}\left(\frac{a}{p}\right)_{\mathrm{Le}}\left(\sum_{i\in\mathbb{F}_p}\left(\frac{a-i}{p}\right)_{\mathrm{Le}}\right)$,由于$a$固定,$i$取$\mathbb{F}_p$上各值时$a-i$也取$\mathbb{F}_p$上不重复的各值,且(读者可自行考虑$\mathbb{F}_p$上的生成元$\sigma$,则二次剩余为$\sigma^2$生成的指数为$2$的子群)勒让德符号有性质$\sum_{l\in\mathbb{F}_p}\left(\frac{l}{p}\right)_{\mathrm{Le}}=0$,故$\sum_{a\in\mathbb{F}_p}\left(\frac{a}{p}\right)_{\mathrm{Le}}\left(\sum_{i\in\mathbb{F}_p}\left(\frac{a-i}{p}\right)_{\mathrm{Le}}\right)=\sum_{a\in\mathbb{F}_p}\left(\frac{a}{p}\right)_{\mathrm{Le}}\cdot 0=0$,即$\sum_{i=1}^{p-1}g_i=-g_0=-(p-1)$,$g_1=g_2=\cdots=g_{p-1}=-1$.
	
	$u=\sum_{i=0}^{p-1}g_i\zeta_p^i=p-1-\sum_{i=1}^{p-1}\zeta_p^i=p-\sum_{i=0}^{p-1}\zeta_p^i=p-0=p$.
}

\subsubsection{(3)}
$\overline{g}=\sum_{a\in\mathbb{F}_p}\zeta_p^{-a}\left(\frac{a}{p}\right)_{\mathrm{Le}}=\sum_{a\in\mathbb{F}_p}\zeta_p^a\left(\frac{-a}{p}\right)_{\mathrm{Le}}=\sum_{a\in\mathbb{F}_p}\zeta_p^a\left(\frac{a}{p}\right)_{\mathrm{Le}}\left(\frac{-1}{p}\right)_{\mathrm{Le}}=g\left(\frac{-1}{p}\right)_{\mathrm{Le}}=(-1)^{(p-1)/2}g$.

故$g^2=g\overline{g}(-1)^{-(p-1)/2}=(-1)^{-(p-1)/2}p=(-1)^{(p-1)/2}$(因$(p-1)/2$是整数),即得结论.

\subsubsection{(4)}
$\mathbb{Q}(\zeta_p)$有唯一的二次子域$K=\mathbb{Q}(g)$.

\zm{
	令$\sigma: \zeta_p\mapsto \zeta_p^r$是$\mathrm{Gal}(\mathbb{Q}(\zeta_p)/\mathbb{Q})\cong(\mathbb{Z}/p\mathbb{Z})^{\times}$的一个生成元,则$\sigma^2:\zeta_p^u\mapsto\zeta_p^{ur^2}$,$r^2$是模$p$的二次剩余,故$\left(\frac{u}{p}\right)_{\mathrm{Le}}=\left(\frac{ur^2}{p}\right)_{\mathrm{Le}}$,即$\sigma^2$保持$g$不变. $\sigma^2$的不变域$K_1$包含$K$(易验证$g\notin\mathbb{Q}$).
	
	另一方面,$\langle \sigma^2 \rangle$是$\mathrm{Gal}(\mathbb{Q}(\zeta_p)/\mathbb{Q})$的唯一指数为$2$的子群,故$[K_1:\mathbb{Q}]=2=[K:\mathbb{Q}]$,故$K_1=K$,$K$是$\mathbb{Q}(\zeta_p)$的唯一二次子域.
}