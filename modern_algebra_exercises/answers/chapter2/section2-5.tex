\section{自由群与群的表现}
\subsection{}
证明或否定:$2$个生成元的自由群同构于两个无限循环群的积.

\jie 否,前者是非阿贝尔群,后者是阿贝尔群.

\subsection{}
设$F$是$x,y$生成的自由群.

\subsubsection{(1)}
证明两个元素$u=x^2$和$v=y^3$生成$F$的一个子群,它同构于$u,v$上的自由群.

\zm{
	$\langle x^2,y^3 \rangle=\{x^{k_1}y^{l_1}\cdots x^{k_s}y^{l_s}, k_i=2n_i>0\text{或}k_1=0, l_i=3m_i>0 \text{或}l_s=0\}$,
	(注:取$s=1,k_1=l_1=0$即得单位元),它到$\langle u,v\rangle$有自然的双射
	$\langle u,v \rangle=\{u^{n_1}v^{m_1}\cdots u^{n_s}y^{m_s}, n_i>0\text{或}n_1=0, m_i>0 \text{或}m_s=0\}$,
	故$\langle u,v \rangle\cong \langle x^2,y^3\rangle$.
}

\subsubsection{(2)}
证明三个元素$u=x^2,v=y^2,z=xy$生成$F$的一个子群,它同构于$u,v,z$上的自由群.

\zm{
	$\langle x^2,y^2,xy \rangle=\{y^{k_0}x^{k_1}y^{k_2}\cdots x^{k_{2n+1}}y^{k_{2n}}, n\geq0, k_0, k_{2n}\geq 0, k_i\geq 1\,(0<i<2n), k_0\equiv 0\mod 2, k_{2i-1}\equiv k_{2i}\mod 2\}$,
	它到$\langle u,v,z\rangle$有自然的双射
	$$\langle u,v,z \rangle=v^{\frac{k_0}{2}}u^{\lfloor\frac{k_1}{2}\rfloor}z^{k_1\;\mathrm{mod}\,2}v^{\lfloor\frac{k_2}{2}\rfloor}\cdots u^{\lfloor\frac{k_{2n-1}}{2}\rfloor}z^{k_{2n-1}\;\mathrm{mod}\,2}v^{\lfloor\frac{k_{2n}}{2}\rfloor}$$
	($\lfloor a\rfloor:=\max\{b\leq a\mid b\in\mathbb{Z}\}$是不超过$a$的最大整数)
	故$\langle u,v,z \rangle\cong \langle x^2,y^2, xy\rangle$.
}

\subsection{}
若$n$为正奇数,求证:$D_{2n}\cong D_{n}\times \mathbb{Z}/2\mathbb{Z}$.

\zm{
	$D_n=\langle \sigma, \tau, \beta\mid \sigma^n=\tau^2=(\sigma\tau)^2=1\rangle$,
	$D_{2n}=\langle \alpha,\tau\mid\alpha^{2n}=\tau^2=(\alpha\tau)^2=1\rangle$.
	
	记$\alpha^{\prime}=(\sigma^{(n+1)/2},\beta)$,其中$\beta\in\mathbb{Z}/2\mathbb{Z}, \beta^2=1$,
	则$\alpha^{\prime k}=(1,1)\Leftrightarrow 2\mid k$且$n\mid (n+1)k/2$,
	即$k=2k_0$且$n\mid (n+1)k_0\;(k_0\in\mathbb{Z})$,然而$n$与$n+1$互素,故$n\mid k_0, 2n\mid k$,即$\alpha^{\prime}$为$2n$阶元.
	
	记$\tau^{\prime}=(\tau, 1)$,则$\tau^{\prime 2}=(1,1)$,由$\tau\sigma\tau=\sigma^{-1}$可得$\sigma^{l}\tau=\tau\sigma^{-l},\forall l\in\mathbb{Z}$.
	于是$(\alpha^{\prime}\tau^{\prime})^2=(\sigma^{(n+1)/2}\tau\sigma^{(n+1)/2}\tau,\beta^2)=(\tau\sigma^{-(n+1)/2}\sigma^{(n+1)/2}\tau, 1)=(\tau^2,1)=(1,1)$.
	
	由于生成关系唯一确定该群,$\varphi: D_n\times\mathbb{Z}/2\mathbb{Z}\rightarrow D_{2n}, \alpha^{\prime}\mapsto \alpha, \tau^{\prime}\mapsto \tau, (1, \beta)\mapsto\alpha^n$是同构. 
}

\subsection{}
若$n\geq 3$,$A_n\times\mathbb{Z}/2\mathbb{Z}$与$S_n$是否同构?

\jie 否,$\mathbb{Z}/2\mathbb{Z}$是$A_n\times\mathbb{Z}/2\mathbb{Z}$的正规子群,若有同构,则它也是$S_n$的正规子群,但由{\heiti 习题}\textbf{2.1.6},这意味着$\mathbb{Z}/2\mathbb{Z}$同构于$\{1\},A_n,S_n$当中的一个,比较阶可得$2=1$或$n!/2$或$n!$,当$n\geq3$时矛盾.

\subsection{}
设$G=G_1\times G_2\times\cdots\times G_n, H\leq G$,问$H$是否一定形如$H=H_1\times H_2\times \cdots \times H_n$,其中$H_i\leq G_i, \forall 1\leq i \leq n$.

\jie 否,令$G=\mathbb{Z}\times\mathbb{Z}$,$H=\langle (1,1)\rangle\subsetneqq G$. 对任意$k\in\mathbb{Z}$有$(k,k)\in H$,若$H=H_1\times H_2$,则$\mathbb{Z}\subseteq H_1, H_2$,得$H=G$,矛盾,故$H$不满足结论.

\subsection{}
设$G_1$和$G_2$是两个非交换单群,证明$G_1\times G_2$的非平凡正规子群只有$G_1$和$G_2$.

\zm{
	设$H\vartriangleleft (G_1\times G_2)$,令$H_1=\{a\mid(a,b)\in H\},H_2=\{b\mid(a,b)\in H\}$,则$H_i$是$G_i$的子群,$H_1\vartriangleleft G_1$,$H_2\vartriangleleft G_2$,这导致$H_1=\{1\}$或$G_1$,$H_2=\{1\}$或$G_2$(请读者自证!)
	
	若$H_1$或$H_2$是平凡群,则$H$是平凡群或$\{1_{G_1}\}\times G_2\cong G_2$或$G_1\times \{1_{G_2}\}\cong G_1$.
	我们考虑$H_1=G_1,H_2=G_2$的情况.
	
	对$\forall b_0\in G_2, \exists a_0\in G_1$ s.t. $(a_0,b_0)\in H$.
	由$H$的正规性,$\forall b_1\in G_2$,
	$(1_{G_1},b_1)(a_0,b_0)(1_{G_1},b_1)^{-1}=(a_0,b_1b_0b_1^{-1})\in H$,
	故$(a_0,b_1b_0b_1^{-1})(a_0,b_0)^{-1}=(1_{G_1},b_1b_0b_1^{-1}b_0^{-1})\in H$.
	由$b_0,b_1$的任意性,$\{1_{G_1}\}\times [G_2,G_2]=\{1_{G_1}\}\times G_2^{\prime}\subseteq H$,但$G_2^{\prime}\vartriangleleft G_2, G_2^{\prime}\neq \{1_{G_2}\}$($G_2$是非交换的),
	故$G_2\cong \{1_{G_1}\}\times G_2\subseteq H$,同理$G_1\cong G_1\times\{1_{G_2}\}\subseteq H$.
	
	由于上述两个群生成整个$G$,我们有$H=G$,综合四种状况可得结论.
}

\subsection{}
证明$455=5\cdot 7\cdot 13$阶群$G$一定是循环群.

\zm{
	$N(7)\equiv 1\mod 7, N(7)\mid 65, N(13)\equiv 1\mod 13,N(13)\mid 35$,得$N(7)=N(13)=1$,西罗$7$-,$13$-子群$H_7\vartriangleleft G, H_{13}\vartriangleleft G$.
	考虑$H_7H_{13}$,由正规性$H_7H_{13}=H_{13}H_7$,故$H_7H_{13}$是$G$的子群,$H_7\vartriangleleft H_7H_{13}$,由$7$与$13$互素,$H_7\bigcap H_{13}=\{1_G\}, |H_7H_{13}|=91$. 由{\heiti 习题}\textbf{2.2.6}, $H_7H_{13}=H_7\rtimes H_{13}$.
	
	我们有同态:$\varphi_1: H_{13}\rightarrow \Aut(H_7)\cong\Aut(\mathbb{Z}/7\mathbb{Z})=(\mathbb{Z}/7\mathbb{Z})^{\times}$,后者阶数为$6$.
	故$\im\varphi_1$的阶数整除$6$(因它是$\Aut(H_7)$的子群)和$13$(因它是$H_{13}$的商群$H_{13}/\ker\varphi_1$),它只能是平凡群,即$H_{13}$在$H_7$上作用平凡,$H_7H_{13}=H_7\rtimes H_{13}=H_7\times H_{13}=\mathbb{Z}/91\mathbb{Z}$,记为$H_{91}$.
	
	$\forall i\in H_7, j\in H_{13}$,由正规性$\forall g\in G, gig^{-1}=i^{\prime}\in H_7, gjg^{-1}=j^{\prime}\in H_{13}$,故$g(ij)g^{-1}=gig^{-1}gjg^{-1}=i^{\prime}j^{\prime}\in H_7H_{13}$,$H_7H_{13}=H_{91}$是$G$的正规子群.  记$G$的任何一个西罗$5$-子群为$H_5$,因$5$与$91$互素,$H_5\bigcap H_{91}=\{1_G\}, |H_5H_{91}|=455=|G|$. 由{\heiti 习题}\textbf{2.2.6}, $G=H_{91}H_5=H_{91}\rtimes H_5$.
	
	我们有同态:$\varphi_2: H_{5}\rightarrow \Aut(H_{91})\cong\Aut(\mathbb{Z}/91\mathbb{Z})=(\mathbb{Z}/91\mathbb{Z})^{\times}$,后者阶数为$(7-1)\times (13-1)=72$.
	故$\im\varphi_2$的阶数整除$72$(因它是$\Aut(H_{91})$的子群)和$5$(因它是$H_{5}$的商群$H_{5}/\ker\varphi_2$),它只能是平凡群,即$H_{5}$在$H_{91}$上作用平凡,$G=H_{91}H_5=H_{91}\rtimes H_5=H_{91}\times H_{5}=\mathbb{Z}/455\mathbb{Z}$,它是循环群.
}

\subsection{}
求(1)圆(2)球(3)圆柱体的对称群.

\jie (1)$\mathrm{O}_2(\mathbb{R})$ (2)$\mathrm{O}_3(\mathbb{R})$ (3)$\mathrm{O}_2(\mathbb{R})\times \mathbb{Z}/2\mathbb{Z}$ ,请读者自行完成证明.

\subsection{}
给定两个水平平面,在顶面有三个点,它们在底面有正投影. 把顶面的三个点与底面的正投影分别用三根不相交的绳子连接起来,且每根绳子与两平面之间的每一个水平面恰好相交一次,这样的绳子称为一个\textbf{$3$-}{\heiti 辫子}. 给定两个$3$-辫子$a,b$,将$b$放在$a$下面连接起来得到一个新的辫子,称为$a$和$b$的乘法. 试证明所有的$3$-辫子构成一个群,并确定它的表现.

\jie $G=\langle a,b\mid aba=bab\rangle$,理由见文献\cite{wiki:001}.

\subsection{}
设$G$由$n$个元素生成,而$G$的子群$A$具有有限指数. 求证:$A$可由$2n(G:A)$个元素生成.

\emph{证明由文献}\cite{3983366}\emph{给出,该处还有人给出了更严格的上界$(n−1)(G:A)+1$.}


\zm{
	设$G=\langle X \rangle, $记$Y=X\bigcup X^{-1}$,则$|Y|\leq 2n$,且$\forall g\in G, g=y_1y_2\cdots y_{r(g)}$,其中$y_i\in Y$,不一定两两不同.
	
	对$A$的所有右陪集$Ax$选定一个代表元,令$f(x)=\left\{
		\begin{matrix}
		1, & x\in A\\
		Ax\text{的代表元}, & x\notin A
		\end{matrix}\right.$
	其中$A$的一个右陪集代表元系$S=\{1,x_1,...,x_{(G:A)-1}\}$为$f(x)$的$(G:A)$个不同值,即$\forall x_1,x_2\in G, Ax_1=Ax_2\Rightarrow f(x_1)=f(x_2)$. 我们记$[x]:=f(x)$. 我们有
	$\forall x,v\in G, x[x]^{-1}=x(ax)^{-1}\;(a\in A)=a^{-1}\in A, [[x]]=[x], [[x]v]=[xv]$.
	
	$\forall h\in A, h=y_1y_2\cdots y_r
	\\=\underbrace{y_1[y_1]^{-1}}_{z_1}\cdot\underbrace{[y_1]y_2[[y_1]y_2]^{-1}}_{z_2}\cdot
	\underbrace{[[y_1]y_2]y_3[[[y_1]y_2]y_3]^{-1}}_{z_3}\cdot\cdots\cdot\underbrace{[[\cdots[[y_1]y_2]\cdots]y_{r-1}]y_r[[[\cdots[[y_1]y_2]\cdots]y_{r-1}]y_r]^{-1}}_{z_r}
	\\\cdot\underbrace{[[[\cdots[[y_1]y_2]\cdots]y_{r-1}]y_r]}_{z_{r+1}}$
	
	而$z_{r+1}=[y_1y_2\cdots y_r]=[h]=1\;(\because h\in A)$,$z_1,z_2,...,z_r\in SY[SY]^{-1}=\{sy[sy]^{-1}\mid s\in S, y\in Y\}\subseteq A$. 因此$SY[SY]^{-1}$生成$A$,它有至多$|S||Y|\leq (G:A)\cdot2n$个元素,故结论成立.
}

\subsection{}
令$G=G_1\times G_2\times \cdots \times G_n$且对任意$i\neq j$,$|G_i|$和$|G_j|$互素. 证明$G$的任意子群$H$都是它的子群$H\bigcap G_i\;(i=1,2,...,n)$的直积.

\zm{
	记$H_i=\{a_i\mid (a_1,...,a_n)\in H\}$,易见$H_i\leq G_i$. 考虑$H_{12}=\{(a_1,a_2)\mid (a_1,...,a_n)\in H\}\leq G_1\times G_2$,对$H_{12}$中第一项的每一个相异值选定一个元素,得$H_{12,1}=\{(a_1,f(a_1))\mid (a_1,a_2)\in H_{12}\}$,则$\varphi: H_{12,1}\rightarrow H_1, (a_1,f(a_1))\mapsto a_1$为双射.
	
	令$N_{12,2}=\{(1,a_{2z})\mid (1,a_{2z})\in H_{12}\}$,则对任意$H_{12}$中第一项相同的元素$x,y$,有$x=(a_1,a_{2x}),y=(a_1,a_{2y}),xy^{-1}=(1,a_{2x}a_{2y}^{-1})\in N_{12,2}$,特别地,$(a_1,f(a_1)a_{2y}^{-1})\in N_{12,2}$,且不同的$a_{2y}$得到$N_{12,2}$中不同元素. 反之,对$N_{12,2}$中元素$a_{2z}$有$(a_1,f(a_1))(1,a_{2z})=(a_1,f(a_1)a_{2z})\in H_{12}$,且不同的$a_{2z}$得到不同的$f(a_1)a_{2z}$,也就是$H_{12}$中不同的第一项为$a_1$的元素. 故$H_{12}=\bigsqcup_{a_i\in H_1}\{(a_1,f(a_1)a_{2z})\mid(1,a_{2z})\in N_{12,2}\}=H_{12,1}\times N_{12,2}$,$|H_1|=|H_{12,1}|\mid |H_{12}|$. 同理(构造$H_{12,2}, N_{12,1}$)$|H_2|\mid|H_{12}|$.
	
	但$|H_1|\mid |G_1|$,$|H_2|\mid |G_2|$,导致$|H_1|$与$|H_2|$互素,即$|H_1||H_2|\mid |H_{12}|\leq |H_1\times H_2|=|H_1||H_2|$,只有$|H_{12}|=|H_1||H_2|, H_{12}=H_1\times H_2$.
	
	由于$|G_1\times G_2|=|G_1||G_2|$与$|G_3|$互素,同理有$H_{12}3=\{(a_1,a_2,a_3)\mid (a_1,a_2,...,a_n)\in H\}=H_{12}\times H_3=H_1\times H_2\times H_3$,继续这一过程可得结论.
}