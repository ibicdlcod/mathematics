\section{西罗定理及其应用}
\subsection{}
若$p$是$|G|$的素因子,则群$G$必有$p$阶元素.

\zm{
	设$H$为$G$的西罗$p$-子群,则$\exists h\neq 1_G\in H$,$h$的阶整除$p^r$记为$p^i\;(i\geq 1)$,则$h^{p^{i-1}}$为$p$阶元.
}

\subsection{}
给出$G=\mathrm{GL}_n(\mathbb{F}_p)$的一个西罗$p$-子群,并求出$\mathrm{GL}_n(\mathbb{F}_p)$中西罗$p$-子群的个数.

\jie $|\mathrm{GL}_n(\mathbb{F}_p)|=\prod_{i=0}^{n-1}(p^n-p^i)=p^{n(n-1)/2}\prod_{i=1}^{n}(p^i-1)$,由于$p\nmid p^i-1$,故其中$p$部分为$p^{n(n-1)/2}$. 而对角线元全为$1$的上三角阵集合$T=T_n(\mathbb{F}_p)$有$p^{n(n-1)/2}$个元素且是$G$的子群,它是一个满足条件的西罗$p$-子群.

由线性代数可知$N_G(T)=B_n(\mathbb{F}_p)$为可逆上三角阵集合,$N(p)=(G:N_G(T))=|G|/|B|
=|G|/(p^{n(n-1)/2}(p-1)^n)=\prod_{i=1}^n\frac{p^i-1}{p-1}$.

\subsection{}
证明{\heiti 定理}\textbf{2.53}.

\subsubsection{(1)}
设$p^k\mid |G|$,$p$为素数,则$G$中存在$p^k$阶子群.

\zm{
	设$X=\{U\subseteq G\mid |U|=p^k\}$,$|G|=p^rm\;(p\nmid m)$.
	
	则$N=|X|=\binom{mp^r}{p^k}=
	\frac{mp^r(mp^r-1)\cdots(mp^r-p^k+1)}{1\cdot2\cdots p^k}$.
	
	由于$i$与$mp^r-i\;(1\leq i\leq p^k-1)$被$p$整除的次数一样,我们有$p^{r-k}\mid N$且$p^{r-k+1}\nmid N$. 考虑$G$在$X$上的左乘作用,由于$p^{r-k+1}\nmid N$,故必存在一个轨道$O_U$使得$p^{r-k+1}\nmid |O_U|$. 而$U=\bigcup_{x\in U}\Stab_G(U)x$是$\Stab_G(U)$的一些陪集的的并,故$|\Stab_G(U)|=p^s$,$0\leq s\leq k$,又$|\Stab_G(U)||O_U|=|G|=p^rm$,故$|O_U|=p^{r-k}m, |\Stab_G(U)|=p^k$,$\Stab_G(U)$满足条件.
}

\subsubsection{(2)}
$G$中$p^k$阶子群个数模$p$余$1$.

\emph{证明由文献}\cite{1593829}\emph{给出}.

\zm{
	令$|G|=p^km$,其中$m$不一定与$p$互素. $X=\{A\subseteq G\mid |A|=p^k\},
	|X|=\binom{p^km}{p^k}$.
	
	由{\heiti 推论}\textbf{2.3.1},$p^km=|O_A|\cdot|\Stab_G(A)|,\;\forall A\in X$.
	
	{\heiti 事实}\textbf{1.} $1\in A\Rightarrow \Stab_G(A)\subseteq A$. 
	
	理由:$g\in\Stab_G(A)\Rightarrow gA=A\Rightarrow g\cdot 1\in A$.
	
	{\heiti 事实}\textbf{2.} $|\Stab_G(A)| \mid |A|=p^k$.
	
	理由:$\Stab_G(A)A=A\Rightarrow A=\bigsqcup_{i=1}^r \Stab_G(A)x_i$是$\Stab_G(A)$的右陪集之并.
	
	将$X$分解为$G$作用下的轨道之并:$X=\bigsqcup_{i=1}^tO_{A_i}$,若$1\notin A_i$,则取$x\in A_i$有$x^{-1}A_i$满足$1\in x^{-1}A_i$且$O_{A_i}=O_{x^{-1}A_i}$. 因此在讨论轨道时不妨假定$1\in A_i,\;\forall i$.
	
	{\heiti 事实}\textbf{3.} $|O_{A_i}|=m\Leftrightarrow A_i$为$G$的子群.
	
	理由:$1\in A_i$,(事实1)$\Stab_G(A_i)\subseteq(A_i)$. $\therefore |O_{A_i}|=m\Leftrightarrow |\Stab_G(A_i)|=p^k=|A_i|\Leftrightarrow \Stab_G(A_i)=A_i\Leftrightarrow A_i\leq G$.
	
	{\heiti 事实}\textbf{4.} $|O_{A_i}|=m\Leftrightarrow O_{A_i}=\{gA_i\mid g\in G\}$.
	
	理由:($\Leftarrow$) $|O_{A_i}|=(G:A_i)=m$
	($\Rightarrow$) $|O_{A_i}|=m\Rightarrow A_i\leq G$,又$|A_i|=p^r$,我们有$O_{A_i}=\{gA_i\}$.
	
	故我们有双射$$\{\mbox{元素数为$m$的轨道}\}\longleftrightarrow\{\mbox{阶为$p^k$的子群}\}$$,又$|O_{A_i}|\neq m\Rightarrow \Stab_ G(A_i)\subsetneqq A_i$. 故$|\Stab_G(A_i)|=p^{k-u},\;(u\geq 1)$,故此时$|O_{A_i}|=p^{u_i}m, pm\mid |O_{A_i}|$.
	
	因此如有$l$个$p^k$阶子群,则有$l$个元素个数为$m$的轨道,其余轨道有$p^{u_i}m$个元素.
	
	$$\binom{mp^r}{p^k}=|X|=\sum_{i=1}^{l}m+\sum_{j}p^{u_j}m\equiv lm\mod pm.$$
	
	故独立于$G$的结构我们有$\binom{mp^r}{p^k}\equiv lm\mod pm.$,取$G_0=\mathbb{Z}/p^km\mathbb{Z}$得$l_{G_0}=1$,$|X|\equiv m\mod pm$,故$l\equiv 1\mod p$.
}

\subsection{}
设$G$是$n$阶群,$p$是$n$的素因子,证明$x^p=1$在$G$中解的个数是$p$的倍数.

\zm{
	由{\heiti 定理}\textbf{2.5.3},$G$中有$np+1\;(n\in\mathbb{N})$个$p$阶子群. 若这样的两个子群$P_1,P_2$满足$P_1\cap P_2\supsetneqq \{1\}$,则$1<|P_1\cap P_2| \mid |P_1|=p$,只能$|P_1|=|P_2|=|P_1\cap P_2|=p$,即$P_1=P_2$,故两个不同的$p$阶子群有且仅有$1$为公共元素.
	
	$\therefore|\{x\mid x^p=1\}|=(np+1)(p-1)+1=np^2+p-np-1+1\equiv0\mod p$.
}

\subsection{}
证明$6$阶非阿贝尔群只有$S_3$.

\zm{
	设这样的群为$G$,若$G$中有$6$阶元,则它是循环群,矛盾. 故其中非单位元素阶数为$2$或$3$.
	
	$G$的西罗$3$-子群个数为$\equiv 1\mod 3$且整除$2$,故只能为$1$,$G$中有且只有$2$个$3$阶元,故其他$3$个元素均为$2$阶.
	
	由{\heiti 习题}\textbf{2.3.19},$G/Z(G)$不是循环群,故其阶只能为$6$,即$Z(G)=\{1_G\}$,$G\cong G/Z(G)\cong I(G)$,我们只要求得内自同构群$I(G)$即可.
	
	$G$只有$3$个$2$阶子群,同构$\alpha$将这三个子群仍映到三个子群,有$\varphi: I(G)\rightarrow S_3$. 若$\alpha \in \ker\varphi$,则三个子群$\{1_G, a_1\},\{1_G,a_2\},\{1_G,a_3\}$皆在$\alpha$作用下不动,故$1_G,a_1,a_2,a_3$都是$\alpha$的不动点,由同态性,不动点的积和逆也是不动点,故$\alpha$的不动点构成群$H, |H|\geq 4, |H|\mid |G|$,只有$|H|=|G|=6$,即$\ker\varphi=\{\id\}$,$G\cong I(G)\cong I(G)/\ker\varphi\cong\im\varphi\leq S_3$,但$|G|=6$,只有$G=S_3$.
}

\subsection{}
证明$148,200,224$阶群不是单群.

\zm{
	(i) $148=2^2\cdot 37$,由{\heiti 命题}\textbf{2.56(2)}即得.
	
	(ii) $200=2^3\cdot 5^2$. $N(5)\equiv 1\mod 5$且$N(5)\mid 8$,得$N(5)=1$,西罗$5$-子群是正规子群.
	
	(iii) $224=2^5\cdot 7$. $N(2)\equiv 1\mod 2$且$N(2)\mid 7$. 得$N(2)=1$(即得结论)或$7$.
	
	若$N(2)=7$,则记$X_H$为所有西罗$2$-子群的集合.
	
	$G$在$X_H$上的共轭作用诱导同态:$\rho: G\rightarrow S_7$. 若$\ker\rho=\{1\},$则$G\leq S_7$,但$|S_7|=7!=2^4\cdot3^2\cdot5\cdot 7$,$|G|=2^5\cdot 7$,$|G|\nmid |S_7|$,矛盾. 故$\ker\rho\neq\{1\}\vartriangleleft G$.
}

\subsection{}
求$S_4$的自同构群$\Aut(S_4)$.

\jie
$|S_4|=24=2^3\cdot 3$,西罗$3$-子群个数为$1$或$4$.

因$H_1=\{\id,(234),(243)\}, H_2=\{\id,(134),(143)\}, H_3=\{\id,(124),(142)\}, H_4=\{\id,(123),(132)\}$为西罗$3$-子群,因此它们是$S_4$全部的西罗$3$-子群.

讨论$S_4$的任何$6$阶子群$M$,它的西罗$3$-子群$H$指数为$2$,一定是正规的,若这$H=H_1$,则若$M$中存在一个把$2,3,4$中一个变为$1$的置换,则与$H_1\vartriangleleft M$矛盾,故$M$是$1$的稳定子群$M_1\cong S_3$的子群,而$M_1$阶数为$6$,所以它就是$M$. 同理,所有的$M$为$M_i=\Stab_{S_4}(i), i=1,2,3,4$. 

考虑$\Aut(S_4)$在$M_i$上的作用:$\varphi:\Aut(S_4)\rightarrow S_4$. 由于$\varphi(I(S_4))=S_4$,故$\im\varphi=S_4$,我们来证明$\ker\varphi=\id_{\Aut}$.

事实上,$\alpha\in\ker\varphi$使得所有$H_i$不动,故使得所有$M_i$不动,对任意对换$(i_1i_2)$,同时属于$M_{i_3},M_{i_4}\;(i_3, i_4\neq i_1,i_2)$的对换有且仅有它,故$\alpha$保持该对换不动,由于$S_4$由对换生成,保持任意对换不动的$\alpha$为恒等映射.

故$\Aut(S_4)\cong\Aut(S_4)/\ker\varphi\cong\im\varphi=S_4$.

\subsection{}
设$N$是有限群$G$的正规子群. 如果$p$和$|G/N|$互素,则$N$包含$G$的所有西罗$p$-子群.

\zm{
	$|G|=p^rm, p\nmid m, p\nmid (G:N)\Rightarrow N=p^rm_1, m_1\mid m$. 故$N$的西罗$p$-子群也是$G$的西罗$p$-子群. 设$A$为$N$的一个西罗$p$-子群,则$G$的所有西罗$p$-子群有形式$\{gAg^{-1}\mid g\in G\}$,而$A\subseteq N\Rightarrow \forall g, gAg^{-1}\subseteq gNg^{-1}=N$,故得结论.
}

\subsection{}
设$G$是有限群,$N$是$G$的正规子群,$P$是$G$的一个西罗$p$-子群. 证明:

\subsubsection{(1)}
$N\cap P$是$N$的西罗$p$-子群.

\zm{
	若$p\nmid |N|$,$N$的西罗$p$-子群为平凡群,且$|P|$与$|N|$互素,由{\heiti 习题}\textbf{1.3.18},$N\cap P$也是平凡群.
	
	若$p\mid |N|$,由西罗第二定理,存在$g$使得$P^{\prime}=gPg^{-1}$使得$P^{\prime}\cap N$是$N$的西罗$p$-子群,其共轭$g^{-1}(P^{\prime}\cap N)g=g^{-1}P^{\prime}g\cap g^{-1}Ng=P\cap N\subseteq N$也是$N$的西罗$p$-子群.
}

\subsubsection{(2)}
$PN/N$是$G/N$的西罗$p$-子群.

\zm{
	$|G|=p^rm, |N|=p^km_1, r\geq k, p\nmid m,m_1, m_1\mid m, |G/N|=p^{r-k}(m/m_1)$,$G/N$的西罗$p$-子群为$p^{r-k}$阶. 而(第二同构定理)$PN/N\cong P/P\cap N$,由(1)知$|P\cap N|=p^k$,$|P/P\cap N|=p^{r-k}$.
}

\subsubsection{(3)}
$N_G(P)N/N\cong N_{G/N}(PN/N)$.

\zm{
	由第二同构定理,只需证$N_G(P)/N_G(P)\cap N\cong N_{G}(P/N\cap P)N$(右边是因映射$h\mapsto hN$诱导$PN/N$与$P/N\cap P$的同构)
	
	令$\varphi: N_G(P)=\{g\mid g^{-1}Pg=P, g\in G\}\rightarrow N_{G}(P/N\cap P)N=\{gN\mid g^{-1}P(N\cap P)g=P(N\cap P)\}=\{gN\mid g^{-1}Pg=P\}$,易见$\varphi$是满射,其核$\ker\varphi=\{n\mid n^{-1}Pn=P,n\in N\}=N_G(P)\cap N$,由第一同构定理即得结论.
}

\subsection{}
令$P_1,P_2,...,P_N$是有限群$G$的全部西罗$p$-子群,若对任意$i\neq j$总有
$$(P_i:P_i\cap P_j)\geq p^r,$$
则$N\equiv 1\mod p^r$.

\zm{
	由$P_i$是$p$-群,$(P_i:P_i\cap P_j)=p^{r+s}, s\geq 0$.
	
	类似西罗第三定理,令$|G|$=$p^um, p\nmid m$. $G$在$P=\{P_1, P_2, ..., P_N\}$上的共轭作用可迁,$N=(G:N_G(P_1))$.
	
	将$P$分解为$P_1$在其上共轭作用的轨道,若$O_{P_i}$仅有一个元素$P_i$,则$P_1\leq N_G(P_i)$,而$P_i\vartriangleleft N_G(P_i)$为$N_G(P_i)$的唯一西罗$p$-子群,故$P_i=P_1$,仅有一个元素的轨道仅有$\{P_1\}$.
	
	对于其他轨道$O_{P_j}$,考虑$N_G(P_j)\cap P_1$,它的阶整除$|P_1|$故是$p$-群,又是$N_G(P_j)$的子群,从而({\heiti 推论}\textbf{2.51})是$N_G(P_j)$的一个西罗$p$-子群的子群,但$P_j\vartriangleleft N_G(P_j)$为$N_G(P_j)$的唯一西罗$p$-子群,故$N_G(P_j)\cap P_1\leq P_j$是$P_1\cap P_j$的子群,$p^r\mid p^r+s=(P_1:P_1\cap P_j)\mid(P_1:N_G(P_j)\cap P_1)$,又$|O_{P_j}|=(P_1:N_{P_1}(P_j))=(P_1:P_1\cap N_G(P_j))$,故$|O_{P_j}|$被$p^r$整除.
	
	综上,$N\equiv |O_{P_1}|+\sum_{j\neq 1,O_{P_j}\text{各异}}|O_{P_j}|\equiv
	1+\sum_{j\neq 1,O_{P_j}\text{各异}}0
	\equiv1\mod p^r$.
}

\subsection{}
试证:若$G$的阶为$n=p^ea, 1\leq a<p, e\geq 1$,则$G$一定有真正规子群.

\zm{
	当$a>1$时,仿照{\heiti 命题}\textbf{2.56(2)}的证明即可.
	
	当$a=1$时,由{\heiti 命题}\textbf{2.42},$G$的中心非平凡,若$G$不是阿贝尔群,则$Z(G)$是$G$的真正规子群. 若$G$是阿贝尔群,当$e>1$时,由{\heiti 定理}\textbf{2.53},对任何$0<k<e$,$G$的任何$p^k$阶子群存在并是$G$的真正规子群.
	
	当$a=e=1$时,$G$是素数阶循环群,故是单群,结论不成立.
}

\subsection{}
令$G$是集合$\Sigma$上的对称群,$P$是$G$的西罗$p$-子群,如果$p^m$整除$|Ga|$,则$p^m$整除$|Pa|$.

\zm{
	$|G|=p^rm_0=|Ga||\Stab_G(a)|$,
	$|P|=p^r=|Pa||\Stab_P(a)|$.
	而$\Stab_P(a)\leq \Stab_G(a)$,所以$|\Stab_P(a)|\mid |\Stab_G(a)|$.
	
	若$|\Stab_G(a)|=p^{k_1}m_1$,有$k_1\leq r-m, m_1\mid m_0$,则$|\Stab_P(a)|=p^{k_2}, k_2\leq k_1$. $|Pa|=p^r/p^{k_2}=p^{r-k_2}$被$p^m$整除.
}

\subsection{}
令$G$是集合$\Sigma$上的对称群,对任意$a\in\Sigma$,设$P$是稳定子群$\Stab_G(a)$的西罗$p$-子群,$\Delta$是轨道$Ga$在$P$作用下的所有不动点集合,证明$N_G(P)$在$G$上的作用是传递的.

\zm{
	若$b\in\Delta$,则$b\in Ga$,$\exists g$ s.t. $b=ga$. 而$P\in\Stab_G(b)$,又$\Stab_G(b)$与$\Stab_G(a)$共轭({\heiti 命题}{2.32(1)}),所以$P$是$\Stab_G(b)$的西罗$p$-子群. 对群$gPg^{-1}$来说,$gPg^{-1}b=gPa$,而$P\subseteq \Stab_G(a)$,故$Pa=a$,$gPa=ga=b$,故$gPg^{-1}\in\Stab_G(b)$,它也是$\Stab_G(b)$的西罗$p$-子群. 故$\exists h\in\Stab_G(b)$,使得$h(gPg^{-1})h^{-1}=P$,故$hg\in N_G(P)$,且$hga=hb=b$,即$N_G(P)$在$\Delta$上可迁.
}

\subsection{}
证明对$24$阶群$G$,$G$的中心平凡则$G$同构于$S_4$.

\zm{
	\subsubsection{(1)}$G$的西罗$3$-子群不能只有$1$个.
	
	反证法,若只有一个这样的子群,则它是正规子群,记为$N_3\cong \mathbb{Z}/3\mathbb{Z}$. 考虑$G$的一个西罗$2$-子群$H_8$,我们有$|H_8|=8$. 由于$N_3H_8$的阶整除$24$且被$3$和$8$整除,故为$24$,即$N_3H_8=G$,类似地,$3$和$8$互素表明$N_3\cap H_8=\{1_G\}$.
	
	因此$N_3,H_8$满足{\heiti 习题}\textbf{2.2.6(2)}中的全部条件,$G\cong N_3\rtimes H_8$,有$\varphi: H_8\rightarrow \Aut(N_3), h\mapsto (\sigma_h: a\mapsto h^{-1}ah)$.
	
	若$h\in Z(H_8)$且$\sigma_h=\id$,则在$G$中
	$(1_N,h)(x_2,y_2)=(1_N\cdot \sigma_h(x_2),hy_2)
	=(x_2\cdot y_2(1_N),y_2h)=(x_2,y_2)(1,h)$对任意$x_2\in N_3, y_2\in H_8$成立,即$(1,h)\in Z(G)$. 故若$h\neq 1_H\in Z(H_8)$则$G$的中心非平凡,矛盾,下证这样的$h$存在.
	
	记$N_3=\{1,b,b^2\}$,$\Aut(N_3)\cong\mathbb{Z}/2\mathbb{Z}=\{\id_N,\rho\}, \rho^2=\id_N$.
	
	\paragraph{(1a)}若$H_8$中存在$4$阶元,记为$a$,则$\{1, a, a^2, a^3\}$在$H_8$中指数为$2$,为正规子群,并且由$4$阶元的共轭也是$4$阶元,有$\forall c\in H_8$有$c^{-1}ac=a$或$c^{-1}ac=a^3$,此时$c^{-1}a^2c=(c^{-1}ac)^2=a^2$或$a^6=a^2$,故$a^2\in Z(H_8)$且由$\varphi(a)=\id_N$或$\rho$得$\varphi(a^2)=\varphi(a)^2=\id_N$. $a^2$是满足条件的$h$.
	
	\paragraph{(1b)}$H_8$中仅有$1,2$阶元,由{\heiti 引理}\textbf{1.66},$H_8$为阿贝尔群,任何元素都在中心中,且$|\im\varphi|\leq 2$导致$|\ker\varphi|\geq 4$,任取$\ker\varphi$中非单位元即为满足条件的$h$.
	
	\subsubsection{(2)}由(1)且$N(3)\equiv 1\mod 3, N(3)\mid 8$得$N(3)=4$. 令$H=\{H_1,H_2,H_3,H_4\}$为$G$中西罗$3$-子群的集合. 由$|G|=|H_i||N_G(H_i)|$得$N_G(H_i)=6$.
	
	又$H_i$彼此共轭,$\forall u\in N_G(H_i), uH_iu^{-1}=H_i$,则$\forall j\in\{1,2,3,4\}, \exists g\in G$使得$gH_ig^{-1}=H_j$,故$gug^{-1}H_j(gug^{-1})^{-1}=gug^{-1}H_jgu^{-1}g^{-1}=guH_iu^{-1}g^{-1}=gH_ig^{-1}=H_j$,即$gug^{-1}\in N_G(H_j)$,故$N_G(H_i)\;i=1,2,3,4$彼此共轭,又$|N_G(H_i)|=6$,我们分情况讨论证明$\bigcap_{i=1}^4N_G(H_i)=\{1_G\}$.
	
	\paragraph{(2a)}
	
	$N_G(H_1)$为阿贝尔群,则由西罗定理$N_G(H_1)$中既存在$2$阶元,也存在$3$阶元,由{\heiti 习题}\textbf{1.3.25(2)},$6\mid d(G)$使得$d(G)=6$,$G$中存在$6$阶元,故为循环群$\mathbb{Z}/6\mathbb{Z}$,所有的$N_G(H_i)$由共轭性也有此结构.
	
	$H_i=\{1,a_i^2,a_i^4\}$是$N_G(H_i)=\{1,a_i,a_i^2,a_i^3,a_i^4,a_i^5\}$的唯一$3$阶子群,
	由于$H_i\cap H_j\neq H_i$,它的阶数不为$3$且整除$3$,只能为$1$,故$H_i\;(i=1,2,3,4)$中$8$个$3$阶元彼此相异(*),而$a_i, a_i^5$的平方各是$a_i^2,a_i^4$,故也彼此相异. 
	
	若$\bigcap_{i=1}^4N_G(H_i)$非平凡,只能$u=a_1^3=a_2^3=a_3^3=a_4^3$. 但这时$u$与$G$内$17$个各异元素交换,$|C_G(u)|\geq 17, |C_G(u)|\mid |G|=24$,只能$C_G(u)=24$,即$u\in Z(G)$,与中心平凡矛盾.
	 
	 \paragraph{(2b)}
	 
	 $N_G(H_1)$为非阿贝尔群,由{\heiti 习题}\textbf{2.4.5}知$N_G(H_i)\cong S_3=\{1,b,b^2,a,ab=b^2a,ab^2=ba\mid b^3=a^2=1\}$,$b,b^2\in H_i$,$N_G(H_i)-H_i$中元素都是$2$阶,且两个不同的$2$阶元素之积属于$H_i-\{1_G\}$.
	 
	 若$N_G(H_i)-H_i$和$N_G(H_j)-H_j$有$2$个以上相同元素,则它们的积又在$H_i-\{1_G\}$中又在$H_j-\{1_G\}$中,由上述(*)得矛盾,故若$\bigcap_{i=1}^4N_G(H_i)$非平凡,则有且只有一个$u$为$N_G(H_i)$的非平凡公共元素.
	 
	 对$\forall g\in G$,$g$的共轭作用得$gN_G(H_i)g^{-1}=N_G(H_{\sigma(i)}), \sigma\in S_4$. 因此$u\in N_G(H_i),\forall i\Rightarrow gug^{-1}\in N_G(H_{\sigma(i)}),\forall i$,因此$gug^{-1}\in\bigcap_{i=1}^4N_G(H_i)$,又$gug^{-1}\neq 1_G$,得$gug^{-1}=u, \forall g\in G$,即$u\in Z(G)$,矛盾.
	 
	 \paragraph{(2c)}
	 综上,$\bigcap_{i=1}^4N_G(H_i)=\{1_G\}$. 考虑$G$的内自同构$I(G)$作用在$H$的置换上,有同态$\varphi: I(G)\rightarrow S_4$,$\sigma \in \ker\varphi \Leftrightarrow \sigma(H_i)=g_{\sigma}H_ig_{\sigma}^{-1}=H_i,  \forall i\Leftrightarrow g_{\sigma}\in \bigcap_{i=1}^4N_G(H_i)\Leftrightarrow g_{\sigma}=1_G\Leftrightarrow \sigma=\id_H$
	 故$\ker\varphi=\id_H$,$G\cong G/Z(G)\cong I(G)\cong I(G)/\ker\varphi\cong \im\varphi\leq S_4$,但$|G|=|S_4|$,只有$G=S_4$.
}

\subsection{}
设$P$是$G$的西罗$p$-子群且$N_G(P)$是$G$的正规子群,证明$P$是$G$的正规子群.

\zm{
	若$P^{\prime}\neq P$与$P$共轭,则(参照{\heiti 习题}\textbf{2.4.14}的证明(2)部分)有$N_G(P)$与$N_G(P^{\prime})$共轭,故$N_G(P^{\prime})=N_G(P)$. 于是$P\vartriangleleft N_G(P), P^{\prime}\vartriangleleft N_G(P)$是$N_G(P)$的两个西罗$p$-子群,它们彼此共轭,与它们的正规性矛盾. 故$P\vartriangleleft G$.
}