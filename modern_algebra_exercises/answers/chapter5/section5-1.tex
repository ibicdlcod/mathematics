\section{域扩张基本理论}
本节中称$\alpha$在域$K$上代数次数为$n$是指$[K(\alpha):K]=n$.

\subsection{}
设$F/K$为域的扩张,$u\in F$是$K$上的奇次代数元素,求证$K(u)=K(u^2)$.

\zm{
	显然我们只需证$u\in K(u^2)$即可. 令$f(x)$是$u$在$K$上的的最小多项式,则$f(u)=0$,将偶数次项移到右边得$ug(u^2)=h(u^2)$,则由$u$的代数次数为奇数知$g(x)\neq 0$,故$u=\frac{h(u^2)}{g(u^2)}\in K(u^2)$.
}

\subsection{}
设$p$为素数,求扩张$\mathbb{Q}(\zeta_p)/\mathbb{Q}$的次数,其中$\zeta_n=e^{\frac{2\pi i}{n}}$为辐角主值最小的$n$次本原单位根. 对一般的$n$,扩张$\mathbb{Q}(\zeta_n)/\mathbb{Q}$的次数是多少?

\jie (i) $\zeta_p$是$p$次分圆多项式({\heiti 例}\textbf{4.34})的根,该多项式在$\mathbb{Q}[x]$中不可约,故是$\zeta_p$的最小多项式,其次数$p-1$即扩张的次数.

(ii) 由{\heiti 习题}\textbf{4.3.4},$x^4+1$在$\mathbb{Q}[x]$中不可约且$\zeta_8$是它的根,故它的次数$4$即扩张的次数.

(iii) 我们利用本节知识证明$n$次分圆多项式$f(x)=\prod_{\gcd(k,n)=1}(x-\zeta_n^k)$是$\mathbb{Q}[x]$中不可约多项式($n$不一定是素数).

\zm{
	$\langle\zeta_n\rangle\cong\mathbb{Z}/n\mathbb{Z}=G$,而$\Aut(G)$为$(\mathbb{Z}/n\mathbb{Z})^{\times}$,且$\mathbb{Q}(\zeta_n)$中的元素均能表示为$G$中元素的线性组合,故$G$的自同构诱导了$\mathbb{Q}(\zeta_n)$到自身的自同构,该同构在$\mathbb{Q}$上的限制必然是恒等映射,令$\varphi\in\Aut(G)$,则$\varphi$将本原单位根$\zeta_n$映射到本原单位根$\zeta_n^k\;(\gcd(n,k)=1)$,由{\heiti 命题}\textbf{5.18}(取$F=\mathbb{Q}, K=\mathbb{Q}(\zeta_n)$),一切$n$次本原单位根有相同的最小多项式$g(x)$,全部$n$次本原单位根都是$g(x)$的根,故$f(x)\mid g(x)$(我们现在尚不确定$f(x)$在$\mathbb{Q}[x]$中)
	
	由于$x^n-1$没有重根且是$\zeta_n$的化零多项式,$g(x)\mid x^n-1$,故$g(x)$无重根,我们只需证明$g(x)$没有任何$f(x)$的根以外根即可。若有这样的根,则该根也是$x^n-1$的根,故为$n$次单位根,令该根为$\zeta_n^r$,则$\gcd(n,r)\neq 1$,且在{\heiti 命题}\textbf{5.19}中取$F=\tilde{F}=\mathbb{Q}, K=\tilde{K}=\mathbb{Q}(\zeta_n)$,最小多项式为$g(x)$,则存在域同构$\varphi_1$使得它保持有理数不变且$\varphi_1(\zeta_n)=\zeta_n^r$,故$\varphi_1(\zeta_n^{n/\gcd{r,n}})=\zeta_n^{rn/\gcd{r,n}}=(\zeta_n^n)^{r/\gcd{r,n}}=1$,但$n/\gcd{r,n}<n$,$\zeta_n^{n/\gcd{r,n}}\neq 1$,与$\varphi_1$保持$1$不变矛盾,故$g(x)$无任何其他根,$g(x)=f(x)$为不可约多项式.
}

这样,扩张$\mathbb{Q}(\zeta_n)/\mathbb{Q}$的次数即$\deg f=|\{k\mid\gcd(k,n)=1\}|=\phi(x)=|(\mathbb{Z}/n\mathbb{Z})^{\times}|$. 

\subsection{}
求元素$u=\sqrt{2}+\sqrt{3}$在域$K$上的极小多项式,其中
\subsubsection{(1)}
$K=\mathbb{Q}$;

\jie $u$是$f(x)=x^4-10x^2+1=0$的根,由{\heiti 命题}\textbf{4.26},有理根只可能为$\pm 1$,代入即得它们都不是根,故$f(x)$若在$\mathbb{Z}[x]$内可约则为$f(x)=(x^2+ax+\pm 1)(x^2+bx\pm 1)$的形式,得$a=-b, ab\pm 2=-10$,即$a^2=8$或$12$,该方程无整数解,故$f(x)$在$\mathbb{Z}[x]$内不可约,由{\heiti 定理}\textbf{4.35},$f(x)$在$\mathbb{Q}[x]$内也不可约,为$u$的最小多项式.

\subsubsection{(2)}
$K=\mathbb{Q}(\sqrt{2})$;

\jie $u$是$f(x)=\sqrt{2}u^2-4u-\sqrt{2}$的根,故$u$的$K$-代数次数最多为$2$. 若$u\in\mathbb{Q}(\sqrt{2})$,则$u-\sqrt{2}=\sqrt{3}\in\mathbb{Q}(\sqrt{2})$,即$\sqrt{3}=a\sqrt{2}+b\;(a,b\in\mathbb{Q})$,即$3=2a^2+b^2+\sqrt{2}ab$,得$ab=0, 2a^2+b^2=3$,$a$或$b=0$,$a^2=\frac{3}{2}$或$b^2=3$,与$a,b$是有理数矛盾,故$u$的$K$-代数次数不为$1$,$f(x)$是$u$的最小多项式.

\subsubsection{(3)}
$K=\mathbb{Q}(\sqrt{6})$.

\jie $u$是$f(x)=u^2-(5+2\sqrt{6})$的根,故$u$的$K$-代数次数最多为$2$.  若$u\in\mathbb{Q}(\sqrt{6})$,则$\sqrt{2}+\sqrt{3}=a\sqrt{6}+b\;(a,b\in\mathbb{Q})$,即$6a^2+b^2+2\sqrt{6}ab=5+2\sqrt{6}$,$6a^2+b^2=5, ab=1$,$b^4-5b^2+6=0$,得$b^2=2$或$3$,与$b$是有理数矛盾,故$u$的$K$-代数次数不为$1$,$f(x)$是$u$的最小多项式.

\subsection{}
证明$\mathbb{Q}(\sqrt{2},\sqrt{3})=\mathbb{Q}(\sqrt{2}+\sqrt{3})$.

\zm{
	我们只需证明$\sqrt{2}$和$\sqrt{3}\in\mathbb{Q}(\sqrt{2}+\sqrt{3})$即可. 令$u=\sqrt{2}+\sqrt{3}$,则$\sqrt{2}=(u^2-9u)/2, \sqrt{3}=-(u^2-11u)/2$.
}

\subsection{}
设$F/K$为域的代数扩张,$D$为整环且$K\subseteq D\subseteq F$,求证$D$为域.

\zm{
	只需证明$D$中任意元素$u$在$D$中可逆即可. 由于$u$在$K$上代数,$f(u)=0$其中$f(x)\in K[x]\subseteq D[x]$,由$f(u)$在$K$上不可约,其常数项不为$0$,故$g(u)u+f_0=0, -f_0^{-1}g(u)u=1$,其中$g(x)\in K[x]\subseteq D[x]$,即$-f_0^{-1}g(u)\in D$为$u$在$D$中的逆.
}

\subsection{}
设$u$属于域$F$的某个扩域,并且$u$在$F$上代数. 如果$f(x)$为$u$在$F$上的最小多项式,则$f(x)$必为$F[x]$中不可约元. 反之,若$f(x)$是$F[x]$中首一不可约多项式,并且$f(u)=0$,则$f(x)$为$u$在$F$上的最小多项式.

\zm{
	(i) 若$f(x)$有非平凡素因子分解$f(x)=g(x)h(x), 1\leq\deg g,h<\deg f$,则$g(u)$或$h(u)=0$,与最小多项式的定义矛盾.
	
	(ii) 令$u$在$F$上的最小多项式为$g(x)$,则$f(x)=g(x)h(x)+r(x)$,其中$\deg r<\deg g$,且$r(u)=f(u)-g(u)h(u)=0-0\cdot h(u)=0$,由最小多项式的定义,只能$r(x)=0$,又$f(x)$不可约,只能$h(x)\in U(F[x])=F-\{0\}$,又$f,g$均首一,只能$h(x)=1, f(x)=g(x)$.
}

\subsection{}
设$K/F$为域扩张,$a\in K$. 若$a\in F(a^m), m>1$,则$a$在$F$上代数.

\zm{
	我们有$a-g(a^m)=0$,由于$m>1$,$f(x)=x-g(x^m)$不是零多项式,即$f(a)=a-g(a^m)=0$,$a$在$F$上代数.
}

\subsection{}
设$K(x_1,x_2,\cdots,x_n)$为$n$元多项式环$K[x_1,x_2,\cdots,x_n]$的商域,若$K(x_1,x_2,\cdots,x_n), u\notin K$,则$u$在$K$上超越.

\zm{
	反证法,若$u=f/g\;(g\neq 0, f,g\in K[x_1,x_2,\cdots,x_n])$在$K$上代数,则$a_r(f/g)^r+a_{r-1}(f/g)^{r-1}+\cdots+a_0=0$其中$a_i\in K\;(0\leq i\leq r)$,故$a_rf^r+a_{r-1}f^{r-1}g+\cdots+a_0g^r=0$,即$f\mid g^r, g\mid f^r$,若$f\neq 0$,则$f$的不可约(自然是素的)因子都是$g$的不可约因子,反之亦然. 故$f\sim g$或$f=0$,$u=f/g\in U(K[x_1,x_2,\cdots,x_n])\cup \{0\}=K$,矛盾.
}

\subsection{}
设$K$为域,$u\in K(x), u\notin K$,证明$x$在$K(u)$上代数.

\zm{
	$u=f(x)$,其中$f(x)$为一次以上$K$-系数多项式,故$f(x)-u$是一次以上$K(u)$-系数多项式,它是$x$在$K(u)$中的化零多项式.
}

\subsection{}
令$K=\mathbb{Q}(\alpha)$,其中$\alpha$是方程$x^3-x-1=0$的一个根,求$\gamma=1+\alpha^2$在$\mathbb{Q}$上的最小多项式.

\zm{
	由{\heiti 命题}\textbf{4.26},$x^3-x-1=0$的有理根只可能为$\pm 1$,代入知$x^3-x-1=0$没有有理根,故$[\mathbb{Q}(\alpha):\mathbb{Q}]$=3.
	
	显然$\mathbb{Q}(\gamma)\subseteq\mathbb{Q}(\alpha)$,故$[\mathbb{Q}(\gamma):\mathbb{Q}]$=3或1. 若$\mathbb{Q}(\gamma)=\mathbb{Q}$,则$1+\alpha^2\in\mathbb{Q}$,$\alpha$在$\mathbb{Q}$上代数次数最多为$2$,矛盾. 故$\gamma$的最小多项式为$3$次.
	
	设$\gamma^3+a\gamma^2+b\gamma+c=0$,则$(7+3a+b)\alpha^2+(5+a)\alpha+(2+a+b+c)=0$,由于$\alpha^2, \alpha, 1$在$\mathbb{Q}$上线性无关,故$7+3a+b=0,5+a=0,2+a+b+c=0$,解得$a=-5,b=8,c=-5$,故$x^3-5x^2+8x-5$是$\gamma$在$\mathbb{Q}$上的最小多项式.
}

\subsection{}
设$a$是正有理数且不是$\mathbb{Q}$中数的平方,证明$[\mathbb{Q}(\sqrt[4]{a}):\mathbb{Q}]=4$.

\zm{
	显然$[\mathbb{Q}(\sqrt{a}):\mathbb{Q}]=2, \mathbb{Q}(\sqrt{a})\subseteq\mathbb{Q}(\sqrt[4]{a})$,并且$(\sqrt[4]{a})^2=\sqrt{a}\in\mathbb{Q}(\sqrt{a})$,故$[\mathbb{Q}(\sqrt[4]{a}):\mathbb{Q}(\sqrt{a})]\leq2$,只需$\sqrt[4]{a}\notin\mathbb{Q}(\sqrt{a})$即可. 若不然,则$\sqrt[4]{a}=b+c\sqrt{a}$,得$=b^4+6b^2c^2a^2+c^4-a+4bc(ac^2+b^2)\sqrt{a}=0$,只能$b=0$或$c=0$,即$a=c^4$或$b^4$,矛盾.
}

\subsection{}
设$u$是多项式$x^3-6x^2+9x+3$的根.
\subsubsection{(1)}
求证$[\mathbb{Q}(u):\mathbb{Q}]=3$.

\zm{
	由艾森斯坦判别法(取$p=3$)得上述多项式在$\mathbb{Q}[x]$中不可约,故$u$的$\mathbb{Q}$-代数次数为$3$.
}
\subsubsection{(2)}
试将$u^4, (u+1)^{-1}, (u^2-6u+8)^{-1}$表示为$1,u,u^2$的线性组合.

\jie $u^3=6u^2-9u-3$.

(i) $u^4=6u^3-9u^2-3u=36u^2-54u-18-9u^2-3u=27u^2-57u-18$.

(ii) $(u+1)(u^2-7u+16)=13$,故$(u+1)^{-1}=(u^2-7u+16)/13$.

(iii) 设$(u^2-6u+8)^{-1}=au^2+bu+c$,则$au^4+(-6a+b)u^3+(8a-6b+c)u^2+(-6c+8b)u+8c=1$,即$-a+c=0, -3a-b-6c=0, -3b+8c-1=0$,解得$c=1/35, a=1/35, b=-9/35$,即$(u^2-6u+8)^{-1}=(u^2-9u+1)/35$.

\subsection{}
设$d\geq 3$为无平方因子的整数,$K=\mathbb{Q}(\sqrt{d})$.
\subsubsection{(1)}
证明$K$中任意元素在$\mathbb{Q}$上的最小多项式是$1$次或$2$次.

\zm{
	$x^2-d$是$\sqrt{d}$在$\mathbb{Q}$上的最小多项式,故$[K:\mathbb{Q}]=2$,对$K$中任意元素$\alpha$有$\mathbb{Q}\subseteq\mathbb{Q}(\alpha)\subseteq K$,故$[\mathbb{Q}(\alpha):\mathbb{Q}]$只能为$1$或$2$,故得结论.
}
\subsubsection{(2)}
设$\mathcal{O}$是$K$中所有在$\mathbb{Q}$上的最小多项式为首一整系数多项式元素的集合,试求$\mathcal{O}$.
 
\jie $\mathbb{Q}[\sqrt{d}]$中元素$u$均有$a+b\sqrt{d}$的形式,其中$a,b\in\mathbb{Q}$. 若最小多项式为一次,则$b=0$,$x-a=0$为$a$的最小多项式,故$u\in\mathbb{Z}$.

若最小多项式为二次,则$b\neq 0$,$x^2+ex+f=0$,其中$e,f\in\mathbb{Z}$. 故$a^2+b^2d+ea+f+(2ab+eb)\sqrt{d}=0$,我们有$a=-e/2, -e^2/4+b^2d\in\mathbb{Z}$,若$e$为偶数,则$a,-e^2/4$为整数,$b^2d$为整数,又$d$没有平方因子,迫使$b$为整数,故$a,b$为整数时满足条件,此时$x^2-2ax+a^2-b^2d=0$为最小多项式.

若$e$为奇数,则$e^2\equiv 1\mod 4$,故$4b^2d=h$其中$h\equiv 1\mod 4$,设$b=p/q\;(\gcd(p,q)=1)$,则$4p^2d=hq^2$,由$p^2,q^2$互素得$q^2\mid 4d$,由于$d$没有平方因子,只能$q^2=4$,即$q=2,p^2d=h$,$p$是奇数,两边模$4$得$d\equiv 1\mod 4$时这样的$p$存在.

由于当$2a,2b$均为奇数,$d\equiv 1\mod 4$时$x^2-2ax+a^2-b^2d$确实为整系数多项式,故当$d\equiv 2,3\mod 4$时$\mathcal{O}=\mathbb{Z}\cup\{a+b\sqrt{d}\mid a,b\in\mathbb{Z}\}$,当$d\equiv 1\mod 4$时$\mathcal{O}=\mathbb{Z}\cup\{a+b\sqrt{d}\mid a,b\in\mathbb{Z}\}\cup\{a+b\sqrt{d}\mid 2a,2b\in\mathbb{Z}-2\mathbb{Z}\}$.

\subsection{}
设$x$是$\mathbb{Q}$上的超越元且$u=x^3/(x+1)$,求$[\mathbb{Q}(x):\mathbb{Q}(u)]$.

\jie 由于$x^3-ux-u=0$,故$\{x^3,x^2,x,1\}$在$\mathbb{Q}(u)$上线性相关,$[\mathbb{Q}(x):\mathbb{Q}(u)]\leq 3$,若$[\mathbb{Q}(x):\mathbb{Q}(u)]<3$,则$x^3-ux-u$不是$\mathbb{Q}(u)$上的不可约多项式,它必有一次因式,即$[\mathbb{Q}(x):\mathbb{Q}(u)]=1$,故我们只需要证明$x\notin\mathbb{Q}(u)$即证明了$[\mathbb{Q}(x):\mathbb{Q}(u)]=3$.

若$x=\frac{f(u)}{g(u)}$,设$\deg f=a, \deg g=b$,则$f(u)(x+1)^a$是$x$的$3a$次多项式,$g(u)(x+1)^b$是$x$的$3b$次多项式,若$a>b$则$xg(u)(x+1)^b(x+1)^{a-b}=f(u)(x+1)^a$,左边为$1+a+2b$次多项式,右边为$3a$次多项式,但$b\leq a-1$,即$1+a+2b\leq 3a-1<3a$,矛盾. 若$a<b$则$xg(u)(x+1)^b=f(u)(x+1)^a(x+1)^{b-a}$,左边为$1+3b$次多项式,右边为$2a+b$次多项式,但$b\geq a+1$,$1+3b\geq 3+2a+b>2a+b$,矛盾. 若$a=b$,则$xg(x)(x+1)^a=f(x)(x+1)^a$,左边为$3a+1$次多项式,右边为$3a$次多项式,矛盾. 故$x\notin\mathbb{Q}(u)$,$[\mathbb{Q}(x):\mathbb{Q}(u)]=3$.

\subsection{}
试写出二元域$\mathbb{F}_2$的一个$2$次不可约多项式$f(x)$. 设$u$是$f(x)$的一个根,写出$\mathbb{F}_2(u)$的全部元素及它们的加法表和乘法表.

\jie $\mathbb{F}_2[x]$中的一次多项式只有$x$和$x+1$,由于$x^2+x+1=(x+1)x+1$,故$x^2+x+1$是不可约多项式,$u^2+u+1=0\Rightarrow u^2=u+1$(请注意$+1$与$-1$并无差别). 故$\mathbb{F}_2(u)=\{0,1,u,u+1\}$,加法表如下:
\begin{center}
	\begin{tabular}{|c|c|c|c|c|}
		\hline
		$+$ & $0$ & $1$ & $u$ & $u+1$\\
		\hline
		$0$ & $0$ & $1$ & $u$ & $u+1$\\
		\hline
		$1$ & $1$ & $0$ & $u+1$ & $u$\\
		\hline
		$u$ & $u$ & $u+1$ & $0$ & $1$\\
		\hline
		$u+1$ & $u+1$ & $u$ & $1$ & $0$\\
		\hline
	\end{tabular}
\end{center}
乘法表如下($0$省略):
\begin{center}
	\begin{tabular}{|c|c|c|c|}
		\hline
		$\times$ & $1$ & $u$ & $u+1$\\
		\hline
		$1$ & $1$ & $u$ & $u+1$\\
		\hline
		$u$ & $u$ & $u+1$ & $1$\\
		\hline
		$u+1$ & $u+1$ & $1$ & $u$\\
		\hline
	\end{tabular}
\end{center}

\subsection{}
设$M/K$为域的扩张,$M$中元素$u,v$分别是$K$上的$m$次和$n$次代数元素,$F=K(u)$,$E=K(v)$.
\subsubsection{(1)}
求证$[FE:K]\leq mn$.

\zm{
	$F$中任意元素均能表示成$1,u,\cdots,u^{m-1}$的线性组合,另一方面$E$中任意元素均能表示成$1,v,\cdots,v^{n-1}$的线性组合,故$FE$中任意元素均能表示成$u^iv^j\;(0\leq i<m, 0\leq j<n)$的线性组合,即$mn$个元素的线性组合,故由线性代数知$[FE:K]$至多为$mn$.
}
\subsubsection{(2)}
如果$\gcd(m,n)=1$,则$[FE:K]=mn$.

\zm{
	由于$F$和$E$都是$FE$的子域,故$m=[F:K]\mid[FE:K], n=[E:K]\mid[FE:K]$,$\lcm(m,n)\mid[FE:K]$,当$\gcd(m,n)=1$时$\lcm(m,n)=mn$,再由(1)可知只能$[FE:K]=mn$.
}

\subsection{}
设$K,L$为域$F$的扩张且均在$F$的某给定代数封闭域中. 称$L$在$F$上{\heiti 线性不相交}于$K$是指$L$中任何$F$-线性无关有限集还是$K$-线性无关集. 即对$L$的任意子集合$\{x_i\mid i\in I\}$如它在$F$上线性无关,则它也在$K$上线性无关.

\subsubsection{(1)}
证明:如$L$在$F$上线性不相交于$K$,则$K$在$F$上线性不相交与$L$.

\zm{
	如$L$在$F$上线性不相交于$K$,任取$L$的一组$F$-基$\{x_i\mid i\in I_L\}$和$K$的一组$F$-基$\{y_j\mid j\in J_K\}$,则$\sum_i b_ix_i=0\;(b_i\in K)\Rightarrow b_i=0,\forall i$,将$b_i$用$K$的$F$-基表示即$\sum_{i,j} c_{ij}x_iy_j=0\;(c_{ij}\in F)\Rightarrow c_{ij}y_j=0,\forall i\Rightarrow c_{ij}=0, \forall i,j$,即$x_iy_j$是$F$-线性无关集.
	
	反之,若$L$在$F$上线性相交于$K$,取$K$-线性相关但$F$-线性无关的集合$\{x_i^{\prime}\mid i\in S\}$,则它可以扩充为一组$F$-基$\{x_i\mid i\in I_L\}$并同样$K$-线性相关,$\sum_i b_ix_i=0$其中$b_i\in K$不全为零,将$b_i$用$K$中$F$-基表示则$\sum_{i,j} c_{ij}x_iy_j=0\;(c_{ij}\in F)$,其中$c_{ij}$不全为$0$,即$x_iy_j$是$F$-线性相关集.
	
	同样,若$K$在$F$上线性相交于$L$,则$x_iy_j$也是$F$-线性相关集,矛盾. 故得结论.
}

\subsubsection{(2)}
设$K$与$L$均是$F$的有限扩张. 证明:$L$在$F$上线性不相交于$K$当且仅当$[KL:F]=[K:F]\cdot[L:F]$.

\zm{
	线性不相交关系$\Leftrightarrow$任取$L,K$的$F$-基,其$[K:F][L:F]$个乘积元素$F$-线性无关,$[KL:F]\geq[K:F]\cdot[L:F]$又因{\heiti 习题}\textbf{5.1.16},$[KL:F]=[K:F]\cdot[L:F]$. 反之,若$L$在$F$上线性相交于$K$,则设$x_i,y_j$各是$L,K$的一组$F$-基,则$x_iy_j$是$F$-线性相关集合,故是少于$[K:F]\cdot[L:F]$个元素$z_k$的线性组合,$KL$上任何元素都是$x_iy_j$的$F$-线性组合,故也是$z_k$的$F$-线性组合,即$[KL:F]<[K:F]\cdot[L:F]$.
}

\subsection{}
设$F$为特征$p$域,$p$为素数,$c\in F$
\subsubsection{(1)}
证明$x^p-x-c$在$F[x]$中不可约当且仅当$x^p-x-c$在$F$中无根.

\zm{
	($\Rightarrow$)显然.
	
	($\Leftarrow$)\emph{证明由文献}\cite{1256120}\emph{给出}.
	
	(证明1)令$f(x)=x^p-x-c$,考虑双射$\varphi: F[x]\rightarrow F[x], x\mapsto x+1$,则$\varphi(f)=f$,故$\varphi$将$f$的不可约因子变作$f$的不可约因子,令$f$在$F[x]$上的因子分解为$f=ug_1g_2\cdots g_n$,由于$f$无根,没有一次因式,因此$n<p$.
	
	由于$\varphi^p: x\mapsto x+p=x$为恒等映射,故$\langle\varphi\rangle\cong\mathbb{Z}/p\mathbb{Z}$,该$p$阶循环群作用在$M=\{g_1,g_2,\cdots,g_n\}$上,有同态$\tau:\mathbb{Z}/p\mathbb{Z}\rightarrow S_n$,由于$n<p$,$p\nmid n!$,$|\im\tau|\mid n!, |\im\tau|\mid p$,只能$|\im\tau|=1$,即$\langle\varphi\rangle$在$M$上作用平凡,任意$f$的不可约因子在$\varphi$下不变.
	
	令$r=\deg g_1$,则$g_1=a_rx^r+a_{r-1}x^{r-1}+\cdots+a_0\;(a_r\neq 0)$,$\varphi(g_1)=a_rx^r+(ra_r+a_{r-1})x^{r-1}+\cdots+b$,即$\overline{r}a_r=\overline{0}$,由于$F$没有零因子,只能$\overline{r}=\overline{0}$,即$p\mid r$,又$1\leq r\leq p$,只能$r=p$,即$g_1=f$,$f$为不可约多项式.
	
	(证明2)\emph{读者应当在学习第六章伽罗瓦理论后再次体会这种证法.}
	
	由于$f$的形式微商$f^{\prime}=-1$,故$\gcd(f,f^{\prime})=1$,$f$无重根. 令$K$为$F$的代数闭包,则$f$在$K$中有$p$个根$\alpha_1,\cdots,\alpha_p$. $L\subseteq K$为$F(\alpha_1,\cdots,\alpha_p)$,记$\mathrm{Gal}(L/F)$为$L$的所有$F$-自同构构成的集合,在其上定义映射复合为乘法运算. 令$\tau: \mathrm{Gal}(L/F)\rightarrow \mathbb{Z}/p\mathbb{Z}, \sigma\mapsto\sigma(\alpha_1)-\alpha_1$,我们证明$\tau$是良好定义的.
	
	由{\heiti 命题}\textbf{5.18},$\sigma(\alpha_1)$也是$f=\sigma(f)$的根,记为$\alpha_i$,则$\tau(\sigma)=\alpha_i-\alpha_1$. 由于$\alpha_i^p-\alpha_i-c=\alpha_1^p-\alpha_1-c=0$,故$\alpha_i-\alpha_1=\alpha_i^p-\alpha_1^p\equiv(\alpha_i-\alpha_1)^p\mod p$,即在$F$中$\alpha_i-\alpha_1$是$x^p-x$的根,但由{\heiti 推论}\textbf{1.63},$F$中所有整数$0,1,\cdots,p-1$都是$x^p-x$的根,故由{\heiti 推论}\textbf{4.25},$x^p-x$在$L$中只有这$p$个根,即$\alpha_i-\alpha_1\in\mathbb{Z}/p\mathbb{Z}$.
	
	由于$p$阶循环群只有平凡子群,若$\im\tau=\mathbb{Z}/p\mathbb{Z}$,则$\mathrm{Gal}(L/F)$将$\alpha_1$映射到$p$个不同值,故$[L:F]\geq|\mathrm{Gal}(L/F)|=p$({\heiti 定理}\textbf{6.8}),又$f$是所有$\alpha_i$的化零多项式,故$[L:F]\leq p$,故$[L:F]=p$,且由$\alpha_i\notin F$,$\alpha_i$的最小多项式为$p$次,故为$f$,即$f$不可约.
	
	若$\im\tau$是平凡群,则$\sigma$保持一切$\alpha_i$不变,故是$L$上的恒等映射,由{\heiti 例}\textbf{6.15},{\heiti 命题}\textbf{6.16}和{\heiti 定理}\textbf{6.24}知$L/F$是伽罗瓦扩张,$[L:F]=|\mathrm{Gal}(L/F)|=1$,即$\alpha_i\in F$,矛盾.
}

\subsubsection{(2)}
若$\mathrm{char}F=0$时,试问(1)中结论是否仍然成立?

\jie 否,例如$F=\mathbb{Q}$,$f=x^5-x-15=(x^2-x+3)(x^3+x^2-2x-5)$是可约多项式,但$f$没有有理根(对每个因式考虑有理根,第一个只能为$\pm3$,第二个只能为$\pm5$,代入即否定).

\subsection{}
\subsubsection{(1)}
设$F$为特征不为$2$的域,证明$F$的每个二次扩张均有形式$F(\sqrt{a})$,其中$a\in F-F^2$.

\zm{
	令$K/F$为二次扩张,任选$\alpha\in K-F$,则$\alpha$的最小多项式为$x^2+bx+c$,故$2\alpha+b=\pm\sqrt{b^2-4c},\alpha\in F(\sqrt{b^2-4c})$,若$b^2-4c\in F^2$,则$\alpha=1/2(\pm\sqrt{b^2-4c}-b)\in F$,矛盾,故$b^2-4c\in F-F^2$,令$a=b^2-4c$即得结论.
}
\subsubsection{(2)}
若$\mathrm{char}F=2$,则(1)的结论是否仍然成立?

\jie 否,考虑{\heiti 习题}\textbf{5.1.15}的结果,由于$\mathbb{F}_2(u)$有$4$个元素,显然$\mathbb{F}_2(u)/\mathbb{F}_2$是二次扩张,但$u^2=u+1\notin \mathbb{F}_2$.

\emph{以下参照文献}\cite{290032}\emph{.}

事实上,令$K=F[t]/(t^2-t+\alpha)$,$\alpha\in F$. 如果$\alpha$不是$x^2-x, x\in F$的形式,那么$t^2-t+\alpha$是$F[t]$中不可约多项式,$K$是域,$K/F$是二次扩张.

\subsection{}
设$K=\mathbb{Q}(\alpha)$是$\mathbb{Q}$的单扩张,其中$\alpha$在$\mathbb{Q}$上代数,证明$|\Aut(K)|\leq [K:\mathbb{Q}]$.

\zm{
	令$\varphi$是$K$的$\mathbb{Q}$-自同构,则对$\alpha$在$\mathbb{Q}$上的最小多项式$f(x)$,$\varphi(\alpha)$仍是$f(x)$的根({\heiti 命题}\textbf{5.18}),并唯一确定$\varphi$,由{\heiti 推论}\textbf{4.25},这样的不同根最多有$\deg f=[K:\mathbb{Q}]$个,即$\Aut(K)$最多有$[K:\mathbb{Q}]$个元素.
	
	{\heiti 注记.} 等号不成立即$f$有重根或者$f$在$K$上不分裂,$\alpha\mapsto$另一个根$\beta$有可能不是$K$的自同构的情形,参见第六章第一节.
}