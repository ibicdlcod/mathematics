\section{尺规作图问题}
\subsection{}
下列哪些量可以尺规作出?

\subsubsection{(1)}
$\sqrt[4]{3+5\sqrt{8}}$;

\jie 可以,因为$\sqrt[4]{a}$即$\sqrt{\sqrt{a}}$,由有理数作加减乘除和开方即能得到.

\subsubsection{(2)}
$\frac{3\sqrt{5}}{\sqrt{7}-4}$;

\jie 可以,理由同上.

\subsubsection{(3)}
$2+\sqrt[5]{7}$;

\jie 不可以,若该量可以尺规作出,则$a=\sqrt[5]{7}$也可以尺规作出,$a$是$x^5-7=0$的根,取$p=7$由艾森斯坦判别法知该多项式在$\mathbb{Q}[x]$中不可约,$[\mathbb{Q}(a):\mathbb{Q}]=5$不是$2$的幂.

\subsubsection{(4)}
$x^5-3x^2+6$的一个根$\alpha$.

\jie 不可以,取$p=3$由艾森斯坦判别法知该多项式在$\mathbb{Q}[x]$中不可约,$[\mathbb{Q}(\alpha):\mathbb{Q}]=5$不是$2$的幂.

\subsection{}
证明可以尺规三等分$45^{\circ}$和$54^{\circ}$角.

\zm{
	若角$\alpha, \beta$可以尺规作出,则$\cos\alpha,\cos\beta,\sin\alpha,\sin\beta$可以尺规作出,故$\cos(\alpha\pm\beta)=\cos\alpha\cos\beta\mp\sin\alpha\sin\beta$也可以尺规作出,即$\alpha\pm\beta$可以尺规作出(也可以由几何知识直接得到该结论).
	
	熟知$\cos60^{\circ}=\frac{1}{2}$可以尺规作出,故给定$45^{\circ}$可以作$15^{\circ}=60^{\circ}-45^{\circ}$,给定$54^{\circ}$可以作$6^{\circ}=60^{\circ}-54^{\circ}$,将该角再三倍即得$18^{\circ}$.
}

\subsection{}
能否尺规作立方体,其体积为原立方体的$2$倍?

\jie 否,题意即尺规作出$a=\sqrt[3]{2}$,由于$a$满足$x^3-2=0$,取$p=2$由艾森斯坦判别法知该多项式在$\mathbb{Q}[x]$中不可约,$[\mathbb{Q}(a):\mathbb{Q}]=3$不是$2$的幂.

\subsection{}
设$3\leq n\leq 10$为正整数,则正$n$边形是否可以尺规作出?

\jie $n=3$:$\cos\frac{2\pi}{3}=-\frac{1}{2}$为可构造数,故正三角形可.

$n=4$:$\cos\frac{2\pi}{4}=0$为可构造数,故正方形可.

$n=5$:$\cos\frac{2\pi}{5}=a$,其中$a$满足$2a^2-1=\cos\frac{4\pi}{5}=\cos\frac{6\pi}{5}=4a^3-3a$,故$4a^3-2a^2-3a+1=(a-1)(4a^2+2a-1)=0$,又$a\neq 1$,故$4a^2+2a-1=0$,$[\mathbb{Q}(a):\mathbb{Q}]\leq 2$,即$a$为可构造数,故正五边形可.

$n=6$:$\cos\frac{2\pi}{6}=\frac{1}{2}$为可构造数,故正六边形可.

$n=7$:$\cos\frac{2\pi}{7}=a$,其中$a$满足$2(2a^2-1)^2-1=\cos\frac{8\pi}{7}=\cos\frac{6\pi}{7}=4a^3-3a$,故$8 a^4 - 4 a^3 - 8 a^2 + 3 a + 1 =(a-1)(8a^3+4a^2-4a-1)=0$,又$a\neq 1$,故$8a^3+4a^2-4a-1=0$,该方程的有理根({\heiti 命题}\textbf{4.26})只能是$\pm 1,\pm1/2,\pm1/4,\pm 1/8$,代入即得该方程没有有理根,从而没有一次因式,从而是$\mathbb{Q}[x]$中不可约多项式,$[\mathbb{Q}(a):\mathbb{Q}]=3$不是$2$的幂,正七边形不能尺规作出.

$n=8$:$\cos\frac{2\pi}{8}=\frac{\sqrt{2}}{2}$,由$2$是可构造数,$\sqrt{2}$也是,故$\frac{\sqrt{2}}{2}$可构造,正八边形可.

$n=9$:若正九边形可尺规作出,则$40^{\circ}$可构造,又$60^{\circ}$可构造,则$20^{\circ}$可构造,与{\heiti 推论}\textbf{5.35}矛盾,故正九边形不能尺规作出.

$n=10$:沿用$n=5$时的$a$,则$\cos\frac{2\pi}{10}=b, 2b^2-1=a, b=\pm\sqrt{\frac{a+1}{2}}$,由$a$可构造推出$b$可构造,故正十边形可以尺规作出.

