\section{唯一因子分解环}
\subsection{}
设$R$为UFD,$a,b,c$为$R$中非零元素. 证明:
\subsubsection{(1)}
$ab\sim \gcd(a,b)\lcm(a,b)$.

\zm{
	若$a$的因子分解为$uc_1c_2\cdots c_n$,$b$的因子分解为$vd_1d_2\cdots d_m$,两者中的非单位重复元素(这里的重复是指相伴)为$c_1,c_2,...,c_l\sim d_1,d_2,...,d_l$(不妨设重新排列后重复的元素居前),则$\gcd(a,b)\sim c_1c_2\cdots c_l$,$\lcm(a,b)\sim c_1c_2\cdots c_l\cdot c_{l+1}c_{l+2}\cdots c_n\cdot d_{l+1}d_{l+2}\cdots d_m$,$ab\sim c_1c_2\cdots c_l\cdot c_{l+1}c_{l+2}\cdots c_n\cdot d_1d_2\cdots d_l\cdot d_{l+1}d_{l+2}\cdots d_m\sim
	c_1c_2\cdots c_l\cdot c_{l+1}c_{l+2}\cdots c_n\cdot d_{l+1}d_{l+2}\cdots d_m \cdot
	c_1c_2\cdots c_l$,故得结论.
}

\subsubsection{(2)}
若$a\mid bc$,$\gcd(a,b)\sim 1$,则$a\mid c$.

\zm{
	由$\gcd(a,b)\sim 1, \gcd(ac,bc)\sim c$. 又$\gcd(a,ac)\sim a$,故
	$\gcd(a,c)\sim\gcd(a,(ac,bc))\sim\gcd((a,ac),bc)\sim\gcd(a,bc)\sim a$,即$a\mid c$.
}

\subsection{}
设$R$为PID. 证明:
\subsubsection{(1)}
$(a)_{\Ideal}\cap(b)_{\Ideal}=(a)_{\Ideal}(b)_{\Ideal}\Leftrightarrow \gcd(a,b)\sim 1$;

\zm{
	$(a)_{\Ideal}\cap(b)_{\Ideal}=(\lcm(a,b))_{\Ideal}$,$(a)_{\Ideal}(b)_{\Ideal}=(ab)_{\Ideal}$;因此题述左侧成立当且仅当$\lcm(a,b)\sim ab$,由{\heiti 习题}\textbf{4.1.1}知这等价于$\gcd(a,b)\sim 1$.
}

\subsubsection{(2)}
方程$ax+by=c$在$R$中有解$(x,y)$的充要条件是$\gcd(a,b)\mid c$.

\zm{
	题述左侧成立即理想$(a,b)_{\Ideal}\supseteq(c)_{\Ideal}$,由于PID上所有理想都是主理想,当然有限生成的理想都是主理想,故PID都是贝祖环,$(a,b)_{\Ideal}\supseteq(\gcd(a,b))_{\Ideal}$,又易验证$(a,b)_{\Ideal}\subseteq(\gcd(a,b))_{\Ideal}$,故二理想相等,$(\gcd(a,b))_{\Ideal}\supseteq(c)_{\Ideal}$的充要条件是$\gcd(a,b)\mid c$.
}

\subsection{}
设$n\geq 3$为无平方因子的整数,$R=\mathbb{Z}[\sqrt{-n}]=\{a+b\sqrt{-n}\mid a,b\in\mathbb{Z}\}$.
\subsubsection{(1)}
证明$2$,$\sqrt{-n}$和$1+\sqrt{-n}$在$R$上均为不可约元.

\zm{
	令$N:R\rightarrow \mathbb{N}, N(a+b\sqrt{-n})=(a+b\sqrt{-n})(a-b\sqrt{-n})=a^2+nb^2$,则
	
	(i) $N(x)=0\Leftrightarrow x=0, N(1)=1$,
	
	(ii) $N(xy)=N(x)N(y)$.
	
	(iii) $N(x)=1\Leftrightarrow x\in U(R)\Leftrightarrow x=\pm 1$.
	
	$N(2)=4$,若$2=xy$,其中$x$不是平凡因子,则$x,y$都不是单位,$N(x)=N(y)=2$,但$a^2+nb^2=2$在$n\geq 3$时没有整数解,矛盾.
	
	$N(\sqrt{-n})=n$,若$\sqrt{-n}=xy$,其中$x$不是平凡因子,则$x,y$都不是单位,$1<N(x)\mid n$,此时由于$n$无平方因子,$N(x)$也没有平方因子且小于$n$,$a^2+nb^2=N(x)$仍然没有整数解,矛盾.
	
	$N(1+\sqrt{-n})=n+1$,若$1+\sqrt{-n}=xy$,其中$x$不是平凡因子,则$x,y$都不是单位,$1<N(x)\mid n+1$,由于$n+1$与$n$互素,故$N(x)$小于$n$,即$x=a+b\sqrt{n}, a^2+nb^2<n$,故$b=0$,$x\in\mathbb{Z}$,同理$y\in\mathbb{Z}$,故$1+\sqrt{n}=xy\in\mathbb{Z}$,矛盾.

	故三者皆是不可约元.
}

\subsubsection{(2)}
证明$\sqrt{-n}$和$1+\sqrt{-n}$在$R$上不能同时为素元.

\zm{
	由于$n\geq 3$,故$n$和$n+1$一奇一偶且大于2,故不能同时为素数.
	
	若$n$不是素数,则存在$n$的非平凡因子$a,b$. 此时$\sqrt{-n}\mid n=\sqrt{-n}\cdot-\sqrt{-n}=ab$,由于$N(a)$,$N(b)$为整数的平方,$N(\sqrt{-n})=n$没有平方因子,且$n^2=a^2b^2$,若$n\mid a^2$,则$1\neq b^2\mid n$,矛盾,故$n\nmid a^2$,由(1)(ii),$\sqrt{n}$也不整除$a$,同理它也不整除$b$,$\sqrt{n}$不是素元.
	
	若$n+1$不是素数,则存在$n+1$的非平凡因子$a,b$. 此时$1+\sqrt{n}\mid n+1=(1+\sqrt{-n})(1-\sqrt{-n})=ab$,若$1+\sqrt{n}\mid a$,则$(1+\sqrt{-n})(c+d\sqrt{-n}=a)$,得$c-nd=a,c+d=0,a=(1+n)c$,即$1+n\mid a$,与$a$是$n+1$的非平凡因子矛盾,同理$1+\sqrt{n}\nmid b$,即$1+\sqrt{-n}$不是素元.
}

\subsection{}
设$p$是$\mathbb{Z}$上的奇素数,$n$为正整数. 证明$x^n-p$是$\mathbb{Z}[i]$上的不可约多项式.

\zm{
}

\subsection{}
证明$x^3+nx+2$对所有$n\neq 1,-3,-5$是$\mathbb{Z}$上的不可约多项式.

\zm{
	若$x^3+nx+2$在$\mathbb{Z}$中可约,则它必有一次因式$(ax+b)$,由余数定理({\heiti 推论}\textbf{4.24}),它必有有理根$-\frac{b}{a}$其中$\gcd(a,b)=1$.
	
	故$b^3+nba^2+2a^3=0$,故$b\mid b^3+nba^2=-2a^3$,$a\mid nba^2+a^3=-b^3$,但$a,b$互素,故$a\mid 1, b\mid 2$,有理根为$\pm 1, \pm 2$,依次带入得$n=1,-3,-5$.
}

\subsection{}
设$R$是整环,证明多项式环$R[x]$是PID当且仅当$R$是域.

\zm{
	($\Leftarrow$)即{\heiti 命题}\textbf{3.40}.
	
	($\Rightarrow$)反证法,若$R$不是域,任取其中不可逆元$a$,考虑理想$(a,x)_{\Ideal, R[x]}$,则它不等于$R$,若它是主理想,则该理想中有常数迫使它的生成元是常数$b$,但$x\in(b)_{\Ideal, R[x]}$,迫使$b$是$R$中单位,与$(b)$是真理想矛盾,故$R[x]$不是PID.
}

\subsection{}
设$R$为整环,$a,b\in R-\{0\}, a\sim b$,求证:
\subsubsection{(1)}
若$a$为不可约元,则$b$也为不可约元;

\zm{
	$b$的非平凡真因子也是$a$的非平凡真因子,故得结论.
}

\subsubsection{(2)}
若$a$为素元,则$b$也为素元.

\zm{
	任给$c,d$,$a\mid cd\Rightarrow a\mid c$或$a\mid d$为条件.
	
	若$b\mid cd$,则$a\mid cd$,故$a\mid c$或$a\mid d$,推出$b\mid c$或$b\mid d$,故得结论.
}

\subsection{}
设$a$为主理想整环$D$中非零元,求证:若$a$为素元,则$D/(a)_{\Ideal, D}$为域;若$a$不是素元,则$D/(a)_{\Ideal, D}$不是整环.

\zm{
	(i) 若$a$为素元,则对任意$b\in D$有$\gcd(a,b)\sim a$或$\gcd(a,b)\sim 1$.
	
	若$\gcd(a,b)\sim a$,则$b\in(a)_{\Ideal, D}$,$b$对应$D/(a)_{\Ideal, D}$中零元素;
	
	若$\gcd(a,b)\sim 1$,则由$D$是贝祖环,存在$x,y\in D$使得$ax+by=1$,此时$b+(a)_{\Ideal, D}$在$D/(a)_{\Ideal, D}$中逆元为$y+(a)_{\Ideal, D}$.
	
	综上,$D/(a)_{\Ideal, D}$为域.
	
	(ii) 若$a$不是素元,则$a$可写作非平凡真因子的积$a=bc$,其中$b,c\notin(a)_{\Ideal, D}$,故$(b+(a)_{\Ideal, D})(c+(a)_{\Ideal, D})=(a)_{\Ideal, D}$,$(b+(a)_{\Ideal, D})$是$D/(a)_{\Ideal, D}$的零因子.
}

\subsection{}
下列哪些环是PID?哪些环是ED?
\subsubsection{(1)}
$\mathbb{Z}[\sqrt{-2}], \mathbb{Z}[\sqrt{-3}].$

\jie (i)$\mathbb{Z}[\sqrt{-2}]$是ED,从而也是PID.

\zm{
	令$\varphi: \mathbb{Z}[\sqrt{-2}]\rightarrow\mathbb{N},
	a+b\sqrt{-2}\mapsto a^2+2b^2.$

	若$\alpha, \beta\in\mathbb{Z}[\sqrt{-2}]$,则$\frac{\alpha}{\beta}=x+y\sqrt{-2}\in\mathbb{Q}[\sqrt{-2}]$,在$\mathbb{Z}$中选择$x_0,y_0$使得$|x-x_0|\leq\frac{1}{2},|y-y_0|\leq\frac{1}{2}$,则
	
	$\alpha=(x_0+y_0\sqrt{-2})\beta+((x-x_0)+(y-y_0)\sqrt{-2})\beta=q\beta+\gamma$,$\varphi(\gamma)=\varphi(\beta)\varphi(x-x_0+(y-y_0)\sqrt{-2})=\varphi(\beta)((x-x_0)^2+2(y-y_0)^2)\leq\frac{3}{4}\varphi(\beta)<\varphi(\beta)$.
	
	故$\mathbb{Z}[\sqrt{-2}]$是ED.
}

(ii)$\mathbb{Z}[\sqrt{-3}]$甚至不是UFD,从而也不是PID或ED.

\zm{
	$4=2\times 2=(1+\sqrt{-3})(1-\sqrt{-3})$,我们来说明$2$,$1\pm\sqrt{-3}$均是$\mathbb{Z}[\sqrt{-3}]$中的不可约元.
	
	沿用{\heiti 习题}\textbf{4.1.3}中的记号. $N(2)=N(1\pm\sqrt{-3})=4$,若$2=xy$或$1\pm\sqrt{-3}=xy$,其中$x$不是平凡因子,则$x,y$都不是单位,$N(x)=N(y)=2$,但$a^2+3b^2=2$没有整数解,矛盾.
	
	由于$\mathbb{Z}[\sqrt{-3}]$中的单位只有$\pm 1$,故三者彼此不相伴,$\mathbb{Z}[\sqrt{-3}]$不是UFD.
}

\subsubsection{(2)}
$\mathbb{R}[x,y].$

\jie 该环不是PID,从而也不是ED.

\zm{
	考虑理想$(x,y)_{\Ideal}$,若它是主理想,则该理想中包含$x$为$y$的零次式,$y$为$x$的零次式,故其生成元只能是实常数,但该理想中不含实常数,矛盾.
}

\subsubsection{(3)}
$\mathbb{Z}[\omega]$,其中$\omega=\frac{-1+\sqrt{-3}}{2}$.

\jie (i)$\mathbb{Z}[\omega]$是ED,从而也是PID.

\zm{
	首先,$\omega^3=1$,$\omega^2=-\omega-1$,故$\mathbb{Z}[\omega]$中元素均有$a+b\omega\;(a,b\in\mathbb{Z})$的形式.
	
	令$\varphi: \mathbb{Z}[\omega]\rightarrow\mathbb{Z},
	a-b\omega\mapsto |a^2+ab+b^2|=|(a-b\omega)(a-b\omega^2)|=|(a+b+b\omega)(a-b\omega)|.$
	
	由于$\mathbb{C}$没有零因子,故$\varphi(a-b\omega)=0\Rightarrow b\omega=0$且$a+b=0$或$a=0\Rightarrow a-b\omega=0$,$\varphi$将环中非零元映射到正整数.
	
	若$\alpha, \beta\in\mathbb{Z}[\omega]$,则$\frac{\alpha}{\beta}=x-y\omega\in\mathbb{Q}[\omega]$,在$\mathbb{Z}$中选择$x_0,y_0$使得$|x-x_0|\leq\frac{1}{2},|y-y_0|\leq\frac{1}{2}$,则
	
	$\alpha=(x_0-y_0\omega)\beta+((x-x_0)-(y-y_0)\omega)\beta=q\beta+\gamma$,$\varphi(\gamma)=\varphi(\beta)\varphi(x-x_0-(y-y_0)\omega)=\varphi(\beta)|(x-x_0)^2+(x-x_0)(y-y_0)+(y-y_0)^2|\leq\frac{3}{4}\varphi(\beta)<\varphi(\beta)$.
	
	故$\mathbb{Z}[\omega]$是ED.
}

\subsection{}
设$D$是PID,$E$是整环,并且$D$是$E$的子环,$a,b\in D-\{0\}$,如果$d$是$a$和$b$在$D$中的最大公因子,证明$d$也是$a$和$b$在$E$中的最大公因子.

\zm{
	显然$d$也是$a$和$b$在$E$中的公因子,只需证明若$f$是$a$和$b$在$E$中的公因子则$f\mid d$即可.
	
	由于$D$是贝祖环,存在$x,y\in D$使得$ax+by=d$,故$f\mid ax,f\mid by\Rightarrow f\mid ax+by=d$.
}