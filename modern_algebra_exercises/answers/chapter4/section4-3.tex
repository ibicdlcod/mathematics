\section{多项式环与高斯引理}
\subsection{}
证明{\heiti 命题}\textbf{4.21}.

\zm{
	令$u=\mathrm{deg}f, v=\mathrm{deg}g, f=\sum_{i=0}^ua_ix^i, g=\sum_{j=0}^vb_jx^j$.
	
	(i) 在$f+g$中次数高于$\max\{u,v\}$的项为$0$,故$\mathrm{deg}f+g\leq \max\{u,v\}$.
	
	(ii) 在$fg$中次数高于$u+v$的项为$\sum_{i=0}^wa_ib_{w-i}x^w=\sum_{i=0}^ua_i\cdot 0\cdot x^w+\sum_{i=w-v}^w0\cdot b_ix^w+\sum_{i=u+1}^{w-v-1}0\times 0\cdot x^w=0\;((u+v)<w)$,$u+v$次项为$a_ub_v$其中$a_u,b_v\neq 0$,故$\mathrm{deg}fg\leq u+v$且当首项系数$a_u$和$b_v$不为零因子时$a_ub_v\neq 0$,$\mathrm{deg}fg=u+v$.
}

\subsection{}
设$R$是环,$f(x)=a_0+a_1x+\cdots+a_nx^n\in R[x]$. 证明:
\subsubsection{(1)}
$f(x)$可逆当且仅当$a_0\in U(R)$且$a_1,a_2,\cdots, a_n\in \mathrm{Nil}(R)$.

\zm{
	$(\Rightarrow)$若$f(x)$的逆$g(x)=b_0+b_1x+\cdots+b_mx^m$存在,则$a_0b_0=1_R$,故$a_0,b_0\in U(R)$.
	
	由于当$r\geq 1$时$\sum_{i=0}^ra_ib_{r-i}=0$,当$r=m+n$时有$a_nb_m=0$,当$r=m+n-1$时$a_{n-1}b_m+a_nb_{m-1}=0$,两边同乘$a_n$得$a_n^2b_{m-1}=0$,依次类推得$a_n^jb_{m+1-j}=0$,$a_n^{m+1}b_0=0$,两边同乘$a_0$得$a_n^{m+1}=0$,故$a_n$是幂零元素,从而$-a_nx^ng(x)$也是,由{\heiti 习题}\textbf{3.1.12(3)}知$1-a_nx^ng(x)=g(x)(f(x)-a_nx^n)$也是单位,由$f(x)$是单位,$f_1(x)=f(x)-a_nx^n=f(x)g(x)(f(x)-a_nx^n)$也是单位,从而$f_1$的首项$a_{n-1}$也是幂零元素,依次推理得$a_1,...,a_n$都是幂零元素.
	
	$(\Leftarrow)$ 令$b_0=a_0^{-1}$,则$b_0f(x)=1+xh(x)$,且$h(x)$的系数$b_0a_i\;(1\leq i\leq n)$都是幂零的,从而$xh(x)$各项$b_0a_ix^i\;(1\leq i\leq n)$在$R[x]$中幂零,由{\heiti 习题}\textbf{3.1.12(1)},$xh(x)$也是幂零的,由{\heiti 习题}\textbf{3.1.12(3)},$1+xh(x)$是$R[x]$中单位,即可逆.
}

\zm{
	$(\Rightarrow)$另一种证法(只适用于$R$为交换环),可得到更多对于素理想的理解. \emph{该证法由}\cite{83886}\emph{给出.}
	
	{\heiti 引理.} 交换环$R$中所有素理想的交集为$\mathrm{Nil}(R)$.
	
	{\heiti 引理的}\zm{ 由{\heiti 习题}\textbf{3.4.3}我们已经知道任何素理想包含$\mathrm{Nil}(R)$,我们只需证明若$a\notin \mathrm{Nil}(R)$则存在素理想$\mathfrak{p}$使得$a\notin\mathfrak{p}$即可.
	
	令$S=\langle a\rangle$,则$0\notin S$,$S$是乘法集,我们在$R$上定义局部化$\frac{r}{s}$其中$(r,s)\sim (r^{\prime},s^{\prime})$如果$\exists t\in S$ s.t. $t(r^{\prime}s-s^{\prime}r)=0$(这里与整环的局部化稍有区别,$t$对$\sim$的传传递性是必要的,并且$R$未必同构于$S^{-1}R$的一个子环:若存在$t\in S$使得$t(a-b)=0$则$a,b$在$S^{-1}R$中的像一致),类似{\heiti 习题}\textbf{3.4.14(3)}我们有$S^{-1}R$的素理想与$R$中与$S$不交的素理想一一对应,由$S^{-1}R$中极大理想存在(当然它也是素理想)知$R$中有与$S$不交的素理想,它当然不包含$a$.
	}
	
	考虑$R$的任意素理想$\mathfrak{p}$,此时$R/\mathfrak{p}$是整环,令$\varphi: R\rightarrow R/\mathfrak{p}$,则$\varphi(f)$的逆是$\varphi(g)$,由$\varphi(g)$的最高次项不为$\overline{0}$知$\varphi(f)$的非常数项都是$0$,即$a_1,a_2,...,a_n$在$\mathfrak{p}$中,由$\mathfrak{p}$的任意性,$a_1,a_2,...,a_n$是幂零元.
}

\subsubsection{(2)}
$f(x)$幂零当且仅当$a_0, a_1, ..., a_n$幂零.

\zm{
	$(\Leftarrow)$ 设$r_0,r_1,...,r_n$满足$a_i^{r_i}=0$. 当$u>\sum_{i=0}^n(r_i-1)$时$f(x)^u$展开后所有项的系数都必然有某个$a_i^{r_i}$为因子,故为$0$,即$f(x)^u=0$,$f(x)$为幂零元.
	
	$(\Rightarrow)$若$f(x)^v=0$,则$a_n^vx^{nv}=0$,故$a_nx^n$是$R[x]$中幂零元,从而$-a_nx^n$也是. 由{\heiti 习题}\textbf{3.1.12(1)},$f_1(x)=f(x)-a_nx^n=a_0+a_1+\cdots+a_{n-1}x^{n-1}$也是$R[x]$中幂零元,对$f_1(x)$同样推理知$a_{n-1}x^{n-1}$幂零.
	
	依次推理下去得$a_ix^i$都是幂零元,从而$a_i\;(0\leq i \leq n)$也都是.
}

\subsubsection{(3)}
$f(x)$是零因子当且仅当存在$0\neq a\in R$使得$af(x)=0$.

\emph{证明由文献}\cite{3347812}\emph{给出}.

\zm{
	题设条件似乎假定$R$是交换环,否则还应有条件$f(x)a=0$.
	
	令$g(x)$为零化$f(x)$的多项式中次数最小的,即$g(x)f(x)=0$且当$0\leq\mathrm{deg}h<\mathrm{deg}g$时$h(x)f(x)\neq 0$.
	
	令$f(x)=\sum_{i=0}^na_ix^i$,$g(x)=\sum_{i=0}^mb_ix^i$,则$a_nb_m=0$,故$a_ng(x)$为次数小于$m$的多项式且$a_ng(x)f(x)=0$,只能$\mathrm{deg}a_ng(x)=-\infty, a_ng(x)=0$,故$f_1(x)=\sum_{i=0}^{n-1}a_ix_i=f(x)-a_nx^n$满足$g(x)f_1(x)=g(x)f(x)-g(x)a_nx^n=0$,依次类推有$a_ig(x)=0\;(0\leq i\leq n)$,故$a_ib_m=0\;(0\leq i\leq n)$,即$f(x)b_m=0$.
}

\subsection{}
设$D$为UFD,$F$为$D$的商域,$f(x)$为$D[x]$中首一多项式. 证明:$f(x)$在$F[x]$中的每个首一多项式因子必属于$D[x]$.

\zm{
	$f(x)$可在$D[x]$上分解为不可约元之积,由{\heiti 引理}\textbf{4.31},该分解式也是$f(x)$在$F[x]$上的一个分解式$\prod_i g_i(x)$,由$f(x)$首一,该分解式可调整为使各$g_i(x)$也是首一多项式. 由因式分解的唯一性,该分解式与$f(x)$在$F[x]$上的任何另外一个分解式$\prod_i h_i(x)$(重新排列后)各因子只相差一个常数因子$a_i$,若对某个$i$有$h_i(x)$首一,则$a_i=1$,$h_i(x)=g_i(x)\in D[x]$.
}

\subsection{}
将$x^n-1\;(3\leq n\leq 10)$在$\mathbb{Z}[x]$中作素因子分解.

\emph{警告:读者必须证明所得的结果全部是$\mathbb{Z}[x]$中不可约多项式,才算完成此题.}

\jie 一次多项式都是不可约的. 由{\heiti 例}\textbf{4.34},$3,5,7$次分圆多项式$f_3(x)=x^2+x+1, f_5(x)=x^4+x^3+x^2+x^1+1, f_7(x)=x^6+x^5+x^4+x^3+x^2+x^1+1$在$\mathbb{Z}[x]$上不可约.

$x^3-1=(x-1)(x^2+x+1)$.

$x^4-1=(x-1)(x+1)(x^2+1)$,最后一项没有实根,故没有一次实因式,故在$\mathbb{R}[x]$中不可约,从而在$\mathbb{Z}[x]$中也不可约.

$x^5-1=(x-1)(x^4+x^3+x^2+x+1)$.

$x^6-1=(x^3-1)(x^3+1)=(x+1)(x-1)(x^2+x+1)(x^2-x+1)$,最后一项为$f_3(-x)$,故也在$\mathbb{Z}[x]$中不可约.

$x^7-1=(x-1)(x^6+x^5+x^4+x^3+x^2+x+1)$.

$x^8-1=(x+1)(x-1)(x^2+1)(x^4+1)$,在最后一项$x^4+1$中以$x+1$代替$x$得$x^4+4x^3+6x^2+4x+2$,取$p=2$满足艾森斯坦判别法的条件,故$x^4+1$也在$\mathbb{Z}[x]$中不可约.

$x^9-1=(x-1)(x^2+x+1)(x^6+x^3+1)$,在最后一项$x^6+x^3+1$中以$x+1$代替$x$得$x^6+6x^5+15x^4+21x^3+18x^2+9x+3$,取$p=3$满足艾森斯坦判别法的条件,故$x^6+x^3+1$也在$\mathbb{Z}[x]$中不可约.

$x^{10}-1=(x+1)(x-1)(x^4+x^3+x^2+x+1)(x^4-x^3+x^2-x+1)$,最后一项为$f_5(-x)$,故也在$\mathbb{Z}[x]$中不可约.

\subsection{}
设$F$为域,$d: F[x]\rightarrow F[x]$为线性映射,若对于任意的$f,g\in F[x], d(fg)=(df)g+f(dg)$,则称$d$为$F[x]$上的一个{\heiti 线性导子}. 请找出$f[x]$上所有线性导子.

\jie $d(1\cdot g)=(d1)g+1\cdot d(g)$,故$(d1)g=0$对一切$g\in F[x]$成立,故$(d1)=0$,由$d$是线性映射,$d(F)=0$.

$d\left(\sum_i a_ix^i\right)=\sum_i d(a_ix^i)=\sum_i (d(a_i)x^i+a_id(x^i))
=\sum_i(0\cdot x^i+a_i(\sum_{j=1}^ix^{j-1}d(x)x^{i-j}))=\sum_i\sum_{j=1}^ia_ix^{i-1}d(x)
=\sum_iia_ix^{i-1}d(x)$

故$d(f)=f^{\prime}d(x)$是形式微商的$d(x)$倍,其中$d(x)\in F[x]$.

\subsection{}
设$f(x)$是$\mathbb{Q}[x]$中奇次不可约多项式,$\alpha$和$\beta$是$f(x)$在$\mathbb{Q}$的某个扩域中两个不同的根,求证$\alpha+\beta\notin\mathbb{Q}$.

\emph{证明由文献}\cite{3126329}\emph{给出.}

\zm{
	由于$\mathbb{Q}$的代数闭包存在(见{\heiti 例}\textbf{5.21}),因此$f(x)$在代数闭包$\overline{\mathbb{Q}}$或者它的任何扩域中有奇数个不同的根.
	
	对$f(x)$的任意一个根$u$,由$f(x)$不可约且不是一次多项式($(x-\alpha)(x-\beta)\mid f(x)$),有$u\notin\mathbb{Q}$;
	
	定义同态$\varphi_u: \mathbb{Q}[x]\rightarrow\overline{\mathbb{Q}}, f(x)\mapsto f(u)$,则$(f(x))_{\Ideal, \mathbb{Q}[x]}\subseteq\ker\varphi_u$,由$\mathbb{Q}$是域,$\mathbb{Q}[x]$是PID,故也是贝祖环,当$g(x)\notin (f(x))_{\Ideal, \mathbb{Q}[x]}$时因$f(x)$不可约,$\gcd(f(x),g(x))=1, af(x)+bg(x)=1\;(a,b\in\mathbb{Q}[x]), bg(u)=af(u)+bg(u)=\varphi_u(1)=1$;
	
	故$u$不是$g(x)$的根,$\ker\varphi_u\subseteq(f(x))_{\Ideal, \mathbb{Q}[x]}$,故$\ker\varphi_u=(f(x))_{\Ideal, \mathbb{Q}[x]}$,$\mathbb{Q}[x]/f(x)=\mathbb{Q}[x]/\ker\varphi_u\cong\im\varphi_u=\mathbb{Q}[u]$.
	
	由$u$的任意性,对$f(x)$的任意一个根$\gamma$,$\varphi_{\gamma\alpha^{-1}}=\varphi_{\gamma}\circ\varphi_{\alpha}^{-1}$诱导$\mathbb{Q}[\alpha]$通过$\mathbb{Q}[x]/f(x)$到$\mathbb{Q}[\gamma]$的同构$f(\alpha)\mapsto f(\gamma)$,该同构在$\mathbb{Q}$上的限制为恒等映射,令$r=\alpha+\beta$.
	
	若$r\in\mathbb{Q}$,则$\beta\in\mathbb{Q}[\alpha]$,$\varphi_{\gamma\alpha^{-1}}(\beta)=\varphi_{\gamma\alpha^{-1}}(r-\alpha)=r-\gamma$,且$f(r-\gamma)=\varphi_{\gamma\alpha^{-1}}(f(\beta))=\varphi_{\gamma\alpha^{-1}}(0)=0$,故对任意一个根$\gamma$总有另一个根$\varphi_{\gamma\alpha^{-1}}(\beta)$与它的和是$r$,由于$\alpha\neq\beta$,易见这两个根$\gamma$和$\varphi_{\gamma\alpha^{-1}}(\beta)$不同.
	
	故$f(x)$在$\overline{\mathbb{Q}}$中的根成对出现,与根为奇数个矛盾,故$r\notin\mathbb{Q}$.
}

\subsection{}
设$R$是含幺环,定义集合
$$R[[x]]=\left\{\sum_{n=0}^{+\infty}a_nx^n\mid a_n\in R\;(n=0,1,2,...)\right\},$$
每个元素$\sum_{n=0}^{+\infty}a_nx^n$叫做$R$上的{\heiti 形式幂级数}. 定义
$$\sum a_nx^n+\sum b_nx^n=\sum(a_n+b_n)x^n,$$
$$\left(\sum a_nx^n\right)\left(\sum b_nx^n\right)=\sum\left(\sum_{i+j=n}a_ib_j\right)x^n.$$

\subsubsection{(1)}
$R[[x]]$对于上述加法和乘法形成含幺环,叫做环$R$上关于$x$的{\heiti 形式幂级数环}.

\Proofbyintimidation

\subsubsection{(2)}
若$R$为交换环,则$R[[x]]$也是交换环.

\Proofbyintimidation

\subsubsection{(3)}
多项式环$R[x]$可自然看成是$R[[x]]$的子环.

\zm{
	易见$R[[x]]$中的只有有限个$a_n$不为$0_R$的元素与$R[x]$中的元素一一对应,并且保持加法和乘法.
}

\subsubsection{(4)}
设$R$是含幺交换环,$f(x)=\sum_{i=0}^{\infty}a_ix^i\in R[[x]]$,则$f(x)$可逆当且仅当$a_0\in R^{\times}$.

\zm{
	($\Leftarrow$)当$a_0$在$R$中可逆时我们构造$f(x)$的逆$g(x)=\sum_{i=0}^{\infty}b_ix^i$:
	
	$b_0=a_0^{-1}\in R$.
	
	假设我们对一切$i<s\in\mathbb{Z}_+$得到了$b_i\in R$,由$\sum_{j=1}^sa_jb_{s-j}+a_0b_s=0$得到$b_s=-a_0^{-1}\sum_{j=1}^sa_jb_{s-j}$.
	
	对一切$s\in\mathbb{N}$计算$b_s$,则$f(x)g(x)$满足对一切$s\in\mathbb{Z}_+, \sum_{j=1}^sa_jb_{s-j}+a_0b_s=0, a_0b_0=1$,故$f(x)g(x)=1$是$R[[x]]$中的单位元.
	
	($\Rightarrow$)若$a_0$在$R$中不可逆,则$1\notin a_0R$,但$f(x)g(x)$的常数项一定在$a_0R$中,故$f(x)g(x)\neq 1$,$f(x)$在$R[[x]]$中不可逆.
}

\subsubsection{(5)}
若$R$为域,则$R[[x]]$是PID且只有唯一的极大理想$\mathfrak{m}$,求出$R[[x]]$的所有理想.

\zm{
	设$I$是$R[[x]]$的理想,则$I=\{0\}\cup\bigsqcup_{i}f_i(x)$,其中$f_i(x)$为非零元素.
	
	当$I$不为零理想时,令$k_i$为$f_i$的次数(因为$f\neq0$,因此$k_i>-\infty$为整数),则$f_i(x)=x^{k_i}g_i(x)$,其中$g_i\notin xR[[x]]$,$g_i$的常数项不为$0$,由于$R$是域,因此$g_i$的常数项可逆,由{(4)}知存在$h_i(x)\in R[[x]]$使得$g_ih_i=1$为单位元,即$(g_i)_{\Ideal, R[[x]]}=R[[x]]$,故$(f_i)_{\Ideal, R[[x]]}=x^{k_i}R[[x]]$是$f_i$生成的理想.
	
	故$I\supseteq\bigsqcup_ix^{k_i}R[[x]]=x^{\min_i\{k_i\}}R[[x]]$,若$I$还包含其他元素,该元素必定有项不被$x^{\min_i\{k_i\}}$整除,也就是不被所有$x^{k_i}$整除,与$I=\{0\}\cup\bigsqcup_{i}f_i$矛盾,故$I=x^kR[[x]]$为$x^k$生成的理想,其中$k=\min_i\{k_i\}\in\mathbb{N}$,故$I$是主理想,$R[[x]]$的所有理想为$x^kR[[x]]$或$\{0\}$.
	
	易见$k_1<k_2$时$x^{k_1}R[[x]]\supsetneqq x^{k_2}R[[x]]$,且$k\geq 1$时$x^kR[[x]]$为真理想,故唯一的极大理想即$\mathfrak{m}=xR[[x]]$.
}

\subsubsection{(6)}
本题与{\heiti 习题}\textbf{3.1.6, 3.4.11}有什么联系?

\zm{
	$\varphi: \mathbb{Z}_p\rightarrow\mathbb{Z}[[x]]/(x-p), (a_1, a_2, ...)\mapsto \sum_{i=0}^{+\infty} \frac{a_{i+1}-a_i}{p^i}x^i (a_0=0)$,由于$a_{i+1}\equiv a_i\mod p^i, a_{i+1}-a_i<p^{i+1}$,故$\varphi$是良好定义的,请读者自行证明$\varphi$是同构.
}

\subsection{}
试确定$\mathbb{R}[x]$和$\mathbb{Z}[x]$的所有素理想和极大理想.

\jie (i) 由于$\mathbb{R}[x]$是整环,因此零理想是素理想. 由于$\mathbb{R}$是域,因此$\mathbb{R}[x]$是PID.

考虑理想$I=(f(x))_{\Ideal, \mathbb{R}[x]}$,其中$f(x)\neq 0$. 若$f(x)$在$\mathbb{R}[x]$中可约,则它的两个非平凡因子都不在$I$中,乘积却在$I$中,因此$I$不是素理想. 反之,若$f(x)$在$\mathbb{R}[x]$中不可约,则$ab\in I\Rightarrow \exists u,v$ s.t. $u\mid a,v\mid b, uv=f(x)$,只能$u\sim f(x),v\sim 1$或$u\sim1,v\sim f(x)$,故$a$或$b$在$I$中,$I$是素理想. 故$I$中素理想为零理想和$\mathbb{R}[x]$中不可约多项式生成的理想.

由代数基本定理,$\mathbb{R}[x]$中多项式$f$也是$\mathbb{C}[x]$中多项式且有$\deg f$个根,即它在$\mathbb{C}[x]$中分解为一次因式. 由于复共轭是$\mathbb{C}[x]$的自同构且保持$\mathbb{R}[x]$不变,因此$a+bi$是$f$的根$(a,b\in\mathbb{R})\Leftrightarrow a-bi$是$f$的根,故当$b\neq 0$时$(x-a-bi)(x-a+bi)=x^2-2ax+a^2+b^2$是$f$的因式,它属于$\mathbb{R}[x]$,且它的判别式$4a^2-4a^2-4b^2=-4b^2<0$,故它没有实数根,在$\mathbb{R}[x]$中不可约. 综上,$\mathbb{R}[x]$中不可约多项式为一次因式或判别式小于$0$的二次因式.

故$\mathbb{R}[x]$的所有素理想为$\{0\}, (x-a)_{\Ideal}\;(a\in\mathbb{R}), (x^2+bx+c)_{\Ideal}\;(b,c\in\mathbb{R}, b^2-4c<0)$.

除零理想外,令$I$是不可约多项式$f(x)$生成的理想,若$J\supsetneqq I$是理想,则它也是多项式生成的理想$(g(x))_{\Ideal}$,故$g(x)\mid f(x), g(x)\nsim f(x)$,由$f(x)$不可约,只能$g(x)\sim 1$,即$J=\mathbb{R}[x]$,故不可约多项式生成的理想$(x-a)_{\Ideal}\;(a\in\mathbb{R}), (x^2+bx+c)_{\Ideal}\;(b,c\in\mathbb{R}, b^2-4c<0)$都是$\mathbb{R}[x]$的极大理想.

(ii) \emph{该部分证明由文献}\cite{174713}\emph{给出.}

若$\mathfrak{p}$是$\mathbb{Z}[x]$的素理想,考虑$\mathfrak{p}\cap\mathbb{Z}$,若$a,b\in\mathbb{Z}, ab\in\mathfrak{p}\cap\mathbb{Z}$,则$a$或$b\in\mathfrak{p}$,即$a$或$b\in\mathfrak{p}\cap\mathbb{Z}$,故$\mathfrak{p}\cap\mathbb{Z}$是$\mathbb{Z}$的素理想,这只能有两种情况:

(iia) $\mathfrak{p}\cap\mathbb{Z}=(0)$. 我们可能有$\mathfrak{p}=(0)$,若不然,则令$S=\mathfrak{Z}-\{0\}$,$S$是$\mathbb{Z}-\{0\}$上乘法含幺半群,故有整环$S^{-1}\mathbb{Z}[x]=\mathbb{Q}[x]$,由{\heiti 习题}\textbf{3.4.14(3)},$S^{-1}\mathfrak{p}$是$\mathbb{Q}[x]$的素理想,而$\mathbb{Q}[x]$是PID,故$S^{-1}\mathfrak{p}=(q(x))_{\Ideal, \mathbb{Q}[x]}$,其中$q(x)$是$\mathbb{Q}[x]$中不可约多项式. 若$q(x)$的容积不为$1$而为$c(q)$,则用$c(q)^{-1}q(x)$替代$q(x)$,理想不变,故可假定$q(x)$的容积为$1$.

由于$S^{-1}\mathfrak{p}\cap\mathbb{Z}[x]=\mathfrak{p}$(否则设$a\in S^{-1}\mathfrak{p}, a\notin\mathfrak{p}$,则对任意$s\in S$,$s\in\mathbb{Z}$,若$sa\in\mathfrak{p}$则由素理想的性质,$s\in\mathfrak{p}$(与(iia)条件矛盾)或$a\in\mathfrak{p}$(仍然矛盾),故$sa\notin\mathfrak{p}, a=s^{-1}sa\notin s^{-1}\mathfrak{p}$对所有$s\in S$成立,与$a\in S^{-1}\mathfrak{p}$矛盾),由$q(x)$容积为$1$,$(q(x))_{\Ideal, \mathbb{Z}[x]}\subseteq\mathfrak{p}$. 若$f(x)=\frac{r}{s}q(x)\in\mathbb{Z}[x]$,则$f(x)$的容积为$\frac{r}{s}$为整数,故$\frac{r}{s}$为整数,$f(x)\in\mathfrak{p}$,故$(q(x))_{\Ideal, \mathbb{Z}[x]}=\mathfrak{p}$为$\mathbb{Q}[x]$上整系数不可约本原多项式生成的理想,即$\mathbb{Z}[x]$上一次以上不可约多项式生成的理想.

(iib) $\mathfrak{p}\cap\mathbb{Z}=p\mathbb{Z}$,令$\varphi: \mathbb{Z}[x]\rightarrow \mathbb{Z}[x]/p\mathbb{Z}\cong\mathbb{F}_p[x]$,则$\varphi$是满同态,由{\heiti 习题}\textbf{3.4.5(1)},$\varphi(\mathfrak{p})$是$\mathbb{F}_p[x]$的素理想.

由于$\mathbb{F}_p$是域,$\mathbb{F}_p(x)$为PID,其素理想为$(q(x))_{\Ideal, \mathbb{F}_p(x)}$其中$q(x)$为$\mathbb{F}_p[x]$上不可约多项式或零理想,对于后一种情况我们有$\mathfrak{p}=p\mathbb{Z}[x]$.

对于前一种情况,由于$\mathbb{F}_p$是域,若$q(x)$的首项系数不为$1$而为$a_n(q)$,则用$p(x)=a_n(q)^{-1}q(x)$替代$q(x)$,则$(p(x))_{\Ideal, \mathbb{F}_p(x)}=(q(x))_{\Ideal, \mathbb{F}_p(x)}$,若$p(x)$在$\mathbb{Z}[x]$中可约,则它在$\mathbb{F}_p[x]$中也可约,矛盾,故$p(x)$是$\mathbb{Z}[x]$中不可约多项式.

显然$(p, p(x))_{\Ideal, \mathbb{Z}[x]}\subseteq\mathfrak{p}$,反之,若$r(x)\in\mathfrak{p}$,则$\varphi(r(x))\in(p(x))_{\Ideal, \mathbb{F}_p[x]}$,存在$\mathbb{F}_p[x]$中多项式$s(x)$使得$s(x)p(x)\equiv r(x)$,故$s(x)p(x)-r(x)\in p\mathbb{Z}[x]$,$r(x)=s(x)p(x)+pu(x)$其中$u(x)\in\mathbb{Z}[x]$,故$r(x)\in(p,p(x))_{\Ideal, \mathbb{Z}[x]}$,故$\mathfrak{p}=(p,p(x))_{\Ideal, \mathbb{Z}[x]}$.

综上,$\mathbb{Z}[x]$中的一切素理想如下:

(A) $\{0\}$;

(B) $(f(x))_{\Ideal}$,其中$f(x)$是$\mathbb{Z}[x]$内一次以上不可约多项式;

(C) $p\mathbb{Z}[x]$;

(D) $(p, f(x))_{\Ideal}$,其中$f(x)$是$\mathbb{F}_p[x]$内不可约多项式.

其中(A) (C)当然不是极大理想,对(B),若对一切素数$p$有$f(x)$在$\mathbb{F}_p[x]=\mathbb{Z}[x]/p\mathbb{Z}[x]$中可约,则由中国剩余定理,$f(x)$在$\mathbb{Z}[x]\cong\mathbb{Z}[x]/\bigcap_{p_i\text{为一切素数}}p_i\mathbb{Z}[x]\cong\prod_{p_i\text{为一切素数}}\mathbb{F}_{p_i}[x]$中可约,矛盾,故存在$p$使得$f(x)$在$\mathbb{F}_p[x]$内也是不可约多项式,存在(D)中的某个素理想包含$(f(x))_{\Ideal}$.

对(D),若$p_1\neq p_2$,则$(p_1, f(x))_{\Ideal}, (p_2, g(x))_{\Ideal}$仅考虑与$\mathbb{Z}$的交集就互不包含,故只需讨论$p$固定的情况.

若$(p, f(x))_{\Ideal, \mathbb{Z}[x]}\supsetneqq (p, g(x))_{\Ideal, \mathbb{Z}[x]}$,则$g(x)\equiv f(x)r(x)\mod p$,即在$\mathbb{F}_p(x)$中$f(x)\mid g(x), g(x)\nmid f(x)$,迫使$f(x)\sim 1$,与$f(x)$不可约矛盾.

综上,$\mathbb{Z}[x]$的极大理想即上述情况(D).

\subsection{}
试确定$\mathbb{Z}[x]$,$\mathbb{Q}[x]$的自同构群.

\jie (i) 令$\varphi\in\Aut(\mathbb{Z}[x])$,则$\varphi$在$\mathbb{Z}$上的限制为恒等映射. 令$\varphi(x)=f(x)$,由同构是双射,$f(x)\notin\mathbb{Z}$. 由{\heiti 习题}\textbf{4.3.8}知对任意素数$p$,$(p,x)_{\Ideal, \mathbb{Z}[x]}$都是极大理想,故$f(x)$也需满足$(p,f(x))_{\Ideal, \mathbb{Z}[x]}$也是极大理想,即$f(x)$对任意素数$p$都是$\mathbb{F}_p$中不可约多项式.

若$f(x)$的次数$\geq 2$,令$f(x)=a_nx^n+\cdots+a_1x+a_0$,$c$为任意整数,则$f(x)=f(c)+(x-c)g(x)$,若有素数$p$整除$f(c)$,则$(x-c)$是$\mathbb{F}_p(x)$中$f(x)$的因式,矛盾. 故$f(c)=1$对任何$c$成立,得$f(x)=1$,矛盾.

故$\varphi(x)=ax+b$,其中$a,b\in\mathbb{Z},a\neq 0$,若$a\neq \pm 1$,则$\varphi^{-1}(x)=a^{-1}(x-b)\notin\mathbb{Z}[x]$,矛盾,故$a=\pm 1$,
$\Aut(\mathbb{Z}[x])\cong D_{\infty}=\langle \sigma,\tau\mid \tau\sigma\tau^{-1}=\sigma^{-1}, \tau^2=\id \rangle=\mathbb{Z}\rtimes(\mathbb{Z}/2\mathbb{Z})$.

(ii) $\mathbb{Q}$是域,类似{\heiti 习题}\textbf{4.3.8}的讨论知$\mathbb{Q}[x]$上的极大理想是不可约多项式生成的理想,即$\mathbb{Q}[x]$的自同构$\varphi$应把不可约多项式映射到不可约多项式,特别地,$\varphi(x)=f(x)$为不可约多项式.

故$\varphi$诱导同构$\tau: \mathbb{Q}\cong\mathbb{Q}[x]/(x)_{\Ideal}\leftrightarrow\mathbb{Q}[x]/f(x)_{\Ideal}\cong\mathbb{Q}(\alpha)$,其中$\alpha$是$f(x)$的一个根. 若$\deg f\geq 2$,则$f(x)$不可约导致它没有有理根,必有$\alpha\notin\mathbb{Q}$,但域的同构$\tau$必将$\mathbb{Q}$映射到它本身,而$\mathbb{Q}(\alpha)$包含无理数,即$\tau$必不是满射,矛盾.

故$\varphi(x)=ax+b, a,b\in\mathbb{Q},a\neq 0$,$\Aut(\mathbb{Q}[x])$同构于$\mathbb{Q}$上$2$阶一般线性群$\mathrm{GL}_2(\mathbb{Q})$的子群
$$\left\{
\begin{pmatrix}
	k & l\\
	0 & 1
\end{pmatrix}
\mid l\in\mathbb{Q}, k\in\mathbb{Q}^{\times} \right\}.$$.

\subsection{}
设$c_0,c_1,\cdots, c_n$是整环$D$中两两相异的$n+1$个元素,$d_0,d_1,\cdots,d_n$是$D$中任意$n+1$个元素,证明:

\subsubsection{(1)}
在$D[x]$中至多存在一个次数$\leq n$的多项式$f(x)$使得$f(c_i)=d_i (\forall 0\leq i \leq n)$;

\zm{
	若存在次数$\leq n$的两个不同多项式$f,g$满足要求,则非零多项式$h(x)=f(x)-g(x)$满足$h(c_i)=0$对所有$0\leq i\leq n$成立,即$h$有全部$c_i$即$n+1$个不同根,而$\deg h\leq n$,与{\heiti 推论}\textbf{4.25}矛盾.
}

\subsubsection{(2)}
若$D$是域,则(1)中所述多项式存在.

\zm{
	令$g_i(x)=\prod_{\substack{j=0\\j\neq i}}^n(x-c_j)$,则对$j\neq i$有$g_i(c_j)=0$,且由$c_i$彼此相异,$g_i(c_i)\neq 0$,记它为$a_i$.
	
	令$h_i(x)=d_i/a_i\cdot g_i(x)$,则$h(x)=\sum_{i=0}^nh_i(x)$满足条件.
}

\subsection{}
判断下列元素是否为$\mathbb{Z}[x], \mathbb{Q}[x], \mathbb{R}[x], \mathbb{C}[x], \mathbb{Z}[[x]]$中的可逆元?是否为不可约元?

(1) $2x+2$;(2) $x^2+1$;(3) $x+1$;(4) $x^2+3x+2$.

\jie 对于前四个环,$n$次多项式生成的理想中元素(除$0$外)次数都$\geq n$,故一次以上多项式总是不可逆的,在$\mathbb{Z}[[x]]$中则不然,由{\heiti 习题}\textbf{4.3.7(4)}知当且仅当常数项在$\mathbb{Z}$中可逆(即为$\pm 1$时$f(x)$总是单位.

故在前四个环中,四个多项式都不可逆,而在$\mathbb{Z}[[x]]$中(2)(3)可逆.

$2x+2=2(x+1)$,它在$\mathbb{Z}[x]$中不是不可约元,而在$\mathbb{Q}[x],\mathbb{R}[x],\mathbb{C}[x]$中由于$2$是单位,它是不可约元,而在$\mathbb{Z}[[x]]$中,$x+1$是单位,$2$却不是,故也是不可约元.

$x^2+1$没有实根,故在$\mathbb{Z}[x], \mathbb{Q}[x], \mathbb{R}[x]$中都是不可约元,而在$\mathbb{C}[x]$中它等于$(x+i)(x-i)$,不是不可约元,在$\mathbb{Z}[[x]]$中它是单位,故不是不可约元.

$x+1$为本原一次多项式,在前四个环中都是不可约元,在$\mathbb{Z}[[x]]$中它是单位,故不是不可约元.

$x^2+3x+2=(x+1)(x+2)$,在前四个环中都不是不可约元,在$\mathbb{Z}[[x]]$中$(x+1)$是单位而$(x+2)$不是,故是不可约元.

\subsection{}
设$f=\sum a_ix^i\in\mathbb{Z}[x]$为首$1$多项式,$p$为素数,以$\overline{a}$表示$a\in\mathbb{Z}$在环的自然同态$\mathbb{Z}\rightarrow\mathbb{F}_p$中的像,而令$\overline{f}(x)=\sum \overline{a_i}x^i\in\mathbb{F}_p[x]$,

\subsubsection{(1)}
求证:若对某个素数$p$,$\overline{f}(x)$在$\mathbb{F}_p[x]$中不可约,则$f(x)$在$\mathbb{Z}[x]$中不可约.

\zm{
	若$f(x)$可约,则它是至少$2$个非单位因子的积,并且由高斯引理,这些非平凡因子都是首一多项式,从而也是首一的一次以上多项式,故在同态下的像也是首一的一次以上多项式,$\overline{f}(x)$也是至少$2$个非单位因子的积,从而在$\mathbb{F}_p[x]$中可约,矛盾.
}

\subsubsection{(2)}
若$f(x)$不是$\mathbb{Z}[x]$中首一多项式,(1)的结论是否成立?

\jie 否,$f(x)=2x^2+3x+1=(2x+1)(x+1)$为$\mathbb{Z}[x]$中可约多项式,但若$p=2$,$\overline{f}=x+1$是$\mathbb{F}_2[x]$中一次式,它是不可约的.

\subsection{}
设$F$为域,$a,b\in F$且$a\neq 0$. 证明$f(x)$在$F[x]$中不可约当且仅当$f(ax+b)$在$F[x]$中不可约.

\zm{
	若$f(x)$有非平凡素因子分解$f(x)=\prod_{i=1}^ng_i(x)$,则$f(ax+b)=\prod_{i=1}^ng_i(ax+b)$,其中$1\leq\deg g_i(ax+b)=\deg g_i<\deg f$,故是$f(ax+b)$的非平凡素因子分解.
	反之,$f(x)=f(a^{\prime}(ax+b)+b^{\prime})$,其中$a^{\prime}=a^{-1}, b^{\prime}=-a^{-1}b$也是$F$中元素,类似上述推理即可.
}

\subsection{}
设$p$是$\mathbb{Z}$上的奇素数,$n$为正整数. 证明$x^n-p$是$\mathbb{Z}[i]$上的不可约多项式.

\zm{	
	$p$是高斯素数或两个共轭的高斯素数之积$\pi_i\overline{\pi_i}$. 由艾森斯坦判别法({\heiti 定理}\textbf{4.33}),我们只要证明$\pi_i^2\nmid p$,$\overline{\pi_i}^2\nmid p$即可,这只需要证$\pi_i=a+bi, \overline{\pi_i}=a-bi$彼此不相伴即可. 由$\varphi(\pi_i)=p$无平方因子知$a,b\neq 0$,且$\gcd(a,b)^2\mid p$得$\gcd(a,b)=1$.

	若$a+bi$被$a-bi$整除,则$(a+bi)=(c+di)(a-bi)$,即$a(c-1)=-bd, b(c+1)=ad, c^2-1=\frac{-bd}{a}\cdot\frac{ad}{b}=-d^2$,若$d=0$,因$c-1$与$c+1$不同时为$0$,得$a$或$b=0$,矛盾,若$d=\pm 1$,则$c=0, a=\pm b$,$p=(a+ai)(a-ai)=2a^2$与$p$是奇素数矛盾,故$a+bi$, $a-bi$彼此不相伴.
}


\subsection{}
证明两个整多项式在$\mathbb{Q}[x]$中互素当且仅当它们在$\mathbb{Z}[x]$中生成的理想含有一个整数.

\zm{
	($\Rightarrow$)由$\mathbb{Q}$是域,$\mathbb{Q}[x]$是PID,故也是贝祖环,两个互素的整多项式$f,g$满足$m(x)f(x)+n(x)g(x)=1, m,n\in\mathbb{Q}[x]$. 令$\frac{p_1}{q_1}, \frac{p_2}{q_2}$为$m,n$的容积,则$q_1q_2m, q_1q_2n$为整多项式,$q_1q_2=q_1q_2mf+q_1q_2ng$在$(f,g)_{\Ideal, \mathbb{Z}[x]}$中.
	
	($\Leftarrow$)我们有整数$u=f(x)r(x)+g(x)s(x)$,其中$r,s\in\mathbb{Z}[x]$. 故$\gcd(f,g)\mid u$,这只能$\gcd(f,g)\sim 1$为非零有理常数.
}

\subsection{}
设$f(x)=\sum_{i=0}^na_ix^i\in\mathbb{Z}[x], \deg f=n$. 若存在素数$p$和整数$k\;(0<k<n)$使得:
$$p\nmid a_n, p\nmid a_k, p\mid a_i\;(0\leq i\leq k-1), p^2\nmid a_0.$$
求证$f(x)$在$\mathbb{Z}[x]$中必存在次数$\geq k$的不可约因子.

\zm{
	令$f_0(x)=f(x)$,我们构造一串多项式$f_j(x)=\sum_{i=0}^{n_j}a_{ij}x^i\in\mathbb{Z}[x]$使得$p\nmid a_{n_j}, p\nmid a_{k_j}, p\mid a_i\;(0\leq i\leq k_j-1), p^2\nmid a_0j$且$n_{j+1}<n_j, k_{j+1}\geq k_j$.
	
	若$f_j(x)$不可约,则由$k_j\geq k_0=k$我们得到结论. 否则,设$f_j(x)=g(x)h(x)$为非平凡素因子分解,$\deg g,h<\deg f_j$,$g(x)=\sum_{i=0}^mb_ix^i$,$h(x)=\sum_{i=0}^lc_ix^i$. 比较首项系数知$m+l=n_j$且$p\nmid b_mc_l=a_{n_j}$,故$p\nmid b_m, c_l$. 比较常数项系数,我们有$p\mid b_0c_0$且$p^2\nmid b_0c_0$. 不妨设$p\mid b_0$且$p\nmid c_0$. 令$s$满足$p\mid b_i, i<s$且$p\nmid b_s$,则
	$$a_{sj}=b_sc_0+b_{s-1}c_1+\cdots+b_0c_s.$$
	当$s<k_j$时$p\mid a_{sj}$,但$p\nmid b_sc_0$,$p\mid b_{u}c_{s-u}\;(u<s)$,故$p\nmid a_{sj}$,矛盾. 故$s\geq k_j$,我们记$k_{j+1}=s$得$f_{j+1}(x)=g(x)$.
	
	由于$n_j$严格递减,$k_j$不减少,因此这个过程不能无限继续下去,要么提前因$f_j$不可约而终止,要么当$j$足够大时$f_j$满足$n_j=k_j$,此时$s\geq k_j$和$s<n_j$不能同时成立,矛盾,即$f(j)$不可约,并是$k_j\geq k$次多项式.
}

\subsection{}
设$D$是整环,$0\neq f(x)=a_0+a_1x+\cdots+a_nx^n\in D[x]$. 若$(a_0,a_1,\cdots,a_n)\sim 1$,则$f(x)$在$D[x]$中不可约分解若存在则必唯一.

\zm{
	由于$c(f)\sim 1$,$f(x)$的两个因式分解必为
	$$f(x)=ug_1(x)g_2(x)\cdots g_s(x)=vh_1(x)h_2(x)\cdots h_t(x),$$
	并且$g_i(x)$和$h_j(x)$的容积全为单位. 故$g_1(x)g_2(x)\cdots g_s(x)\sim_{D[x]}vh_1(x)h_2(x)\cdots h_t(x)$,它们在$D$的商域的多项式环$F[x]$中也相伴,由$F[x]$中因式分解的唯一性,在合适次序下$s=t, g_i(x)\sim_{F[x]}h_i(x)$,但$g_i,h_i$的容积都是$D$中单位,从而它们在$D$中也相伴.
}
