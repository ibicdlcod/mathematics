\section{高斯整数和二平方和问题}
该节不加说明地引用了Legendre符号$\displaystyle\left(\frac{p}{q}\right)$(本书记为$\displaystyle\left(\frac{p}{q}\right)_{\mathrm{Le}}$),没有学习过的读者请参考《代数学I:代数学基础》)
\subsection{}
设$p$是奇素数,$p\equiv 1\mod 4$. 如果$(a,b)$是不定方程$x^2+y^2=p$的一组整数解,则它的全部整数解为$(x,y)=(\pm a,\pm b), (\pm b,\pm a)$

\zm{
	易验证上面$8$组整数确实是解,根据{\heiti 定理}\textbf{4.20},该方程只有$8$组解,故得结论.
}

\subsection{}
将$60$和$81+8\sqrt{-1}$在环$\mathbb{Z}[\sqrt{-1}]$中分解为不可约元之积.

\jie
$60=2*2*3*5$,其中$3$为高斯素数,$2,5$为共轭的高斯素数之积,易见
$2=(1+i)(1-i)=-i(1+i)^2$,$5=(1+2i)(1-2i)$为在相伴意义下唯一的高斯素数分解,
故$60=-3(1+i)^4(1+2i)(1-2i)$.

我们沿用正文4.2节中的记号,$\varphi(81+8\sqrt{-1})=6625=5^3\times 53$,其中$5=1^2+2^2$和$53=2^2+7^2$都是共轭的高斯素数之积.
故$(81+8\sqrt{-1})(81-8\sqrt{-1})=6625=\epsilon_0(1+2i)^3(1-2i)^3(2+7i)(2-7i)$,由于$81+8\sqrt{-1}$和$81-8\sqrt{-1}$共轭,故前者的非单位高斯素因子只有$4$个,易见$5\nmid 81+8\sqrt{-1}, 53\nmid 81+8\sqrt{-1}$,故因子分解只能是$81+8\sqrt{-1}=\epsilon_1(1\pm 2i)^3(2\pm 7i)$的形式,其中$\epsilon_i=1,-1,i,-i$为单位. 依次验证可知$81+8\sqrt{-1}=-i(1-2i)^3(2-7i)$.

\subsection{}
试求方程$x^2+y^2=585$的所有整数解.

\jie $585=3^2\times 5\times 13$,其中$3$是高斯素数,$5=1^2+2^2$,$13=2^2+3^2$,故$585=a\overline{a}$其中$a=\epsilon_1\pi_1^{\beta_{11}}\overline{\pi}_1^{\beta_{12}}\pi_2^{\beta_{21}}\overline{\pi}_2^{\beta_{22}}\cdot 3$,$\pi_1\overline{\pi}_1=5, \pi_2\overline{\pi}_2=13, \beta_{11}+\beta{12}=\beta{21}+\beta{22}=1$.

选择$\pi_1=1+2i, \pi_2=2+3i$,则$a=\epsilon_1b\cdot 3$,其中$b=-4+7i, 8+i, -4-7i, 8-i$,故方程有$16$组整数解:$(\pm 12, \pm 21), (\pm 21, \pm 12), (\pm 3, \pm 24), (\pm 24, \pm 3)$.

\subsection{}
利用正文的方法研究如下问题:
\subsubsection{(1)}
对于正整数$n$,$x^2+2y^2=n$何时有整数解?有多少组整数解?

\jie 由{\heiti 习题}\textbf{4.1.9(1)}知$\mathbb{Z}[\sqrt{-2}]$是ED,我们称之为2-高斯整数环,元素称为2-高斯整数,其素元称为2-高斯素数. 定义$\varphi: \mathbb{Z}[\sqrt{-2}]\rightarrow\mathbb{N}, a+b\sqrt{-2}\mapsto a^2+2b^2$,类似正文我们有(请读者自行补充详细):

2-高斯整数环的单位群$U(\mathbb{Z}[\sqrt{-2}])=\{\pm 1\}$.

(1)设$p$为素数,则$p$或为2-高斯素数或为两个共轭的2-高斯素数之积.

(2)设$\pi_0$为2-高斯素数,则$\pi_0\overline{\pi}_0$或为素数,或为素数的平方.

(3)素数$p$为2-高斯素数当且仅当$p\equiv 5,7 \mod 8$.

(4)设$p$为素数,则下列条件等价:

(i) $p$为两个共轭的2-高斯素数之积.

(ii) $p=a^2+2b^2,\;a,b\in\mathbb{Z}.$

(iii) $x^2=-2\mod p$有整数解,即$\left(\frac{-2}{p}\right)_{\mathrm{Le}}=1$或$0$.

(iv) $p\equiv 1, 3\mod 8$或$p=2$.

\zm{
	(1)(2)与正文类似.
	
	(3)
	注意奇素数模$8$只能余$1,3,5,7$. 我们只要证$p\equiv 5,7\mod 8$则$p$是2-高斯素数,另一方面的证明由(4)即知. 如果$p$不是高斯素数,则
	
	$$p=\pi_0\overline{\pi}_0=\varphi(\pi_0)=a^2+2b^2, \pi_0=a+b\sqrt{-2},$$
	
	则$p\equiv 0,1,2,3,4,6\mod 8$,即$p\not\equiv 5,7\mod 8$.
	
	(4) (i) $\Leftrightarrow$ (ii) 显然.
	
	(ii) $\Rightarrow$ (iii) 如果$p=a^2+2b^2$,则$0<2b^2<p$,故$b$在$\mathbb{F}_p$中可逆,令$\overline{c}=\overline{b}^{-1}$,则$(ac)^2+2=a^2c^2+2b^2c^2\equiv 0\mod p$,即$ac$满足条件.
	
	(iii) $\Leftrightarrow$ (iv) 如果$p$为奇素数,则$\left(\frac{-2}{p}\right)_{\mathrm{Le}}=\left(\frac{-1}{p}\right)_{\mathrm{Le}}\left(\frac{2}{p}\right)_{\mathrm{Le}}=(-1)^{\frac{p-1}{2}}(-1)^{\frac{p^2-1}{8}}$,前者在$p\equiv 1,5\mod 8$时为$1$,$p\equiv 3,7\mod 8$时为$-1$,后者在$p\equiv 1,7\mod 8$时为$1$,$p\equiv 3,5\mod 8$时为$-1$,两者相乘即得结论. 如果$p=2$,取$x=0$即可.
	
	(iii) $\Leftarrow$ (i) 由$p\mid x^2+2=(x+\sqrt{-2})(x-\sqrt{-2})$,但易见$p\nmid x\pm\sqrt{-2}$,则$p$不是素元,由(1)知$p$必是共轭2-高斯素数之积.
}

故设$n\geq 1$的(整数-)素因子分解为$2^{\alpha}\prod_{i=1}^sp_i^{\beta_i}\prod_{j=1}^tq_j^{\gamma_j}$,其中$p_i\equiv 1,3\mod 8$,$q_j\equiv 5,7\mod 8$,则$n=a^2+2b^2$有整数解当且仅当$\gamma_j\;(1\leq j\leq t)$全为偶数,此时共有$2\prod_{i=1}^s(\beta_i+1)$对解. (当$n=1$时,$\alpha$和$\beta_i, \gamma_j$全部为$0$,有$2$组解)

\zm{
	与正文一致,但把$1\pm i$换作$\pm 2i$,同时注意单位$\epsilon$只有$2$种而不是$4$种可能的取值.
}

\subsubsection{(2)}
对于正整数$n$,$x^2+xy+y^2=n$何时有整数解?有多少组整数解?


\jie 由{\heiti 习题}\textbf{4.1.9(3)}知$\mathbb{Z}[\omega]$是ED,我们称之为$\omega$-高斯整数环,元素称为$\omega$-高斯整数,其素元称为$\omega$-高斯素数. 定义$\varphi: \mathbb{Z}[\omega]\rightarrow\mathbb{N}, a-b\omega\mapsto a^2+ab+b^2$,(注意,$a^2+ab+b^2=(a+b/2)^2+3/4\cdot b^2$,故当$a,b$不全为零时$\varphi(a+b\omega)$为正)类似正文我们有(请读者自行补充详细):

$\omega$-高斯整数环的单位群$U(\mathbb{Z}[\omega])=\{\pm 1, \pm\omega, 1\pm\omega\}$.

(1)设$p$为素数,则$p$或为$\omega$-高斯素数或为两个共轭的$\omega$-高斯素数之积.

(2)设$\pi_0$为$\omega$-高斯素数,则$\pi_0\overline{\pi}_0$或为素数,或为素数的平方.

(3)素数$p$为$\omega$-高斯素数当且仅当$p\equiv 2\mod 3$.

(4)设$p$为奇素数,则下列条件等价:

(i) $p$为两个共轭的$\omega$-高斯素数之积.

(ii) $p=a^2+ab+b^2,\;a,b\in\mathbb{Z}.$

(iii) $x^2=-3\mod p$有整数解,即$\left(\frac{-3}{p}\right)_{\mathrm{Le}}=1$或$0$.

(iv) $p\equiv 1\mod 3$或$p=3$.

\zm{
	(1)(2)与正文类似.
	
	(3)
	注意非$3$的素数模$3$只能余$1,2$. 我们只要证$p\equiv 2\mod 3$则$p$是$\omega$-高斯素数,另一方面的证明由(4)即知. 如果$p$不是高斯素数,则
	
	$$p=\pi_0\overline{\pi}_0=\varphi(\pi_0)=a^2+ab+b^2, \pi_0=a-b\omega, \overline{\pi}_0=a-b\omega^2$$
	
	当$a^2$或$b^2\equiv 0\mod 3$时$ab\equiv 0\mod 3$,由于$b^2$或$a^2$模$3$不可能余$2$,故$p=a^2+b^2+ab\equiv 0$或$1\mod 3$.
	
	当$a^2\equiv b^2\equiv 1\mod 3$时$ab$不可能被$3$整除,故故$p=a^2+b^2+ab\equiv 1+1+1$或$1+1+2\mod 3\equiv 0$或$1\mod 3$,综上$p\not\equiv 2\mod 3$.
	
	(4) (i) $\Leftrightarrow$ (ii) 显然.
	
	(ii) $\Rightarrow$ (iii) 如果$p=a^2+ab+b^2$,则$0<3/4\cdot b^2<p$,又$p\geq 2$,故$b$在$\mathbb{F}_p$中可逆,令$\overline{c}=\overline{b}^{-1}$,则$(2ac+1)^2+3=4((ac)^2+ac+1)=4(a^2c^2+abc^2+b^2c^2)\equiv 0\mod p$,即$2ac+1$满足条件.
	
	(iii) $\Leftrightarrow$ (iv) 如果$p$为奇素数,则$\left(\frac{-3}{p}\right)_{\mathrm{Le}}=\left(\frac{-1}{p}\right)_{\mathrm{Le}}\left(\frac{3}{p}\right)_{\mathrm{Le}}=(-1)^{\frac{p-1}{2}}\left(\frac{p}{3}\right)_{\mathrm{Le}}^{-1}(-1)^{\frac{3-1}{2}\cdot\frac{p-1}{2}}=\left(\frac{p\mod 3}{3}\right)_{\mathrm{Le}}$,当$p\equiv 1\mod 3$时为$1$,当$p\equiv 2\mod 3$时为$-1$.如果$p=3$,取$x=0$即可.
	
	(iii) $\Leftarrow$ (i) 由$p\mid x^2+3=(x-1)^2+2(x-1)+4=(x-1-2\omega)(x-1-2\omega^2)=(x-1-2\omega)(x+1+2\omega)$,但易见$p\nmid x\pm1\pm2\omega$,则$p$不是素元,由(1)知$p$必是共轭$\omega$-高斯素数之积.
}

注意到$p=2$时$a^2+ab+b^2$无整数解,此时$p$为$\omega$-高斯素数.

故设$n\geq 1$的(整数-)素因子分解为$3^{\alpha}\prod_{i=1}^sp_i^{\beta_i}\prod_{j=1}^tq_j^{\gamma_j}$,其中$p_i\equiv 1\mod 3$,$q_j\equiv 2\mod 3$,则$n=a^2+ab+b^2$有整数解当且仅当$\gamma_j\;(1\leq j\leq t)$全为偶数,此时共有$6\prod_{i=1}^s(\beta_i+1)$对解. (当$n=1$时,$\alpha$和$\beta_i, \gamma_j$全部为$0$,有$6$组解$(\pm 1, 0), (0, \pm 1), (\pm 1, \mp 1)$)

\zm{
	与正文一致,但把$1+i$换作$2+\omega$,$1-i$换作$2+\omega^2=1-\omega$,同时注意单位$\epsilon$有$6$种而不是$4$种可能的取值.
}
