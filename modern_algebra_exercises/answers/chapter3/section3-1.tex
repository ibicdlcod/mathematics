\section{环和域的定义}
\subsection{}
设$n\geq 2$是正整数,
\subsubsection{(1)}
环$\mathbb{Z}/n\mathbb{Z}$中元素$a$可逆的充要条件是$\gcd(a,n)=1$.
\subsubsection{(2)}
若$p$为素数,则$\mathbb{Z}/p\mathbb{Z}$为域,若$n\geq 2$不为素数,则
$\mathbb{Z}/n\mathbb{Z}$不是整环.

\Proofbyintimidation

\subsection{}
证明$\mathbb{H}$中任何非零元素均乘法可逆.

\zm{
	令$p=t+xi+yj+zk\neq 0\in\mathbb{H}$,我们有$t^2+x^2+y^2+z^2\neq 0$.
	
	定义$p$的共轭$p^{\ast}=t-xi-yj-zk$,则$pp^{\ast}=p^{\ast}p
	=(t^2+x^2+y^2+z^2)
	+(tx-tx-yz-(-yz))i
	+(ty-ty-zx-(-zx))j
	+(tz-tz-xy-(-xy))k
	=t^2+x^2+y^2+z^2\in\mathbb{R}-\{0\}$
	
	故$p^{-1}=p^{\ast}/(t^2+x^2+y^2+z^2)$满足$pp^{-1}=p^{-1}p=1$.
}

\subsection{}
设$d\geq 1$为正整数,利用$R=\mathbb{Z}[\sqrt{-d}]\subseteq \mathbb{C}$说明$R$是整环,
并确定$R$的单位群.

\jie $R$的零因子也是$\mathbb{C}$的零因子,然而后者没有零因子,故$R$是整环.

对$R$中任意元素$y=a+b\sqrt{-d}\neq 0$,$y$在$R$中可逆$\Leftrightarrow
y^{-1}_{\mathbb{C}}=\frac{a-b\sqrt{-d}}{a^2+b^2d}\in R$,即$a^2+b^2d\mid a, a^2+b^2d\mid -b$.

若$|a|>1$,则$a^2>|a|,a^2+b^2d\geq a^2>|a|$,与整除矛盾.

若$|a|=1$,则$1+b^2d\mid \pm1$,又$d\geq 1$,只能$b^2=0$,即$y=\pm 1$.

若$|a|=0$,则$b^2d\mid -b$,得$bd=\pm 1$或$0$,又$d\neq 0, a^2+b^2d\neq 0$,只能$b=\pm 1$且$d=1$.

因此$U(R)=\left\{
\begin{matrix}
\{\pm 1\}, & \text{当}d>1\\
\{\pm 1, \pm\sqrt{-1}\}, & \text{当}d=1
\end{matrix}
\right.$.

\subsection{}
设$A$是阿贝尔群(其上的运算称为加法),$\mathrm{End}(A)$是群$A$的全部自同态构成的集合,对于$f,g\in\mathrm{End}(A)$,定义
$$(f+g)(a)=f(a)+g(a), (f\cdot g)(a)=f(g(a)), \forall a\in A$$
证明$\mathrm{End}(A)$对上述运算是含幺环,并求出单位群.

\zm{
	定义$(-g)(a)=-g(a), \forall a\in A$,则对任意$f,g\in\mathrm{End}(A)$,$-g$也是$A$到自身的同态,$(f+(-g))(a)=f(a)-g(a)=-g(a)+f(a)=(-g+f)(a)$,故$\mathrm{End}(A)$对加法是阿贝尔群(结合律请自证),并有加法单位元:$f_0(a): a\mapsto 0_A$.
	
	又$((fg)\cdot h)(a)=f(g(h(a)))=(f\cdot (gh))(a)$,且$f_1(a):a\mapsto a$满足$f_1\cdot g=g\cdot f_1=g$,故$\mathrm{End}(A)$对乘法为含幺半群.
	
	又$((f+g)\cdot h)(a)=(f+g)(h(a))=f(h(a))+g(h(a))=(f\cdot h+g\cdot h)(a)$(同理可证左分配律),故$\mathrm{End}(A)$和上述运算是含幺环.
}

\jie $\exists f^{-1}$ s.t. $f(f^{-1}(a))=f^{-1}(f(a))=a=f_1(a), \forall a\in A\Leftrightarrow
f$是同构,故$U(\mathrm{End}(A))=\Aut(A)$.

\subsection{}
设$G$是乘法群,$R$为含幺环,定义集合$R[G]=\{\sum_{g\in G}r_gg\mid r_g\in R\text{且只有有限多个}r_g\neq 0_R\}$,在集合$R[G]$上定义
$$\sum_{g\in G}r_gg+\sum_{g\in G}t_gg=\sum_{g\in G}(r_g+t_g)g;\;
\left(\sum_{g\in G}r_gg\right)\left(\sum_{g\in G}t_gg\right)=\sum_{g\in G}
\left(
\sum_{g^{\prime}g^{\prime\prime}=g}r_{g^{\prime}}t_{g^{\prime\prime}}
\right)g$$

\subsubsection{(1)}
求证:上面定义的加法和乘法是集合$R[G]$中的二元运算(警告:这不是显然的!),并且$R[G]$由此形成环,称为群$G$在环$R$上的{\heiti 群环}.

\zm{
	令$T_1=\{g\mid r_g\neq 0\}$,$T_2=\{g\mid t_g\neq 0\}$,由定义他们都是有限集.
	
	对$r+t$的指标集$r_g+t_g$,$\{g\mid r_g+t_g\neq 0\}\subseteq T_1\bigcup T_2 $,故它是有限集,故加法为$R[G]$上的二元运算.
	
	对$rt$的指标集,$g\neq T_1T_2\Rightarrow
	\sum_{g^{\prime}g^{\prime\prime}=g}r_{g^{\prime}}t_{g^{\prime\prime}}
	=\sum_{g^{\prime}g^{\prime\prime}=g}0\cdot0
	=0$,故$\{g\mid \sum_{g^{\prime}g^{\prime\prime}=g}r_{g^{\prime}}t_{g^{\prime\prime}}\neq 0\}\subseteq T_1\cdot_GT_2$,它也是有限集,故乘法为$R[G]$上的二元运算.
	
	$R[G]$的加法为阿贝尔群:留给读者.
	
	$R[G]$的乘法为含幺半群:
	
	结合律:
	\begin{flalign*}
	&\left(\left(\sum_{g\in G}r_gg\right)\left(\sum_{g\in G}t_gg\right)\right)
	\left(\sum_{g\in G}s_gg\right)&&\\
	=&\left(\sum_{g\in G}
	\left(
	\sum_{g^{\prime}g^{\prime\prime}=g}r_{g^{\prime}}t_{g^{\prime\prime}}
	\right)g\right)
	\left(\sum_{g\in G}s_gg\right)&&\\
	=&\sum_{g\in G}
	\left(
	\sum_{g_{12}g^{\prime\prime\prime}=g}
	\left(
	\sum_{g^{\prime}g^{\prime\prime}=g_{12}}r_{g^{\prime}}t_{g^{\prime\prime}}
	\right)
	s_{g^{\prime\prime\prime}}
	\right)
	g&&\\
	=&\sum_{g\in G}
	\left(
	\sum_{g^{\prime}g^{\prime\prime}g^{\prime\prime\prime}=g}
	r_{g^{\prime}}t_{g^{\prime\prime}}s_{g^{\prime\prime\prime}}
	\right)
	g&&\tag{$\ast$}
	\end{flalign*}
	同理可证$\left(\sum_{g\in G}r_gg\right)\left(\left(\sum_{g\in G}t_gg\right)
	\left(\sum_{g\in G}s_gg\right)\right)=(\ast)$.
	
	乘法单位元:
	\begin{flalign*}
	&\left(
	\sum_{g\in G}r_gg
	\right)
	\left(
	\sum_{g\in G}1_gg
	\right)&&(g=1_G\Rightarrow 1_g=1_R, g\neq 1_G\Rightarrow 1_g=0_R)\\
	=&\sum_{g\in G}
	\left(
	\sum_{g^{\prime}g^{\prime\prime}=g}r_{g^{\prime}}1_{g^{\prime\prime}}
	\right)
	g&&\\
	=&\sum_{g\in G}
	\left(
	\sum_{g^{\prime}1_G=g}r_{g^{\prime}}1_R
	\right)
	g+
	\left(
	\sum_{\substack{g^{\prime}g^{\prime\prime}=g\\g^{\prime\prime}\neq 1_G}}r_{g^{\prime}}0_R
	\right)
	g&&\\
	=&\sum_{g\in G}
	\left(
	\sum_{g^{\prime}1_G=g}r_{g^{\prime}}1_R
	\right)
	g&&\\
	=&\sum_{g\in G}r_gg&&\tag{$\ast\ast$}
	\end{flalign*}
	同理可证$
	\left(
	\sum_{g\in G}1_gg
	\right)\left(
	\sum_{g\in G}r_gg
	\right)=(\ast\ast)$.
	
	分配律:只证右分配律,请读者完成左分配律.
	\begin{flalign*}
	&\left(\left(\sum_{g\in G}r_gg\right)+\left(\sum_{g\in G}t_gg\right)\right)
	\left(\sum_{g\in G}s_gg\right)&&\\
	=&\left(\sum_{g\in G}(r_g+t_g)g\right)
	\left(\sum_{g\in G}s_gg\right)&&\\
	=&\sum_{g\in G}
	\left(
	\sum_{g^{\prime}g^{\prime\prime}=g}(r_{g^{\prime}}+t_{g^{\prime}})s_{g^{\prime\prime}}
	\right)
	g&&\\
	=&\sum_{g\in G}
	\left(
	\sum_{g^{\prime}g^{\prime\prime}=g}(r_{g^{\prime}}s_{g^{\prime\prime}}+t_{g^{\prime}}s_{g^{\prime\prime}})
	\right)
	g&&\\
	=&\sum_{g\in G}
	\left(
	\sum_{g^{\prime}g^{\prime\prime}=g}r_{g^{\prime}}s_{g^{\prime\prime}}
	\right)g
	+
	\sum_{g\in G}
	\left(
	\sum_{g^{\prime}g^{\prime\prime}=g}t_{g^{\prime}}s_{g^{\prime\prime}}
	\right)g&&\\
	=&\left(\sum_{g\in G}r_gg\right)
	\left(\sum_{g\in G}s_gg\right)
	+\left(\sum_{g\in G}t_gg\right)
	\left(\sum_{g\in G}s_gg\right)&&
	\end{flalign*}
	故$R[G]$有环结构.
}
\subsubsection{(2)}
$R[G]$是交换环$\Leftrightarrow R$是交换环且$G$是阿贝尔群.

$(\Leftarrow)$请读者自证.

$(\Rightarrow)$反证法,当$R$不是交换环时,$\exists r_1t_1\neq t_1r_1, r_1,t_1\in R$,
令$r_g=r_1, t_g=t_1$当$g=1_G$,$r_g=t_g=0_R$当$g\neq 1_G$,则
$\left(\sum_{g\in G}r_gg\right)\left(\sum_{g\in G}t_gg\right)=r_1t_11_G$,
$\left(\sum_{g\in G}t_gg\right)\left(\sum_{g\in G}r_gg\right)=t_1r_11_G$,
两者不等,故$R[G]$不是交换环.

当$G$不是阿贝尔群时,$\exists ab\neq ba, a,b\in G$,令$r_g=1_R$当$g=a$,$r_g=0_R$当$g\neq a$,$t_g=1_R$当$g=b$,$t_g=0_R$当$g\neq b$,则
$\left(\sum_{g\in G}r_gg\right)\left(\sum_{g\in G}t_gg\right)=1_Rab$,
$\left(\sum_{g\in G}t_gg\right)\left(\sum_{g\in G}r_gg\right)=1_Rba$,
两者不等,故$R[G]$不是交换环.

由反证法即得结论.
\subsubsection{(3)}
如果环$R$的单位元为$1_R$,群$G$的单位元为$e$,则$1_Re$是群环$R[G]$的单位元.

\Proofbyintimidation

\subsubsection{(4)}
可以将$R$自然看作$R[G]$的子环:

\jie $R$到$R[G]$有单同态$f: r\mapsto r1_G$,证明留给读者.

\subsubsection{(5)}
试确定
\paragraph{(5a)}
$\mathbb{Z}[\mathbb{Z}/2\mathbb{Z}]$的单位群.

\jie $\mathbb{Z}[\mathbb{Z}/2\mathbb{Z}]$的元素有$a\cdot 1+b\cdot x$的形式,其中$x^2=1$,$1\cdot x=x\cdot 1=x$,故$a\cdot 1+b\cdot x$可记为$a+bx\;(a,b\in\mathbb{Z})$.

由于$\mathbb{Z}[\mathbb{Z}/2\mathbb{Z}]$的单位元为$1+0x$,则
\begin{flalign*}
&(a+bx)(c+dx)=1+0x&&(a,b,c,d\in\mathbb{Z})\\
\Leftrightarrow&ac+bd=1,ad-bc=1&&(a,b,c,d\in\mathbb{Z})\\
\Rightarrow&a^2+b^2=1,c^2+d^2=1,ad-bc=0&&(a,b,c,d\in\mathbb{Z})\\
\Rightarrow&a+bx=\pm 1,\pm x, c+dx=\mp 1, \mp x
\end{flalign*}
易验证上述四个元素的确是$\mathbb{Z}[\mathbb{Z}/2\mathbb{Z}]$的单位,即
$U(\mathbb{Z}[\mathbb{Z}/2\mathbb{Z}])=\{\pm 1, \pm x\}$.

\paragraph{(5b)}
$R[\mathbb{Z}]$的单位群,其中$R$为整环.

\jie 将$\mathbb{Z}$看作秩为$1$的自由阿贝尔群,其生成元记为$x$,则$R[\mathbb{Z}]$中的非零元有形式$r=x^{u}(a_0+a_1x+\cdots+a_mx^m)=x^uf(x), 0\leq m<\infty, u\in\mathbb{Z}, a_i\in R$,
$s=x^{v}(b_0+b_1x+\cdots+b_nx^n=x^vg(x), 0\leq n<\infty, v\in\mathbb{Z}, b_j\in R$.
此时$f,g$为多项式环$R[x]$中的元素且常数项不为$0_R$,由$R$是整环,$f(x)g(x)$的常数项也不为$0_R$,且$rs=x^{u+v}f(x)g(x)$,要使$rs=1_{R[\mathbb{Z}]}$,只能$u+v=0$,$f(x)g(x)=1_R$,我们有$\deg fg=0$,由{\heiti 命题}\textbf{4.21}(读者也可自证)$\deg f+\deg g=0$,即$f,g$为常数多项式,$fg=1$得到$f=a_0\in U(R), g=b_0\in U(R)$.

故$U(R[\mathbb{Z}])$为$\{x^ua_0\mid a_0\in U(R), u\in\mathbb{Z}\}$.

\subsection{}
令$R=\{a=(a_1,a_2,...)\mid a_n\in \mathbb{Z},0\leq a_n\leq p^n-1,a_n\equiv a_{n+1}\mod p^n\}$,设$a,b\in R$,定义
$$a+b=c,0\leq c_n\leq p^n-1, c_n\equiv a_n+b_n\mod p^n,$$
$$ab=d,0\leq d_n\leq p^n-1, d_n\equiv a_nb_n\mod p^n.$$
\subsubsection{(1)}
$R$成为一个含幺交换环,称为$p$进整数环,记为$\mathbb{Z}_p$.

\zm{
	我们只证两个元素的和、积满足仍满足$f_n\equiv f_{n+1}\mod p^n$,其余请读者自己完成. 由于$p^n$整除$p^{n+1}$,我们有
	$a+b=c\Rightarrow c_{n+1}\equiv a_{n+1}+b_{n+1}\mod p^{n+1}\equiv a_{n+1}+b_{n+1}\mod p^n\equiv a_n+b_n\mod p^n\equiv c_n\mod p^n$.
	$ab=d\Rightarrow d_{n+1}\equiv a_{n+1}b_{n+1}\mod p^{n+1}\equiv a_{n+1}b_{n+1}\mod p^n\equiv a_nb_n\mod p^n\equiv c_n\mod p^n$.
	
	请读者自己证明$(1,1,1,...)$是$R$中的单位元.
}
\subsubsection{(2)}
$\mathbb{Z}$可自然看成是$\mathbb{Z}_p$的子环.

\zm{
	请读者自己证明$\mathbb{Z}\rightarrow \mathbb{Z}_p, a_0\mapsto (a_0\mod p, a_0\mod p^2, ...)$为单射且保持环$\mathbb{Z}$的加法与乘法.
}
\subsubsection{(3)}
试确定$\mathbb{Z}_p$的单位群.

\jie 对任意环$R$中元素$a=(a_1,a_2,...)$,若$a_1=0$,则对环$R$中任意元素$b$,$ab=(0b_1, a_2b_2,...)=(0,a_2b_2,...)\neq(1,1,...)=1_R$,故$a$不是单位. 若$a_1\neq 0$,则$\gcd(a_1,p)=1$,并且我们有$a_i=m_ip+a_1, m_i\in\mathbb{N}$,$\gcd(a_i,p^i)=1$. 由贝祖定理,存在$b_i$使得$a_ib_i+p^ik=\gcd(a_i,p^i)=1$,即$a_ib_i\equiv 1\mod p^i$,且$a_nb_{n+1}\equiv a_{n+1}b_{n+1}\mod p^n\equiv (1\mod p^{n+1})\mod p^n\equiv 1\mod p^n$,故$a_n(b_{n+1}-b_n)\equiv 0\mod p^n$,由$p\nmid a_n$知$b_{n+1}-b_n\equiv 0\mod p^n$,即$b_{n+1}\equiv b_n\mod p^n$对一切$n\geq 1$成立. 故$b=(b_1,b_2,...)\in R$满足$ab=(1,1,...)$是$a$在$R$中的逆元,$U(R)=\{(a_1,a_2,...)\mid a_n\in\mathbb{Z}, a_1\neq 0, 0\leq a_n\leq p^n-1,a_n\equiv a_{n+1}\mod p^n\}=\mathbb{Z}_p-p\mathbb{Z}_p$.

\subsection{}
设$d\in\mathbb{Q}^{\times}-(\mathbb{Q}^{\times})^2$. 证明$\mathbb{Q}[\sqrt{d}]=\{a+b\sqrt{d}\mid a,b\in \mathbb{Q}\}$是$\mathbb{C}$的子域,并确定$\mathbb{Q}[\sqrt{d}]$的全部子域.

\zm{
	请读者自证$\mathbb{Q}[\sqrt{d}]\subseteq\mathbb{C}$,我们只需证$\mathbb{Q}[\sqrt{d}]$对加减乘除封闭. 加减法请读者自证. 对任意该集合中元素$a+b\sqrt{d}, c+e\sqrt{d},\;(a,b,c,d\in\mathbb{Q})$,我们有
	$(a+b\sqrt{d})(c+e\sqrt{d})=(ac+bed)+(ae+bc)\sqrt(d)\in\mathbb{Q}[\sqrt{d}]$,当$a+b\sqrt{d}\neq 0$时$a,b$不同时为$0$,又$d\notin(\mathbb{Q}^{\times})^2$,有$d\neq (\frac{a}{b})^2$,即$a^2-b^2d\neq 0$,我们有$(a+b\sqrt{d})(\frac{a-b\sqrt{d}}{a^2-b^2d})=1$其中$(\frac{a-b\sqrt{d}}{a^2-b^2d})=a/(a^2-b^2d)-b/(a^2-b^2d)\sqrt{d}\in\mathbb{Q}[\sqrt{d}]$,故每个非零元均有逆元.
}

\jie 对任何$\mathbb{Q}[\sqrt{d}]$的子域$F$,由于$1\in F$,得到$\mathbb{Q}\subseteq F$,并且$\mathbb{Q}$满足条件. 若存在$e\neq 0$使得$c+e\sqrt{d}\in F$,则$\sqrt{d}=\frac{1}{e}\cdot(c+e\sqrt{d}-c)\in F$,得$\forall a,b\in\mathbb{Q}, a+b\sqrt{d}\in F$,即$F=\mathbb{Q}[\sqrt{d}]$,故子域只有两个,$\mathbb{Q}$和$\mathbb{Q}[\sqrt{d}]$本身.

\subsection{}
设$R$为环,$a\in R$,求证$S=\{r\in R\mid ar=ra\}$为$R$的子环.

\zm{
	易见$0_R\in S, 1_R\in R\Rightarrow 1_R\in S$,我们只要证明$\forall r,s\in S, r-s\in S, rs\in S$.
	
	$(r-s)a=ra-sa=ar-as=a(r-s), (rs)a=r(sa)=ras=ars=a(rs)$.
}

\subsection{}
设$U$是一个集合,$S=\{X\mid X\subseteq U\}$,对$A,B\in S$,定义
$$A-B=\{c\in U\mid c\in A,c\notin B\},$$
$$A+B=(A-B)\cup(B-A), A\cdot B=A\cap B.$$
求证$(S,+,\cdot)$是含幺交换环.

\zm{
	加法和乘法的二元运算性、交换性留给读者自证.
	
	加法结合律:
	\begin{flalign*}
	&(A+B)+C&&\\
	=&((A-B)\cup(B-A))+C&&\\
	=&(((A-B)\cup(B-A))-C)\cup(C-((A-B)\cup(B-A)))&&\\
	=&((A-(B\cup C))\cup(B-(A\cup C)))\cup((C-(A\cup B))\cup (C\cap A\cap B))&&\\
	=&(A-(B\cup C))\cup(B-(A\cup C))\cup (C-(A\cup B))\cup (A\cap B\cap C)&&\tag{$\ast$}
	\end{flalign*}
	注意$(\ast)$式关于$A,B,C$对称,故$(\ast)=(A+B)+C=(B+C)+A=A+(B+C)$.
	
	加法单位元:$0_S=\varnothing$满足$A+\varnothing=\varnothing+A=(A-\varnothing)\cup(\varnothing-A)=A$.
	
	加法逆元:$\forall A\in S, A+A=(A-A)\cup(A-A)=\varnothing=0_S$,故任何元素的逆元是它自身.
	(警告:这里$A+(-B)=A+B\neq A-B$,故上文定义的$A-B$不是$S$的减法)
	
	乘法结合律:
	$(A\cdot B)\cdot C=A\cap B\cap C=A\cdot(B\cdot C)$.
	
	乘法单位元:
	$1_S=U$满足$A\cdot 1_S=1_S\cdot A=A\cap U=A$.
	
	分配律:
	\begin{flalign*}
	&(A+B)\cdot C&&\\
	=&((A-B)\cup(B-A))\cap C&&\\
	=&((A\cap C)-(B\cap C))\cup((B\cap C)-(A\cap C))&&\\
	=&(A\cap C)+(B\cap C)&&\\
	=&A\cdot C+B\cdot C&&
	\end{flalign*}
	同理可证左分配律.
	
	综上,$(S,+,\cdot)$是含幺交换环.
}

\subsection{}
设$R$为环,如果每个元素$a\in R$均满足$a^2=a$,称$R$为{\heiti 布尔环}\textbf{(Boolean ring)},求证:
\subsubsection{(1)}
布尔环必交换,且对任意$a\in R, a+a=0_R$.

\zm{
	$a+a=(a+a)^2=a^2+a^2+a^2+a^2=(a+a)+(a+a)$,故$0_R=a+a$.
	
	对任意元素$a,b$,$a+b=(a+b)^2=a^2+b^2+ab+ba=a+b+ab+ba$,故$ab+ba=0_R$,但$ab+ab=0_R$,有$ab-ba=0_R$恒成立,$R$是交换环.
}
\subsubsection{(2)}
{\heiti 习题}\textbf{3.1.9}中的环$S$是布尔环.

\Proofbyintimidation

\subsection{}
非零有限整环$R$必为域.

\zm{
	因$1_R\in R$,若$R$无$0_R,1_R$以外元素,则$R\cong \mathbb{F}_2$为域,否则对任意$x\neq 0_R,1_R$,令$\varphi: \mathbb{N}\rightarrow
	\{x^i\mid i\in\mathbb{N}\}, i\mapsto x^i$,由于$\{x^i\mid i\in\mathbb{N}\}\subseteq R$是有限集,$\mathbb{N}$是无限集,故$\varphi$不是单射,$\exists i\neq j$使得$x^i=x^j$,不妨设$j>i$,即$j\geq i+1$,则$(x^{j-i}-1_R)x^i=x^j-x^i=0_R$,由$R$是整环,$x$不是零因子,导致$x^i\neq 0_R$,并且它也不是零因子,只能$(x^{j-i}-1_R)=0$,即$x^{j-i}=1_R, x^{j-i-1}x=1_R$,$x^{j-i-1}$为$x$的逆元.
}

\subsection{}
环$R$中元素$a$叫做幂零的,是指存在正整数$m$使得$a^m=0$.
\subsubsection{(1)}
证明:若$R$是交换环,$a,b$为幂零元素,则$a+b$也是幂零元素.

\zm{
	若$a^n=0, b^m=0$,则
	\begin{flalign*}
	&(a+b)^{n+m}=
	\sum_{0\leq i\leq m+n}\left(
	\begin{matrix}
	m+n\\
	i
	\end{matrix}
	\right)a^ib^{m+n-i}&&\\
	=&\sum_{n\leq i\leq m+n}\left(
	\begin{matrix}
	m+n\\
	i
	\end{matrix}
	\right)a^na^{i-n}b^{m+n-i}
	+
	\sum_{0\leq i<n}\left(
	\begin{matrix}
	m+n\\
	i
	\end{matrix}
	\right)a^ib^{m}b^{n-i}&&\\
	=&\sum_{n\leq i\leq m+n}\left(
	\begin{matrix}
	m+n\\
	i
	\end{matrix}
	\right)0\cdot a^{i-n}b^{m+n-i}
	+
	\sum_{0\leq i<n}\left(
	\begin{matrix}
	m+n\\
	i
	\end{matrix}
	\right)a^i\cdot 0\cdot b^{n-i}&&\\
	=&\;0
	\end{flalign*}
	故$a+b$为幂零元.
}

\subsubsection{(2)}
如果$R$不为交换环,则(1)中结论是否仍旧成立?

\jie 未必成立,令$R=M_2\mathbb{Z}$,$a=
\begin{pmatrix}
0 & 1\\0 & 0
\end{pmatrix}, b=
\begin{pmatrix}
0 & 0\\1 & 0
\end{pmatrix}\in R$,则$a^2=b^2=
\begin{pmatrix}
0 & 0\\0 & 0
\end{pmatrix}=0_R, a+b=
\begin{pmatrix}
0 & 1\\1 & 0
\end{pmatrix}, (a+b)^2=I_2, (a+b)^n=I_2$或$a+b$,故$a+b$不是幂零元.

\subsubsection{(3)}
证明如果$x$是幂零的,那么$1+x$是单位.

\zm{
	若$x^n=0\;(n\in\mathbb{Z}_+)$,则令$y=(1-x+x^2-\cdots+(-1)^{n-1}x^{n-1})$,有
	$y(1+x)=(1+x)y=1+(-1)^nx^n=1$,故$1+x$是单位.
}

\subsection{}
设$a,b$是含幺环$R$中的元素,则$1-ab$可逆$\Leftrightarrow 1-ba$可逆.

\zm{
	($\Rightarrow$)$(1-ba)^{-1}=1+ba+baba+\cdots
	=1+b\cdot 1\cdot a+b(ab)a+b(abab)a+\cdots
	=1+b(1+ab+abab+\cdots)a
	=1+b(1-ab)^{-1}a$.
	证明此式严格成立需要更深知识,参见\emph{文献}\cite{31790}. 这里我们不做,而是直接证明结果成立:
	
	已知$1-ab$可逆,令$u=1+b(1-ab)^{-1}a\in R$,则$u(1-ba)
	=1-ba+b(1-ab)^{-1}a-b(1-ab)^{-1}aba
	=1-ba+(b(1-ab)^{-1}-b(1-ab)^{-1}ab)a
	=1-ba+b(1-ab)^{-1}(1-ab)a
	=1-ba+ba
	=1$.
	$(1-ba)u
	=1-ba+b(1-ab)^{-1}a-bab(1-ab)^{-1}a
	=1-ba+b((1-ab)^{-1}a-ab(1-ab)^{-1}a)
	=1-ba+b(1-ab)(1-ab)^{-1}a
	=1-ba+ba
	=1$.
	故存在$u$是$1-ba$的逆元.

	($\Leftarrow$)交换$a,b$的地位即可.
}

\subsection{}
含幺环中元素有多于一个右逆$\Rightarrow$它有无限多个右逆.

\emph{证明思路由文献}\cite{kaplansky}\emph{给出,但细节为作者补完}


\zm{
	设$u$有多个右逆. 若$u$有左逆,则由{\heiti 引理}\textbf{3.12(1)},$u$的左逆等于右逆且唯一,矛盾. 故$u$没有左逆.
	
	考虑元素$v=v_0+(1-v_0u)u^k$,其中$k\in\mathbb{N}$且$uv_0=1$,即$v_0$是$u$的任何一个右逆. 我们有$uv=uv_0+u^{k+1}-uv_0uu^k=1+u^{k+1}+u\cdot 1\cdot u^k=1$,故$v$是$u$的右逆. 当$i\neq j$时,不妨设$j>i$,则$v_i=(1-v_0u)u^i=v_j=(1-v_0u)u^j
	\Rightarrow u^i-u^j-v_0u^{i+1}+v_0u^{j+1}=0
	\Rightarrow (u^i-u^j-v_0u^{i+1}+v_0u^{j+1})v_0^j=0
	\Rightarrow v_0^{j-i}-1+v_0v_0^{j-i-1}+v_0u=0
	\Rightarrow v_0u=1$,即$v_0$是$u$的左逆,矛盾,故$f:\mathbb{N}\Rightarrow\{v_0+(1-v_0u)u^k\mid k\in\mathbb{N}\}, k\mapsto v_0+(1-v_0u)u^k$是单射,满足条件的$v$有无限多个.
}

\zm{
	另一个证明,令$S=\{x\mid ux=1\}, T=\{xu-1+s_0\mid x\in S\}$,其中固定$s_0\in S$.
	
	由$u$没有左逆知$xu-1+s_0\neq s_0$,故$s_0\in S-T$. 但$\forall y\in T,
	y=xu-1+s_0$,有$uy=uxu-u+us_0=u-u+1=1$,故$T\subseteq S$. 令$f: S\rightarrow T,
	x\mapsto xu-1+s$,则由$u$没有左逆,$x\neq z\Rightarrow (x-z)u\neq 0\Rightarrow
	f(x)=xu-1+s\neq zu-1+s=f(z)$,即$f$是单射,由$T$的定义知$f$是满射,故$S$到其真子集$T$有双射$f$,这迫使$S$为无限集.
}

\subsection{}
令$C(\mathbb{R})$为连续实函数$f:\mathbb{R}\rightarrow \mathbb{R}$构成的集合,定义$(f+g)(a)=f(a)+g(a),(fg)(a)=f(a)g(a),\forall a\in\mathbb{R}$,证明:
\subsubsection{(1)}
$C(\mathbb{R})$为含幺交换环.

\Proofbyintimidation

\subsubsection{(2)}
$C(\mathbb{R})$是否为整环?

\jie 否,请读者自证$C(\mathbb{R})$中的加法单位元为$f_0: a\mapsto 0,\forall a\in A$.

令$f,g$使得$f^{-1}(0)=A\subsetneqq\mathbb{R}, g^{-1}(0)=B\subsetneqq\mathbb{R}$,$A\cup B=\mathbb{R}$,则易见$fg=f_0$,故该环不是整环.

\subsubsection{(3)}
该环是否有幂零元?

\jie 否,若存在$m\in\mathbb{Z}_+$使得$f^m_{C(\mathbb{R})}=f_0$,则$\forall a\in\mathbb{R}, (f(a))^m=0$,只能$f(a)=0$,故$f=f_0$.

\subsubsection{(4)}
该环的单位群是什么?

\jie 请读者自行证明单位群是$\{f\mid f\in C(\mathbb{R}), \forall a\in R,f(a)\neq 0\}$,由$f$是连续的,该单位群也可写作$\{f\mid f\in C(\mathbb{R}), \forall a\in R,f(a)> 0\}\cup\{f\mid f\in C(\mathbb{R}), \forall a\in R,f(a)< 0\}$.

\subsection{}
设$D$为有限体,证明对任意$a\in D$,$a^{|D|}=a$.

\zm{
	$D^{\times}=D-\{0\}$为乘法群,$|D|=|D^{\times}|+1$,由{\heiti 推论}\textbf{1.61},$\forall a\neq 0\in D$有$a^{|D^{\times}|}=1$,即$a^{|D|}=a^{|D^{\times}|+1}=a$,又$0^{|D|}=0$显然,故得结论.
}