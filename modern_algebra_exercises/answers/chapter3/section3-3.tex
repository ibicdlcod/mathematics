\section{环的同态基本定理}
\subsection{}
证明交换环$R$中全部幂零元素组成的集合$N=\mathrm{Nil}(R)$是环$R$的理想,并且商环$R/N$中只有零元素是幂零的.

\zm{
	若$a,b$为幂零元素,则由{\heiti 习题}\textbf{3.1.12(1)}知$a+b$也是幂零元素,对任意$r\in R$,因$R$交换有$a^k=0\Rightarrow (ar)^k=a^kr^k=0r^k=0$,故$ar$(同理$ra$也)是幂零元素,$N$是$R$的理想.
	
	对于商环$R/N$,若$a+N\in R/N$是幂零元,我们有
	$(a+N)^k=\sum_{i=0}^k\binom{k}{i}a^{k-i}N^i\in N$,由于$\sum_{i=1}^{k}\binom{k}{i}a^{k-i}N^i\in RN+\cdots+RN\in N$,因此
	$a^k=\sum_{i=0}^k\binom{k}{i}a^{k-i}N^i-\sum_{i=1}^{k}\binom{k}{i}a^{k-i}N^i\in N$,故存在$l$使得$(a^k)^l=0\Rightarrow a^{kl}=0$,即$a\in N, a+N=0+N$是$R/N$中的零元素.
}

\subsection{}
线性代数

\subsection{}
设$f: R\rightarrow S$是环之间的同态,$I$和$J$是环$R$和$S$的理想,并且$f(I)\subseteq J$,按如下方式作商环之间的映射:
$$\overline{f}: R/I\rightarrow S/J, \overline{a}\mapsto [f(a)],$$
其中对$a\in R,\overline{a}=a+I$为$R/I$中元素,$[f(a)]=f(a)+J$为$S/J$中元素.

\subsubsection{(1)}
证明上述映射$\overline{f}$是良好定义的,并且是环同态.

\zm{
	如$\overline{a_1}=\overline{a_2}$,则$a_1+I=a_2+I$即$a_1-a_2\in I$,故$f(a_1)-f(a_2)=f(a_1-a_2)\in J$,$[f(a_1)]-f[(a_2)]=f(a_1)-f(a_2)+J=J$为$S/J$中零元素,即$\overline{f}(\overline{a_1})=[f(a_1)]=[f(a_2)]=\overline{f}(\overline{a_1})$,故$\overline{f}$是良好定义的.
	
	又$\overline{f}(\overline{a}+\overline{b})=f(a+b)+J=f(a)+J+f(b)+J=\overline{f}(\overline{a})+\overline{f}(\overline{b}),
	\overline{f}(\overline{a}\overline{b})=f(ab)+J=f(a)f(b)+J=f(a)f(b)+f(a)J+Jf(b)+J$(因$J$是$S$的理想)$=(f(a)+J)(f(b)+J)=\overline{f}(\overline{a})\overline{f}(\overline{b})$,且
	$\overline{f}(\overline{1_R})=f(1_R)+J=1_S+J=1_{S/J}$,故$\overline{f}$是同态.
}

\subsubsection{(2)}
证明$\overline{f}$是环同构$\Leftrightarrow f(R)+J=S$且$I=f^{-1}(J)$.

\zm{
	($\Rightarrow$)留给读者.
	
	($\Leftarrow$)$\overline{f}$是单射:若$\overline{a_1}\neq\overline{a_2}$,则$a_1-a_2\notin I$,由$I=f^{-1}(J)$知$f(a_1-a_2)\notin J$,故$f(a_1)-f(a_2)\notin J$,$[f(a_1)]=f(a_1)+J$与$[f(a_2)]=f(a_2)+J$为$S/J$中不同元素.
	
	$\overline{f}$是满射:对$\forall u+J\in S/J$,则$u\in S$,由于$f(R)+J=S$,则存在$u_0$使得$u_0+j_0=u, j_0\in J, u_0\in f(R)$,故$u+J=u_0+j_0+J=u_0+J$,令$f(r_0)=u_0$,则$\overline{f}(\overline{r_0})=u_0+J=u+J$.
}

\subsection{}
设$(R,+,\cdot)$是含幺环,定义$a\oplus b=a+b+1, a\otimes b=ab+a+b$,证明$(R,\oplus,\otimes)$也是含幺环,并且与环$(R,+,\cdot)$同构.

\zm{
	加法为阿贝尔群:
	
	加法满足结合律:$(a\oplus b)\oplus c=a\oplus (b\oplus c)=a+b+c+1+1\in R$.
	
	加法有单位元:$R$中存在$1_{R,+,\cdot}$的逆元$-1_{R,+,\cdot}$,对任意$a\in R$满足
	$a\oplus(-1)=(-1)\oplus a=a-1+1=a$,故$0_{R,\oplus,\odot}=-1_{R,+,\cdot}$.
	
	加法有逆元:令$\ominus a=-a+(-1)+(-1)$,则$a\oplus(\ominus a)=(\ominus a)\oplus a=-a+(-1)+(-1)+a+1=-1=0_{R,\oplus,\odot}$.
	
	加法交换:$a\oplus b=b\oplus a=a+b+1\in R$.
	
	乘法满足结合律:$(a\odot b)\odot c=(ab+a+b)\odot c=(ab+a+b)c+(ab+a+b)+c
	=abc+bc+ac+ab+a+b+c$,
	$a\odot (b\odot c)=a\odot(bc+b+c)=a(bc+b+c)+(bc+b+c)+a
	=abc+bc+ac+ab+a+b+c$,两者一致.
	
	乘法有单位元:$R$中存在加法单位元$0_{R,+,\cdot}$,对任意$a\in R$满足
	$a\odot 0=a\cdot 0+a+0=a, 0\odot a=0\cdot a+0+本题有错误见文献a=a$,故$1_{R,\oplus,\odot}=0_{R,+,\cdot}$.
	
	分配律:$\lambda \odot (a\oplus b)=\lambda \odot (a+b+1)
	=\lambda a+\lambda b+\lambda+\lambda+a+b+1$,
	$(\lambda \odot a)\oplus (\lambda \odot b)
	=(\lambda a+\lambda+a)\oplus (\lambda b+\lambda+b)
	=\lambda a+\lambda b+\lambda+\lambda+a+b+1$,两者相等.
	$(a\oplus b)\odot\lambda=(a+b+1)\odot\lambda 
	=a\lambda+b\lambda+\lambda+\lambda+a+b+1$,
	$(a\odot\lambda)\oplus(b\odot\lambda)
	=(a\lambda+\lambda+a)\oplus (b\lambda+\lambda+b)
	=a\lambda+b\lambda+\lambda+\lambda+a+b+1$,两者相等.
	
	综上,$(R,\oplus,\odot)$为含幺环.
	
	同构性:令$f: (R,+,\cdot)\rightarrow (R,\oplus,\odot); u\mapsto u-1_{R,+,\cdot}$,
	则$f(a+b)=a+b-1=(a-1)+(b-1)+1=f(a)\oplus f(b)$,
	$f(ab)=ab-1=ab-a-b+1+a-1+b-1=(a-1)(b-1)+(a-1)+(b-1)=f(a)\odot f(b)$,
	$f(1_{R,+,\cdot})=0_{R,+,\cdot}=1_{R,\oplus,\odot}$,易见$f$为双射,故$f$为同构.
}

\subsection{}
\subsubsection{(1)}
证明主理想环的每个同态像也是主理想环.

\zm{
	设$\varphi: R\rightarrow S$为环同态,$R$为主理想环.
	
	则$\ker\varphi$是主理想,$\ker\varphi=(x)_{\Ideal,R}=Rx+xR+RxR$,且$\im\varphi$与$R/\ker\varphi$同构. 设$J$为$\im\varphi$的任意理想,则$f^{-1}(J)$为$R/\ker\varphi$的理想,$f^{-1}(J)+\ker\varphi$为$R$的理想(请读者自证!),因$R$为主理想环,存在$y\in R$满足$f^{-1}(J)+\ker\varphi=Ry+yR+RyR$,故$f^{-1}(J)=R(y-x)+(y-x)R+R(y-x)R$,$J=f(R)f(y-x)+f(y-x)f(R)+f(R)f(y-x)f(R)=\im\varphi f(y-x)+f(y-x)\im\varphi+\im\varphi f(y-x)\im\varphi=(y-x)_{\Ideal, \im\varphi}$为$\im\varphi$的主理想,即$\im\varphi$是主理想环.
}
\subsubsection{(2)}
求证$\mathbb{Z}/m\mathbb{Z}\;(m\geq 1)$为主理想环.

\Proofbyintimidation

\subsection{}
设$S, R_i\;(i\in I)$为环,$R=\prod_{i\in I}R_i$为$R_i$的笛卡尔积.
\subsubsection{(1)}
令$\pi_i: R\rightarrow R_i, (a_j)_{j\in I}\mapsto a_i$,证明$\pi_i$为环同态,这样的环同态称为正则投射.
\subsubsection{(2)}
设对于每个$i\in I, \varphi_i: S\rightarrow R_i$均为环同态,求证存在唯一的环同态$\varphi: S\rightarrow R$使得对任意$i\in I$均有$\pi_i\circ\varphi=\varphi_i$.

\Proofbyintimidation

\subsection{}
设$D$为整环,$m,n$为互素的正整数,$a,b\in D$,若$a^m=b^m, a^n=b^n$,证明$a=b$.

\zm{
	对任意非零元$d\in D$,$D$没有零因子使得$ad=bd\Leftrightarrow (a-b)d=0\Leftrightarrow a-b=0\Leftrightarrow a=b$,即消去律成立:等式两边可消去$d$. 当$a=0$或$b=0$时,由$b$或$a$不是零因子得到$b=0$或$a=0$,结论显然成立,因此我们以下假定$a,b\neq 0$,故$a,b$的任意方幂均不为$0$,满足消去律.
	
	不妨设$m>n$,因$m,n$为正整数,存在$k_0$使得$m=k_0n+r_0, k_0\in\mathbb{Z}_+,0\leq r_0<n$,则$a^m=b^m\Leftrightarrow (a^n)^{k_0}a^{r_0}=(b^n)^{k_0}b^{r_0}\Leftrightarrow(a^n)^{k_0}a^{r_0}=(a^n)^{k_0}b^{r_0}$,故由消去律$a^{r_0}=b^{r_0}$. 此时我们有$\gcd(m,n)=\gcd(m-k_0n,n)=\gcd(n,r_0)$.
	
	由$m,n$的任意性,我们对$n,r_0$同样操作得到$a^{r_1}=b^{r_1}$,其中$n=k_1r_0+r_1, k_1\in\mathbb{Z}_+,0\leq r_1<r_0, \gcd(n,r_0)=\gcd(r_0,r_1)=\gcd(m,n)$.
	
	当$r_{j-1}>0$时继续这个过程,我们有$a^{r_j}=b^{r_j}$,其中$r_{j-2}=k_jr_{j-1}+r_j, k_j\in\mathbb{Z}_+,0\leq r_j<r_{j-1}, \gcd(r_{j-1},r_j)=\gcd(r_{j-2},r_{j-1})=\gcd(m,n)$.
	
	由于$0\leq r_j<r_{j-1}$,该过程可以持续到$r_l=0$,此时$\gcd(r_{l-1},0)=\gcd(m,n)$,我们有$r_{l-1}=\gcd(m,n)$,$a^{\gcd(m,n)}=b^{\gcd(m,n)}$,在本题中$\gcd(m,n)=1$,我们有$a=b$.
}

{\heiti 注记}. 本题的证明即对$a,b$的指数使用欧几里得算法.

\subsection{}
设$I_1,I_2,...,I_n$为$R$的理想,且:

(1) $I_1+\cdots+I_n=R$.

(2) $\forall 1\leq i \leq n, I_i\cap(I_1+\cdots+I_{i-1}+I_{i+1}+\cdots+I_n)=\{0\}$

求证$R\cong \prod_{i=1}^nI_i$.

\zm{
	由$\sum_{i=1}^nI_i=R$知$\forall r\in R$有形式$r^{\prime}=(a_1,a_2,...,a_n)$使得$r=\sum_{i=1}^na_i$且$a_i\in I_i, \forall 1\leq i\leq n$,并且给定$a_1,a_2,...,a_n$其中$a_i\in I_i, \forall 1\leq i\leq n$有$r=\sum_{i=1}^na_i\in R$. 我们只要证明这形式唯一即可,即若$r\in R, r=\sum_{i=1}^na_i=\sum_{i=1}^nb_i, a_i,b_i\in I_i$则$\forall 1\leq i\leq n, a_i=b_i$.
	
	这时$0=r-r=\sum_{i=1}^na_i-b_i, a_i-b_i\in I_i$,$a_k-b_k=-\sum_{\substack{1\leq i \leq n\\i\neq k}}a_i-b_i$,左边属于$I_i$,右边属于$I_1+\cdots+I_{i-1}+I_{i+1}+\cdots+I_n$,由条件(2)知两边必须为$0$,即$a_i=b_i$对一切$1\leq i\leq n$成立,上述形式是唯一的.
}

\subsection{}
环$R$中元素$e$叫做幂等元素,是指$e^2=e$. 若$e$又属于环$R$的中心,则称$e$为中心幂等元素. 设$R$是含幺环,$e$为$R$的中心幂等元素,证明
\subsubsection{(1)}
$1-e$也是中心幂等元素.

\zm{
	已知$e^2=e$,$\forall r\in R$有$er=re$,则
	$(1-e)^2=1-e-e+e^2=1-e+0=1-e, r(1-e)=r-re=r-er=(1-e)r$,故得结论.
}
\subsubsection{(2)}
$eR$和$(1-e)R$均是$R$的理想,且$R\cong eR\times (1-e)R$.

\zm{
	$e,(1-e)\in C(R)$已证. 故$eR=eR+Re+ReR$和$(1-e)R=(1-e)R+R(1-e)+R(1-e)R$为$R$的理想,又$\forall r\in R, r=er+(1-e)r$,故$eR+(1-e)R=R$,且$er_1=(1-e)r_2\Rightarrow e^2r_1=e(1-e)r_2\Rightarrow e^2r_1=0\Rightarrow er_1=0, er_1=(1-e)r_2\Rightarrow (1-e)er_1=(1-e)^2r_2\Rightarrow 0=r_2-er_2-er_2+e^2r_2\Rightarrow 0=(1-e)r_2$,即$eR\cap(1-e)R=\{0\}$,在{\heiti 习题}\textbf{3.3.8}中令$I_1=eR, I_2=(1-e)R$即得结论.
}

\subsection{}
本题有错误见文献

\subsection{}
环$R$中幂等元素集合$\{e_1,e_2,...,e_n\}$叫做正交的,是指当$i\neq j$时$e_ie_j=0$,设$R, R_1, R_2,...,R_n$为含幺环,则下列两个条件等价:

\subsubsection{(1)}
$R\cong R_1\times R_2\times \cdots \times R_N$.
\subsubsection{(2)}
$R$具有正交的中心幂等元集合$\{e_1,e_2,...,e_n\}$ s.t. $e_1+e_2+\cdots e_n=1_R$,且$e_iR\cong R_i\;(\forall 1\leq i \leq n)$

\zm{
	($\Rightarrow$) 取$e_i=(a_{i1}, a_{i2}, ..., a_{in})$ 其中$a_ii=1 \forall i, a_ij=0 \forall i\neq j$,则易验证$\{e_i\mid 1\leq i \leq n\}$满足(2)的条件.
	
	($\Leftarrow$) 由于$e_i\in C(R)$,故$(e_i)_{\Ideal, R}=e_iR$为$R$的理想,$\forall r\in R$有$e_1r+e_2r+\cdots+e_nr=r$故$e_1R+e_2R+\cdots+e_nR=R$,考虑任何元素$e_ir_i\in
	e_iR\cap (e_1R+\cdots+e_{i-1}R+e_{i+1}R+\cdots+e_nR)$,存在$r_j (j\neq i)$使得$e_ir_i=
	\sum_{j\neq i}e_jr_j$,两边同乘$e_i$得$e_iR_i=e_i^2r_i=\sum_{j\neq i}e_ie_jr_j=\sum_{j\neq i}0r_j=0$,故$e_iR\cap (e_1R+\cdots+e_{i-1}R+e_{i+1}R+\cdots+e_nR)=\{0\}$,在{\heiti 习题}\textbf{3.3.8}中令$I_i=e_iR$即得结论.
}
