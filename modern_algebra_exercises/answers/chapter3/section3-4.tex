\section{整环与域}
\subsection{}
设$R$是含幺交换环,$\mathfrak{p}_1, \mathfrak{p}_2, ..., \mathfrak{p}_m$为$R$的素理想,而$A$为$R$的理想,如果$A\subseteq
\mathfrak{p}_1 \cup \mathfrak{p}_2 \cup \cdots \cup \mathfrak{p}_m$,则必存在某个$i\;(1\leq i\leq m)$使得$A\subseteq \mathfrak{p}_i$.

\emph{证明由文献}\cite{igusa}\emph{给出.}

\zm{
	用数学归纳法.
	
	当$m=1$时,结论显然成立. 当$m=2$时,若结论不成立,则存在$a_1,a_2\in A,
	a_1\in \mathfrak{p}_1-\mathfrak{p}_2, a_2\in \mathfrak{p}_2-\mathfrak{p}_1$,若$a_1+a_2\in\mathfrak{p}_1$,则$a_2=a_1+a_2-a_1\in\mathfrak{p}_1-\mathfrak{p}_1\subseteq\mathfrak{p}_1$,矛盾,同理$a_1+a_2\notin\mathfrak{p}_2$,两者综合与$a_1+a_2\in A\subseteq \mathfrak{p}_1\cup\mathfrak{p}_2$矛盾.
	
	假设我们已有$m=k$时结论成立,下证$m=k+1$时结论成立,用反证法. 若不存在$\mathfrak{p}_i$包含$A$,则固定$1\leq i\leq k+1$,因$m=k$时对$\mathfrak{p}_j\;(j\neq i)$结论不成立,则条件也不成立,即$A\nsubseteq \bigcup_{\substack{1\leq j\leq k+1\\j\neq i}}\mathfrak{p}_j$,但$A\subseteq \bigcup_{\substack{1\leq j\leq k+1\\j\neq i}}\mathfrak{p}_j\cup\mathfrak{p}_i$,故存在$a_i\in A$使得$a_i\in \mathfrak{p}_i, a_i\notin \mathfrak{p}_j\;(\forall j\neq i)$.
	
	考虑元素$b=\prod_{i=1}^k a_i, c=b+a_{k+1}$,则对$1\leq i\leq k$有$b\in \mathfrak{p}_i$,且由$\mathfrak{p}_{k+1}$是素理想,$a_i\notin \mathfrak{p}_{k+1}, \forall 1\leq i\leq k$有$b\notin \mathfrak{p}_{k+1}$,又$a_{k+1}\in \mathfrak{p}_{k+1}, a_{k+1}\notin \mathfrak{p}_j\;(j\neq k+1)$,
	若$c\in\mathfrak{p}_{k+1}$,则$b=c-a_{k+1}\in
	\mathfrak{p}_{k+1}-\mathfrak{p}_{k+1}\subseteq \mathfrak{p}_{k+1}$,矛盾. 故由$c\in A\subseteq \bigcup_{i=1}^{k+1}\mathfrak{p}_i$,存在$1\leq i\leq n$使得$c\in\mathfrak{p}_i$,此时$a_{k+1}=c-b\in \mathfrak{p}_{i}-\mathfrak{p}_{i}\subseteq \mathfrak{p}_i$,矛盾. 故$m=k+1$时结论成立.
	
	综上,$m$为一切正整数时结论成立.
}

\subsection{}
有限含幺交换环的素理想必是极大理想.

\zm{
	$\mathfrak{p}$是$R$的素理想$\Leftrightarrow R/\mathfrak{p}$是整环,由$\mathfrak{p}$是真理想,$R/\mathfrak{p}$非零,又$R$有限,故$R/\mathfrak{p}$是非零有限整环,由{\heiti 习题}\textbf{3.1.11},$R/\mathfrak{p}$必为域,故$\mathfrak{p}$为极大理想.
}

\subsection{}
交换环$R$中所有素理想均包含$\mathrm{Nil}(R)$

\zm{
	对任意$a\in\mathrm{Nil}(R)$,由于素理想$\mathrm{p}$为真理想,故存在$b\in R,b\notin\mathfrak{p}$. 由于$\exists m\in \mathbb{Z}_+$使得$a^m=0$,因此$a^mb=0\in \mathfrak{p}$,若$a\notin \mathfrak{p}$,则$a^{m-1}b\in\mathfrak{p}, \cdots, a^0b=b\in\mathfrak{p}$,矛盾. 故$a\in\mathfrak{p}$,由$a$的任意性即得结论.
}

\subsection{}
设$\mathfrak{p}$是含幺交换环$R$的素理想,$I_1,...,I_n$为$R$的理想,如果$\mathfrak{p}=\bigcap_{i=1}^nI_i$,则$\mathfrak{p}$必等于某个$I_i$.

\zm{
	由{\heiti 习题}\textbf{3.2.6(1)}, $\prod_{i=1}^nI_i\subseteq \bigcap_{i=1}^nI_i=\mathfrak{p}$,有$I_n\subseteq \mathfrak{p}$或$\prod_{i=1}^{n-1}I_i\subseteq\mathfrak{p}$,继续归纳下去即得存在$i\;(k< i\leq n)$使得$I_i\subseteq\mathfrak{p}$或$\prod_{i=1}^{k}I_i\subseteq\mathfrak{p}$,取$k=0$,即结论成立或$1=\prod_{i=1}^{0}I_i\subseteq\mathfrak{p}$,后者与$\mathfrak{p}$是真理想矛盾,故结论成立.
}

\subsection{}
设$f: R\rightarrow S$是交换环的满同态,$K=\ker f$.
\subsubsection{(1)}
若$\mathfrak{p}$是$R$的素理想并且$\mathfrak{p}\supseteq K$,则$f(\mathfrak{p})$也是$S$的素理想;

\zm{
	若$cd\in f(\mathfrak{p})$,由于$f$是满射,任选$a,b\in R$使$f(a)=c,f(b)=d$,由$f(ab)=cd\in f(\mathfrak{p})$,得$ab+k\in \mathfrak{p}$,其中$k\in\ker f=K\subseteq \mathfrak{p}$,故$ab\in\mathfrak{p}$. 由$\mathfrak{p}$是素理想,$a\in\mathfrak{p}$或$b\in \mathfrak{p}$,我们有$c=f(a)\inf(\mathfrak{p})$或$d=f(b)\in f(\mathfrak{p})$.
	
	我们仍需验证$f(\mathfrak{p})$是$S$的真理想即得结论. 由同态性$f(\mathfrak{p})$对加减法自封,而$f$是满射,对任意$s\in S, c\in f(\mathfrak{p})$有$r\in R, f(r)=s, a\in \mathfrak{p}$使得$sc=f(r)f(a)=f(ta)$,其中$ra\in\mathfrak{p}$,故$sc\in f(\mathfrak{p})$,$f(\mathfrak{p})$为理想,若$f(\mathfrak{p})=S$,则$R=f^{-1}(S)+\ker f=\mathfrak{p}+K\subseteq \mathfrak{p}$,与$\mathfrak{p}$为真理想矛盾,故$f(\mathfrak{p})$是$S$的真理想.
}

\subsubsection{(2)}
若$\mathfrak{q}$是$S$的素理想,则$f^{-1}(\mathfrak{q})$也是$R$的素理想.

\zm{
	若$ab\in f^{-1}(\mathfrak{q})$,则$f(a)f(b)=f(ab)\in\mathfrak{q}$,由$\mathfrak{q}$是素理想,$f(a)\in\mathfrak{q}$或$f(b)\in\mathfrak{q}$,故$a\in f^{-1}(\mathfrak{q})$或$b\in f^{-1}(\mathfrak{q})$.
	
	我们仍需验证$f^{-1}(\mathfrak{q})$是$R$的真理想即得结论.
	由同态性$f(\mathfrak{p})$对加减法自封,对任意$r\in R, a\in f^{-1}(\mathfrak{q})$有$s=f(r)\in S, c=f(a)\in\mathfrak{q}$使得$ra=(f^{-1}(s)+k_1)(f^{-1}(c)+k_2)=f^{-1}(sc)+k_1f^{-1}(c)+k_2f^{-1}(s)+k_1k_2$其中$k_1,k_2\in K$,由于$0\in\mathfrak{q}$,故$K\subseteq f^{-1}(\mathfrak{q})$,又$sc\in\mathfrak{q}$,故$ra\in f^{-1}(\mathfrak{q})+KR+KR+KK\in f^{-1}(\mathfrak{q})$,$f^{-1}(\mathfrak{q})$是理想,若$f^{-1}(\mathfrak{q})=R$,则由$f$是满射$\mathfrak{q}=S$,与$\mathfrak{q}$为真理想矛盾,故$f^{-1}(\mathfrak{q})$是$R$的真理想.
}

\subsubsection{(3a)}
$S$中的素理想和$R$中包含$K$的素理想是一一对应的.

\zm{
	我们已经有(1)(2)的结果,我们只需验证对不同的$R$的理想$\mathfrak{p}_1\supseteq K,\mathfrak{p}_2\supseteq K$,它们的像不同即可.
	
	不妨设$a\in R, a\in \mathfrak{p}_1, a\notin \mathfrak{p}_2$,则若$f(a)\in f(\mathfrak{p}_2)$,有$b\in \mathfrak{p}_2$使得$a-b\in K$,但$K\subseteq \mathfrak{p}_2$,即$a=a-b+b\in K+ \mathfrak{p}_2\in \mathfrak{p}_2$,矛盾.
	
	对于$a\in R, a\notin \mathfrak{p}_1, a\in \mathfrak{p}_2$,交换两个理想的地位讨论即可. 故$f$诱导$R$中包含$K$的素理想到$S$中的素理想的双射.
}

\subsubsection{(3b)}
$S$中的极大理想和$R$中包含$K$的极大理想是一一对应的.

\zm{
	若$\mathfrak{p}$是$R$中包含$K$的极大理想,则$R/\mathfrak{p}$为域并且$f^{\prime}: R/\mathfrak{p}\rightarrow S/f(\mathfrak{p})$中$f^{-1}(f(\mathfrak{p}))=\mathfrak{p}$,$f(R)+f(\mathfrak{p})=S+f(\mathfrak{p})=S$,故由{\heiti 习题}\textbf{3.3.3(2)},$f^{\prime}$是良好定义的环同构,故$S/f(\mathfrak{p})$为域,$f(\mathfrak{p})$为$S$的极大理想.
	
	反之,若$\mathfrak{q}$为$S$的极大理想,则$S/\mathfrak{q}$为域且同理$g^{\prime}: R/f^{-1}\mathfrak{q}\rightarrow S/\mathfrak{q}$是良好定义的环同构,故$R/f^{-1}\mathfrak{q}$为域,$f^{-1}\mathfrak{q}$为$R$的包含$K$的极大理想.
	
	由(3a)的证明不同的理想$\mathfrak{p}_1,\mathfrak{p}_2$若均包含$K$则它们的像不同,故 故$f$诱导$R$中包含$K$的极大理想到$S$中的极大理想的双射.
}

\subsection{}
设$I$是环$R$的理想,求证$R/I$中素理想均可写成$\mathfrak{p}/I$的形式,其中$\mathfrak{p}$是$R$中包含$I$的素理想. 由此证明交换环$R$的素谱$\mathrm{Spec}R$与$R/\mathrm{Nil}(R)$的素谱$\mathrm{Spec}R/\mathrm{Nil}(R)$一一对应.

\zm{
	$\mathfrak{q}$是$R/I$的素理想$\Leftrightarrow \forall a+I, b+I\in \mathfrak{q}$有$(a+I)(b+I)=ab+aI+Ib+II=ab+I\in\mathfrak{q}\Rightarrow (a+I)\in \mathfrak{q}$或$b+I\in\mathfrak{q}$ 故$\mathfrak{p}=\{x\mid x+I\in\mathfrak{q}\}$满足$ab\in\mathfrak{p}\Rightarrow a\in\mathfrak{p}$或$b\in\mathfrak{p}$,且$0\in\mathfrak{q}\Rightarrow I\in\mathfrak{p}$,故$\mathfrak{p}$为包含$I$的素理想,且$\mathfrak{p}/I=\mathfrak{q}$.
	
	反之,$R$的包含$I$的素理想$\mathfrak{p}$满足$ab\in\mathfrak{p}\Rightarrow a\in\mathfrak{p}$或$b\in\mathfrak{p}$,即$(a+I)(b+I)=ab+aI+Ib+II\in\mathfrak{p}+I\Rightarrow (a+I)\in \mathfrak{p}+I$或$b+I\in\mathfrak{p}+I$,又$I\subseteq \mathfrak{p}$,故$\mathfrak{p}/I$是良好定义的,为$R/I$的素理想.
	
	由{\heiti 习题}\textbf{3.4.3},素谱$\mathrm{Spec}R$总满足其中元素包含理想$\mathrm{Nil}(R)$({\heiti 习题}\textbf{3.3.1}),故上述包含性可以省去,并且由{\heiti 习题}\textbf{3.4.5(3a)}的证明,$\mathrm{Spec}R$中不同元素总有在$R/\mathrm{Nil}R$中不同的像,故得结论.
}

\subsection{}
设$m\geq 2$,确定$\mathrm{Spec}(\mathbb{Z}/m\mathbb{Z})$和$\mathrm{Max}(\mathbb{Z}/m\mathbb{Z})$.

\jie 易见$\mathbb{Z}/m\mathbb{Z}$的理想只有$R_n=n\mathbb{Z}/m\mathbb{Z}$其中$n\mid m$,并且$(\mathbb{Z}/m\mathbb{Z})/R_n\cong\mathbb{Z}/n\mathbb{Z}$,它是整环(域)当且仅当$n$为素数.

故$\mathrm{Spec}(\mathbb{Z}/m\mathbb{Z})=\mathrm{Max}(\mathbb{Z}/m\mathbb{Z})
=\begin{cases}
m\mathbb{Z}/m\mathbb{Z}=\{0\}, & m\text{为素数时}\\
n\mathbb{Z}/m\mathbb{Z}, &m\text{不为素数时,其中}n\text{是}m\text{的素因子}
\end{cases}
$.

\emph{注:这里认为零环不是整环,以与素理想的定义一致,参见文献}\cite{1326600}.

\subsection{}
确定环$\mathbb{Z}[x]/(x^2+3, p)$的结构,其中$p=3,5$.

\jie 只给出结果,请读者自证.

$\mathbb{Z}[x]/(x^2+3,3)_{\Ideal}=\{a+bx\mid a,b\in\mathbb{F}_3\}$;

$\mathbb{Z}[x]/(x^2+3,5)_{\Ideal}=\mathbb{F}_5[\sqrt{2}]$.

\subsection{}
确定下面每个环的极大理想.
\subsubsection{(1)}
$\mathbb{R}\times\mathbb{R}$;

\jie 设该环的一个理想为$I$,若$(a,b)\in I$其中$a\neq 0, b\neq 0$,则$(1,1)=(a,b)(a^{-1},b^{-1})\in I$,$I=\mathbb{R}\times\mathbb{R}$,因此若$I$是真理想,必有$a$或$b$恒等于$0$,这时$I$同构于$\mathbb{R}$上的理想,后者只可能是$\{0\}$或$\mathbb{R}$,因此真理想只有三种可能:$\{0\}\times\{0\}$、$\{0\}\times\mathbb{R}$和$\mathbb{R}\times\{0\}$. 显然后两者为极大理想.

\subsubsection{(2)}
$R=\mathbb{R}[x]/(x^2)_{\Ideal}$

\jie 考虑$I\subsetneqq R$为$R$的理想,由于$x^2\equiv 0$,故$R$中元素为$x$的零次或一次式,若$ax+b\in I\;(a,b\in\mathbb{R})$,则当$b\neq 0$时$(ax+b)(-ax/b^2+1/b)=1\in I$,$I=R$矛盾,故$b=0$,若$a\neq 0$,则$I=(ax)(cx+d)=\{adx\mid d\in \mathbb{R}\}=\{a^{\prime}x\mid a^{\prime}\in\mathbb{R}\}$. 若$a=0$,则$I=\{0\}$. 故$\{ax\mid a\in\mathbb{R}\}$为$R$的极大理想.

\subsubsection{(3)}
$R=\mathbb{R}[x]/(x^2-3x+2)_{\Ideal}$.

\jie 由于$x^2\equiv 3x-2$,故$R$中元素为$x$的零次或一次式,考虑理想$I\subsetneqq R$,若$ax+b\in I\;(a,b\in\mathbb{R})$,则当$2a+b\neq 0, a+b\neq 0$时$(ax+b)(cx+d)=1, I=R$矛盾,其中当$b\neq 0$时$c=-a/(2a+b)(a+b), d=1+2ac/b$,当$b=0$时$c=-1/2a, d=3/2a$.

若$a+b=0, a\neq 0$,则$(ax-a)(cx+d)=a(2c+d)(x-1)$,$(ax-a)_{\Ideal}=\{ax-a\mid a\in R\}$. 若$ex+f\in I$并且$e+f$为非零实数,则$(ex+f)-(ex-e)=e+f\in I, 1=(e+f)(e+f)^{-1}\in I, I=R$,矛盾,故$I=\{ax-a\mid a\in R\}=(x-1)_{\Ideal}$.

若$2a+b=0, a\neq 0$,则$(ax-2a)(cx+d)=a(c+d)(x-2)$,同理可证$I=(x-2)_{\Ideal}$.

若$I$不含$a\neq 0$的元素$ax+b$,则易见$I$为真理想迫使$I=\{0\}$.

故$R$的真理想$I$为$\{0\}, (x-1)_{\Ideal}, (x-2)_{\Ideal}$,后两者是极大理想.

\subsubsection{(4)}
$R=\mathbb{R}[x]/(x^2+x+1)_{\Ideal}$.

\jie 易见$x^2+x+1$在$\mathbb{R}[x]$中不可约,此时由{\heiti 例}\textbf{3.64}它是域,只有平凡理想,故极大理想是$\{0\}$.

\subsection{}
\subsubsection{(1)}
描述环$R=\mathbb{Z}[i]/(2+i)_{\Ideal}$.

\jie 我们可证明$R\cong \mathbb{Z}/5\mathbb{Z}$.

\zm{
	令$\varphi: \mathbb{Z}\rightarrow R, n\mapsto\overline{n}$.
	由$\overline{i}=\overline{-2}$,$\varphi$为满射,若$n\in\ker\varphi$,则$\overline{n}=\overline{0}\Rightarrow (2+i)(x+yi)=n\Rightarrow x=-2y,n=-5y$,即$n\in 5\mathbb{Z}$. 反之,$n\in 5\mathbb{Z}\Rightarrow n=5x=(2+i)(2x-xi)$,故$n\in\ker\varphi$,$\ker\varphi=5\mathbb{Z}, \mathbb{Z}/\ker\varphi=\mathbb{Z}/5\mathbb{Z}$与$\im\varphi=R$同构.
}

\subsubsection{(2)}
描述环$R=\mathbb{Z}[x]/(x^2-3,2x+4)_{\Ideal}$.

\jie 我们可证明$R\cong \mathbb{Z}/2\mathbb{Z}$.

\zm{
	因$\overline{2x+4}=\overline{0}$,$\overline{x}\overline{2x+4}=\overline{2x^2+4x}=\overline{6+4x}=\overline{0}, \overline{2}=\overline{2}\overline{2x+4}-\overline{6+4x}=\overline{0}$.
	
	故$R\cong \mathbb{Z}/2\mathbb{Z}[x]/(x^2-3,2(x+2))_{\Ideal}
	\cong \mathbb{Z}/2\mathbb{Z}[x]/((x+1)^2)_{\Ideal}$,
	得$\overline{x}=\overline{1}, \overline{x+1}=\overline{0}$.
	
	故$R\cong\mathbb{Z}/2\mathbb{Z}$.
}

\subsection{}
证明{\heiti 习题}\textbf{3.1.6}中的$\mathbb{Z}_p$为PID,且仅有唯一的极大理想$p\mathbb{Z}_p$.

\zm{
	设$I$是$\mathbb{Z}_p$的理想,则$I=\{0\}\cup\bigsqcup_{i}a_i$,其中$a_i$为非零元素.
	
	当$I$不为零理想时,令$k_i$为满足$p^{k_i}$整除$a_i=(a_{i1},a_{i2},...)$各项的最大整数(因为各项中必有至少一项不为$0$,因此这是可以做到的),则$a_i=p^{k_i}b_i$,其中$b_i\notin p\mathbb{Z}_p$,由{\heiti 习题}\textbf{3.1.6(3)}知存在$c_i\in\mathbb{Z}_p$使得$b_ic_i=(1,1,1,...)$为单位元,即$(b_i)_{\Ideal}=\mathbb{Z}_p$,故$(a_i)_{\Ideal}=p^{k_i}\mathbb{Z}_p$是$a_i$生成的理想.
	
	故$I\supseteq\bigsqcup_ip^{k_i}\mathbb{Z}_p=p^{\min_i\{k_i\}}\mathbb{Z}_p$,若$I$还包含其他元素,该元素必定有项不被$p^{\min_i\{k_i\}}$整除,也就是不被所有$p^{k_i}$整除,与$I=\{0\}\cup\bigsqcup_{i}a_i$矛盾,故$I=p^k\mathbb{Z}_p$为$(0,0,...,p^k,p^k,...)=(c_1,c_2,...)$生成的理想或零理想,其中当$i\leq k$时$c_i=0$,$i>k$时$c_i=p^k$,$k=\min_i\{k_i\}\in\mathbb{N}$,故$I$是主理想.
	
	易见$k_1<k_2$时$p^{k_1}\mathbb{Z}_p\supsetneqq p^{k_2}\mathbb{Z}_p$,且$k\geq 1$时$p^k\mathbb{Z}_p$为真理想,故唯一的极大理想即$p\mathbb{Z}_p$.
}

\subsection{}
设$R$为环,$\mathfrak{m}$为$R$的一个理想,假设$R$的每一个不属于$\mathfrak{m}$的元素都是$R$中的单位,证明$\mathfrak{m}$为$R$的唯一极大理想.

\zm{
	题意并未明确排除$\mathfrak{m}=R$,但这种情况明显不合题意,以下我们认为$\mathfrak{m}$为真理想,此时$R/\mathfrak{m}$不是零环.
	
	令$a+\mathfrak{m}\in R/\mathfrak{m}$为商环中的非零元,此时$a\notin \mathfrak{m}$,它在$R$中有逆$a^{-1}$,若$a^{-1}\in\mathfrak{m}$,则$1=aa^{-1}\in\mathfrak{m}$,与$\mathfrak{m}$是真理想矛盾,故$a+\mathfrak{m}$在$R/\mathfrak{m}$中有逆$a^{-1}+\mathfrak{m}$,$R/\mathfrak{m}$是域,$\mathfrak{m}$为极大理想.
	
	设$\mathfrak{n}$为$R$的任意一个极大理想,
	若存在$a\in R, a\in\mathfrak{n}$且$a\notin\mathfrak{m}$,则$a$可逆,$1=aa^{-1}\in\mathfrak{n}R\subseteq \mathfrak{n}$,迫使$\mathfrak{n}=R$,与极大理想为真理想矛盾. 故$\mathfrak{n}\in\mathfrak{m}$,由前者的极大性质,$\mathfrak{n}=\mathfrak{m}$,故$\mathfrak{m}$为唯一的极大理想.
}

\subsection{}
设$R$为PID,$F$是它的商域,$S\supseteq R$是$F$的子环,证明$S$是PID.

\zm{
	对$R$上任意元素$a,b$,两个元素生成的理想是主理想$(c)_{\Ideal, R}$,我们将这个主理想的生成元$c$记作$\gcd(a,b)$.
	
	对$S$上任意元素$\frac{x}{y}$,$\gcd(x,y)$在$x,y$生成的$R$-理想中,故存在$m,n\in R$使得$mx+ny=\gcd(x,y)$,这时$m\frac{x}{y}+n=\frac{\gcd(x,y)}{y}\in RS+R\subseteq S$,并且$x=x^{\prime}\gcd(x,y)$,$y=y^{\prime}\gcd(x,y)$.
	
	易见$\frac{x}{y}=\frac{x^{\prime}}{y^{\prime}}$,记后者为前者的简化形式. 同时$\frac{\gcd(x,y)}{y}\in S$的简化形式为$\frac{1}{y^{\prime}}$. 如果我们把$S$中所有元素皆写为简化形式,得到引理:$\frac{x}{y}\in S\Rightarrow \frac{1}{y}\in S$.
	
	由于$R\subseteq S$,因此反过来$\frac{x}{y}\in S\Leftarrow \frac{1}{y}\in S$也成立,故$S=T^{-1}R$,其中$T$为$S$中所有出现在简化形式中的分母的集合,由引理,对任意$t_1,t_2\in T, \frac{1}{t_1}, \frac{1}{t_2}\in S$,故$\frac{1}{t_1t_2}\in S$(显然它也是简化形式),即$T$是乘法集.
	
	考虑$S$上任意理想$I=\bigsqcup \frac{a_i}{b_i}$,记$J$为所有分子$\{a_i\}$生成的理想. 由于$b_i\in R$,故对任何$a_i$有$a_i\in R$,即$J\subseteq I$,又全部分子为1的分数$T^{-1}\in S$,故$T^{-1}J\subseteq I$,且若$I$有任何其他元素,其分子不能在$J$中,矛盾. 故$J$的生成元$d\;(dR=J)$满足$dS=dT^{-1}R=T^{-1}dR=T^{-1}J=I$,即任意理想$I$都是主理想.
}

\subsection{}
设$D$为整环,$K$是$D$的商域,集合$S\subseteq D$为乘法集,即满足条件

(i) $0\notin S, 1\in S$.

(ii) 对$\forall x,y\in S, xy\in S$.

定义$$S^{-1}D=\left\{ \frac{m}{n}\mid m\in D, n\in S\right\}\subseteq K$$

试证:

\subsubsection{(1)}$S^{-1}D$是$K$中包含$D$的子环.

\Proofbyintimidation

\subsubsection{(2)} $S^{-1}D$中的素理想必有$S^{-1}\mathfrak{p}=\left\{\frac{m}{n}\mid m\in\mathfrak{p}, n\in S\right\}$的形式,其中$\mathfrak{p}$是$D$的素理想.

\zm{	
	先证$S^{-1}D$的理想$\mathfrak{q}$必有$S^{-1}A=\left\{\frac{m}{n}\mid m\in A, n\in S\right\}\;(A\subseteq D)$的形式.
	
	任取$\frac{m}{n_1}\in\mathfrak{q}$,对任意$n_2\in S$有$\frac{n_1}{n_2}\in S^{-1}D$,得$\frac{m}{n_2}=\frac{m}{n_1}\cdot\frac{n_1}{n_2}\in\mathfrak{q}$,故令$A=\left\{m\mid\frac{m}{n}\in\mathfrak{q}\right\}$,对$A$中一切元素应用上述推理即有$\mathfrak{q}\supseteq S^{-1}A$,前者若包含后者没有的元素,则与它们的定义矛盾,故两者只能一致.
	
	再证$\mathfrak{q}$是素理想时$A$为$D$的素理想. 对任意$m_1,m_2\in D$,若$m_1m_2\in A$,则$\frac{m_1m_2}{1}\in\mathfrak{q}$,故由$\mathfrak{q}$为素理想,$\frac{m_1}{1}\in\mathfrak{q}$或$\frac{m_2}{1}\in\mathfrak{q}$,得$m_1\in A$或$m_2\in A$. 
}

\subsubsection{(3a)}
$\mathrm{Spec}(S^{-1}D)$与集合$\{\mathfrak{p}\in\mathrm{Spec}D\mid p\cap S=\varnothing\}$一一对应.

\zm{
	$\forall \mathfrak{q}\in\mathrm{Spec}(S^{-1}D)$,由$(2)$有$\mathfrak{p}=\{m_1\mid\frac{m_1}{n_1}\in\mathfrak{q}\}\in\mathrm{Spec}D$,若$\mathfrak{p}\cap S\neq\varnothing$,取$n_p\in \mathfrak{p}\cap S$,则$1=\frac{n_p}{n_p}\in \mathfrak{q}$,与$\mathfrak{q}$是真理想矛盾,故$\mathfrak{p}\cap S=\varnothing$,并且显然$\mathfrak{p}$由$\mathfrak{q}$唯一确定.
	
	另一方面,给定$\mathfrak{p}\in\mathrm{Spec}D$且$\mathfrak{p}\cap S=\varnothing$,请读者自证$S^{-1}\mathfrak{p}$是$S^{-1}D$的理想,且$\frac{m_1}{n_1}\cdot\frac{m_2}{n_2}\in S^{-1}\mathfrak{p}\Leftrightarrow m_1m_2\in\mathfrak{p}\Leftrightarrow m_1\in\mathfrak{p}$或$m_2\in\mathfrak{p}\Leftrightarrow \frac{m_1}{n_1}\in S^{-1}\mathfrak{p}$或$\frac{m_2}{n_2}\in S^{-1}\mathfrak{p}$,且$\mathfrak{p}\cap S=\varnothing\Rightarrow 1=\frac{u}{u}\notin \mathfrak{p}$,$S^{-1}\mathfrak{p}$为真理想且满足素理想的条件.
	
	我们只需证明$\mathfrak{p}$唯一确定$\mathfrak{q}=S^{-1}\mathfrak{p}$即可. 因$\mathfrak{p}$是素理想,任何$\mathfrak{q}$中元素$\frac{m}{n}, m\in\mathfrak{p}$中元素若有等价形式$\frac{m^{\prime}}{n^{\prime}}$,则$mn^{\prime}=nm^{\prime}$,左边属于$\mathfrak{p}$,右边有$n\notin\mathfrak{p}$,故$m^{\prime}\in\mathfrak{p}$,即$\mathfrak{q}$中所有元素的分子的集合与$\mathfrak{p}$完全相等,故不同的$\mathfrak{p}$确定不同的$\mathfrak{q}$.
}

\subsubsection{(3b)}
举例说明若不要求两者为素理想,$S^{-1}D$上的理想可以不与$D$上不与$S$相交的理想一一对应,并说明(3a)的证明不再成立的理由.

\jie 令$D=\mathbb{Z}$,$S=\mathbb{Z}-3\mathbb{Z}$,则$D$上的理想$3\mathbb{Z}$与$6\mathbb{Z}$确定同样的$S^{-1}\mathfrak{p}$. 在上节的证明中,$\mathfrak{p}$不为素理想导致$m^{\prime}\in\mathfrak{p}$不一定成立.

\subsubsection{(4)}
设$D=\mathbb{Z}, \mathfrak{p}=p\mathbb{Z}, S=\mathbb{Z}-\mathfrak{p}$,试证$\mathbb{Z}/\mathfrak{p}=\mathbb{Z}/p\mathbb{Z}\cong S^{-1}\mathbb{Z}/S^{-1}\mathfrak{p}$.

\zm{
	对任意$\frac{m}{n}\in S^{-1}\mathbb{Z}/S^{-1}\mathfrak{p}$,由于$n\notin\mathfrak{p}$,故$n$在$\mathbb{Z}/p\mathbb{Z}$中可逆,记$n^{-1}\in\mathbb{Z}-\mathfrak{p}$满足$nn^{-1}\equiv 1\mod p$,则$\frac{m}{n}-n^{-1}m=\frac{m-(ps+1)m}{n}, s\in\mathbb{Z}=-\frac{psm}{n}\in S^{-1}\mathfrak{p}$. 故$\frac{m}{n}$和$n^{-1}m$在$S^{-1}\mathbb{Z}/S^{-1}\mathfrak{p}$中总表示同一元素.
	定义$f: S^{-1}\mathbb{Z}\rightarrow \mathbb{Z}/p\mathbb{Z}, \frac{m}{n}\mapsto n^{-1}m$,则$n^{-1}m\equiv 0\mod p\Rightarrow m\equiv 0\mod p$,故$\ker f=S^{-1}\mathfrak{p}$,由环同态基本定理即得同构.
}

\subsubsection{(5)}
设$\mathfrak{p}$是$D$的素理想,$S=D-\mathfrak{p}$,问何时$D/\mathfrak{p}$同构于$S^{-1}D/S^{-1}\mathfrak{p}$.

\jie $S/\mathfrak{p}$中一切元素在$D/\mathfrak{p}$中有同构. 该条件的充分性请读者类似上小节证明,下证必要性.

若$\overline{n}$在$D/\mathfrak{p}$中不可逆,由于$n\notin\mathfrak{p}$知$\frac{n}{1}$在$S^{-1}D/S^{-1}\mathfrak{p}$中不等于零,$\frac{1}{n}\cdot\frac{n}{1}=\frac{1}{1}$得$\frac{n}{1}$在$S^{-1}D/S^{-1}$中可逆,与同构性矛盾.

\subsection{}
设$R$为整环,$F$为$R$的分式域,$\mathfrak{p}$为$R$的素理想,$R$在$\mathfrak{p}$处的局部化$R_{\mathfrak{p}}$是指$F$的子环$(R-\mathfrak{p})^{-1}R=\{\frac{a}{d}\mid a,d\in R, d\notin\mathfrak{p}\}$试确定$R_{\mathfrak{p}}$的所有极大理想.

\jie 记$(R-\mathfrak{p})=S$,由{\heiti 习题}\textbf{3.4.14(2)},$R_{\mathfrak{p}}$的一切素理想均有形式$S^{-1}\mathfrak{q}$,其中$\mathfrak{q}$为$R$的素理想. 由于极大理想均是素理想,因此也有此形式.

若$\exists u\in\mathfrak{q}, u\notin\mathfrak{p}$,则$1=\frac{u}{u}\in S^{-1}\mathfrak{q}$与$S^{-1}\mathfrak{q}$为真理想矛盾. 故$\mathfrak{q}\subseteq\mathfrak{p}$.

若$\exists a\in\mathfrak{p}, a\notin\mathfrak{q}$则考虑$\frac{a}{d}+S^{-1}\mathfrak{q}$为$S^{-1}R/S^{-1}\mathfrak{q}$中非零元,由$S^{-1}\mathfrak{q}$为极大理想,$S^{-1}R/S^{-1}\mathfrak{q}$是域,故该非零元可逆,存在$\frac{b}{c}\neq 0\in S^{-1}R/S^{-1}\mathfrak{q}, \frac{e}{f}=0\in S^{-1}R/S^{-1}\mathfrak{q}$,满足$c,e\notin\mathfrak{p}, b\notin\mathfrak{q}, f\in\mathfrak{q}$使得

$$\frac{a}{d}\cdot\frac{b}{c}=\frac{1}{1}+\frac{f}{e}\;\text{即}\;abe=edc+fdc, abe-fdc=edc.$$

由$a\in\mathfrak{p}, f\in\mathfrak{q}\subseteq\mathfrak{p}$知$edc=abe-fdc\in\mathfrak{p}$,由$\mathfrak{p}$为素理想,$e$或$d$或$c\in\mathfrak{p}$,矛盾. 故$\mathfrak{p}\subseteq \mathfrak{q}$.

故只能$\mathfrak{q}=\mathfrak{p}$,易验证$S^{-1}R/S^{-1}\mathfrak{p}$是域,故$S^{-1}R$的极大理想只有$S^{-1}\mathfrak{p}$.